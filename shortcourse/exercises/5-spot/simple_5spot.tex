\documentclass[12pt]{article} 
%\usepackage{times,helvet}
\usepackage{fourier}
%\usepackage{ssss}
\usepackage{amsmath,amsbsy,amssymb}
%\usepackage{fancyheadings}
\usepackage{fancyhdr}
\usepackage{tabularx}
\usepackage{verbatim}
\usepackage{moreverb}
\usepackage{float}
%\usepackage{deflist}
\usepackage{graphicx}
\usepackage{longtable}
\usepackage{portland}
\usepackage{booktabs}
\usepackage{comment}

%\usepackage{multimedia}
\usepackage{movie15}

\usepackage[debug=false, colorlinks=true, pdfstartview=FitV, linkcolor=blue, citecolor=black, urlcolor=blue]{hyperref}

\textwidth 6.5in
\textheight 9.5in
%\topmargin -.1in
\topmargin -.75in
\newlength{\boxwidth}
\setlength{\boxwidth}{5.8in}
\oddsidemargin -0in
\evensidemargin -0in
\headheight 0.25in
\lhead{{\sl PFLOTRAN Shortcourse: Copper Leaching}}
\chead{\rm - \thepage\ -}
\rhead{\today}
\cfoot{}
\newcommand\flotran{{\sl FLOTRAN}}
\renewcommand{\baselinestretch}{1.2}
\def\EQ#1\EN{\begin{equation}#1\end{equation}}
\def\BA#1\EA{\begin{align}#1\end{align}}
\def\BS#1\ES{\begin{split}#1\end{split}}
%\newcommand{\EQ}{\begin{equation}}
%\newcommand{\EN}{\end{equation}}
\newcommand{\bc}{\begin{center}}
\newcommand{\ec}{\end{center}}
\newcommand{\eq}{\ =\ }
\newcommand{\degc}{$^\circ$C}
\newcommand{\ecm}{{\rm ecm}}
\newcommand{\eff}{{\rm eff}}
\newcommand{\eqr}{{\rm le}}
\newcommand{\equ}{{\rm eq}}
\newcommand{\kin}{{\rm kin}}
\newcommand{\rdx}{{\rm rdx}}
\newcommand{\ind}{{\rm id}}
\newcommand{\dep}{{\rm dp}}
\newcommand{\e}{{\rm{e}}}
\newcommand{\erf}{{\rm{erf}}}
\newcommand{\erfc}{{\rm{erfc}}}
\newcommand{\p}{{\partial}}
\newcommand{\A}{{\mathcal A}}
\newcommand{\B}{{\mathcal B}}
\newcommand{\C}{{\mathcal C}}
\newcommand{\D}{{\mathcal D}}
\newcommand{\E}{{\mathcal E}}
\newcommand{\F}{{\mathcal F}}
\newcommand{\G}{{\mathcal G}}
\newcommand{\J}{{\mathcal J}}
\newcommand{\M}{{\mathcal M}}
\newcommand{\cO}{{\mathcal O}}
\renewcommand{\P}{{{\mathcal P}}}
\newcommand{\Q}{{\mathcal Q}}
\newcommand{\R}{{{\mathcal R}}}
\renewcommand{\S}{{\mathcal S}}
\newcommand{\T}{{\mathcal T}}
\newcommand{\W}{{\mathcal W}}
\newcommand{\Y}{{\mathcal Y}}
\newcommand{\Z}{{\mathcal Z}}
\newcommand{\rev}{{\rm rev}}
\newcommand{\irr}{{\rm irr}}
\renewcommand{\a}{{\alpha}}
\newcommand{\abar}{{\bar \alpha}}
\renewcommand{\b}{{\beta}}
\newcommand{\w}{{\rm H_2O}}
\newcommand{\air}{{\rm N_2}}
\newcommand{\pe}{{\rm Pe}}
\newcommand{\da}{{\rm Da}}
\renewcommand{\k}{{\dot R}^0}
\renewcommand{\L}{\widehat{\mathcal L}}
\renewcommand{\bar}{\overline}
\newcommand{\dsty}{{\displaystyle}}
\newcommand{\diff}{{\mathcal D}}
\newcommand{\surf}{\equiv \!\!\!}
\newcommand{\bnabla}{\boldsymbol{\nabla}}
\newcommand{\bA}{\boldsymbol{A}}
\newcommand{\ba}{\boldsymbol{a}}
\newcommand{\bB}{\boldsymbol{B}}
\newcommand{\bC}{\boldsymbol{C}}
\newcommand{\bE}{\boldsymbol{E}}
\newcommand{\bi}{\boldsymbol{i}}
\newcommand{\bI}{\boldsymbol{I}}
\newcommand{\bJ}{\boldsymbol{J}}
\newcommand{\bK}{\boldsymbol{K}}
\newcommand{\bM}{\boldsymbol{M}}
\newcommand{\bGamma}{\boldsymbol{\Gamma}}
\newcommand{\bOmega}{\boldsymbol{\Omega}}
\newcommand{\bPsi}{\boldsymbol{\Psi}}
\newcommand{\bO}{\boldsymbol{O}}
\newcommand{\bnu}{\boldsymbol{\nu}}
\newcommand{\bdS}{\boldsymbol{dS}}
\newcommand{\bq}{\boldsymbol{q}}
\newcommand{\br}{\boldsymbol{r}}
\newcommand{\bR}{\boldsymbol{R}}
\newcommand{\bS}{\boldsymbol{S}}
\newcommand{\bu}{\boldsymbol{u}}
\newcommand{\bv}{\boldsymbol{v}}
\newcommand{\bz}{\boldsymbol{z}}
\newcommand{\arrows}{~\rightleftharpoons~}
\newcommand{\arrowstab}{\!\!\!\rightleftharpoons\!\!\!}
\newcommand{\longline}{\noindent\rule[-0.1in]{\textwidth}{0.01in}}

%\numberwithin{equation}{section}
%\renewcommand{\theequation}{\arabic{section}.\arabic{equation}}

%\newcounter{saveeqn}%
%\newcommand{\seteqn}{\setcounter{saveeqn}{\value{equation}}%
%\stepcounter{saveeqn}\setcounter{equation}{0}%
%\renewcommand{\theequation}
%      {\mbox{\arabic{saveeqn}-\alph{equation}}}}%
%\newcommand{\reseteqn}{\setcounter{equation}{\value{saveeqn}}%
%\renewcommand{\theequation}{\arabic{equation}}}%

\newcounter{saveeqn}%
\newcommand{\seteqn}{\setcounter{saveeqn}{\value{equation}}%
\stepcounter{saveeqn}\setcounter{equation}{0}%
\renewcommand{\theequation}
      {\mbox{\arabic{section}.\arabic{saveeqn}\alph{equation}}}}%
\newcommand{\reseteqn}{\setcounter{equation}{\value{saveeqn}}%
\renewcommand{\theequation}{\arabic{section}.\arabic{equation}}}%

\setcounter{secnumdepth}{5}
\setcounter{tocdepth}{5}

\setlength{\parindent}{0.3125in}
\setlength{\parskip}{2ex plus 0.2ex minus 0.2ex}

\renewcommand{\contentsname}{TABLE OF CONTENTS}
\setcounter{secnumdepth}{5}

\setlongtables

\pagestyle{fancy}

\thispagestyle{empty}

\begin{document}

\section*{Five-Spot Copper Leaching Reactive Transport Calculation}

Five-Spot Copper Leaching Notes: PCL (LANL) \hfill \today

\section{Introduction}

In this problem leaching of copper from a porphyry copper deposit containing chrysocolla (CuSiO$_3 \cdot {\rm H_2O}_{\rm (s)}$) ore is modeled. A lixiviant consisting of sulfuric acid with pH 1 is introduced into the deposit dissolving the chrysocolla. A quarter symmetry element of a five-spot well pattern is considered with an injection well in one corner of the domain and an extraction well in the opposite corner.

\section{Chemical Reactions}

Primary Species: Cu$^{2+}$, SO$_4^-$, H$^+$, SiO$_{2(aq)}$

\noindent
Secondary Species: OH$^-$, HSO$_4^-$, H$_2$SO$_{\rm 4(aq)}$, CuOH$^+$, CuSO$_{\rm 4(aq)}$

\noindent Solid: CuSiO$_3 \cdot {\rm H_2O}_{\rm (s)}$

\begin{comment}
\begin{center}
\begin{tabular}{rlr}
\toprule
Reaction && log $K$ (25\degc) \\
\midrule
$\rm H_2O - H^+ $ &$\arrows$ $\rm OH^-$ & $-$13.9951\\
$\rm HCO_3^- - H^+$ &$\arrows$ $\rm CO_3^{2-}$ & $-$10.3288\\
$\rm HCO_3^- + H^+ - H_2O$ &$\arrows$ $\rm CO_{2(aq)}$ & 6.3447\\
$\rm Ca^{2+} + HCO_3^- - H^+$ &$\arrows$ $\rm CaCO_{3(aq)}$ & $-$7.0017\\
$\rm Ca^{2+} + HCO_3^- $ &$\arrows$ $\rm CaHCO_3^+$ & 1.0467\\
\midrule
$\rm Ca^{2+} + HCO_3^- - H^+$ &$\arrows$ $\rm CaCO_{3(s)}$ & $-$1.8487\\
\bottomrule
\end{tabular}
\end{center}
\end{comment}

\noindent
For the general reaction, 
\EQ
\sum_j\nu_{ji}\A_j \arrows \A_i,
\EN
the equilibrium constant is defined as
\EQ
K_i \eq \dfrac{a_i}{\displaystyle\prod_j \big(a_j\big)^{\nu_{ji}}},
\EN
where $a_l$ refers to the activity of species $\A_l$.

\noindent
In the formulation that follows the activity of H$_2$O is assumed to be one and therefore is not considered in the mass balance equations.

\section{Mass Conservation Equations}

The flow field in the five-spot well field is described by the mass conservation equation
\EQ
\frac{\p}{\p t} \varphi \rho + \bnabla\cdot\bq\rho \eq Q,
\EN
where $\rho$ denotes the molar fluid density, $\varphi$ refers to the porosity, $Q$ is a source/sink term for injection and extraction of fluid, and $\bq$ denotes the Darcy velocity defined as
\EQ
\bq \eq -\frac{k}{\mu}\big(\bnabla p-W\rho g \bz\big),
\EN
where $p$ denotes the fluid pressure, $W$ refers to the formula weight of water, g is the acceleration of gravity, $k$ is permeability, $\mu$ the fluid viscosity,  and 
$\bz$ 
a unit vector.

Mass conservation equations for reaction with chrysocolla have the following form for aqueous primary species
\EQ
\frac{\p}{\p t} \varphi\Psi_j + \bnabla\cdot\bOmega_j \eq \R_j + Q_j,
\EN
and for the $m$th mineral
\EQ
\frac{\p\phi_{m}}{\p t} \eq I_{m},
\EN
where $\phi_{m}$ denotes the mineral volume fraction, $\R_j$ represents the rate of reaction of the subscripted primary species, $Q_j$ denotes a source/sink term, and $I_{m}$ denotes the rate of dissolution and precipitation.
The source/sink terms $Q$ and $Q_j$ are related by the equation
\EQ
Q_j \eq Q \Psi_j.
\EN

The total primary species concentrations $\Psi_j$ are defined as
\EQ
\Psi_j \eq C_j + \sum_i\nu_{ji}C_i,
\EN

\begin{comment}
\begin{subequations}
\BA
\Psi_{\rm Ca^{2+}} &\eq C_{\rm Ca^{2+}} + C_{\rm CaCO_{3(aq)}} + C_{\rm CaHCO_3^-},\\
\Psi_{\rm HCO_3^-} &\eq C_{\rm HCO_3^-} + C_{\rm CO_3^{2-}} + C_{\rm CO_{2(aq)}} + C_{\rm CaCO_{3(aq)}} + C_{\rm CaHCO_3^-} ,\\
\Psi_{\rm H^+} &\eq C_{\rm H^+} - C_{\rm OH^-}
\EA
\end{subequations}
\end{comment}

\noindent
where $C_i$ denotes the aqueous concentration of the subscripted species.
Total primary species flux $\bOmega_j$ is defined as:
\EQ
\bOmega_j \eq \big(\bq - \varphi D \bnabla\big)\Psi_j,
\EN
with $\bq$ the Darcy velocity, and $D$ diffusion/dispersion coefficient. This relation is valid provided the diffusion coefficient $D$ is species independent.

\section{Kinetic Rate Law}

The reaction rate $\R_j$ is related to the mineral reaction rates according to
\EQ
\R_j \eq -\sum_m\nu_{jm} I_m,
\EN
where the sum over $m$ includes the primary mineral chrysocolla and other secondary products.

\noindent
The mineral reaction rate $I_m$ is
taken to have the form given by transition state theory
\EQ
I_{m} \eq - k_{m} a_{m} \big(1-K_{m}Q_{m}\big)\zeta_{m},
\EN
with equilibrium constant $K_{m}$, specific surface area $a_{m}$, and ion activity product $Q_{m}$ defined by
\EQ
Q_{m} \eq K_m \prod_j\big(\gamma_j C_j\big)^{\nu_{jm}}.
\EN
The factor $\zeta_{m}$ takes on the values zero or one according to
\EQ
\zeta_{m} \eq \left\{
\begin{array}{ll}
1, & \phi_{m} > 0 \ \text{or} \ K_{m}Q_{m}>1,\\
0, & \text{otherwise}.
\end{array}
\right.
\EN
The factor $\zeta_{m}$ ensures that the mineral does not dissolve if it is not present. The activity $a_i = \gamma_i C_i$ is related to concentration by the activity coefficient $\gamma_i$. According to the sign convention used here, $I_{m}$ is positive for precipitation and negative for dissolution.

In general the specific surface area is a function of the mineral concentration. One such relation is to assume the power law
\EQ
a_{m} \eq a_{m}^0 \left(\frac{\phi_{m}}{\phi_{m}^0}\right)^n,
\EN
where $a_{m}^0$ and $\phi_{m}^0$ denote the initial surface area and mineral volume fraction, and $n$ is an exponent typically with the value 2/3 reflecting the surface to volume ratio.

In general the permeability of the porous medium may also be related to the mineral volume fraction through a similar relation.

\section{Example Calculation}

A 2D problem with a domain measuring 15 m on a side and with injection and extraction wells located at opposite corners is considered. No flow boundary conditions are imposed on the sides of the domain. A solution with pH 1 is injected into the porous medium which contains the ore mineral chrysocolla. The initial volume fraction of chrysocolla used in the simulation is 0.005, with an effective rate constant of $5\times 10^{-12}$ mol/cm$^3$/s.
Results for the chrysocolla volume fraction and pH are shown in the embedded
movies (1) and (2).
%Figures~\ref{fchry} and \ref{fph}.
As can be seen in the movies a quasi-stationary state is established with the time evolution of the chrysocolla dissolution front form a sequence of stationary states (Lichtner, 1988).

\setcounter{figure}{0}
\begin{center}
\begin{figure}[ht]
\label{fchry}
\includemovie[
text={\bf\color{blue} Movie 1: Chrysocolla}
]{6in}{6in}{./figs/5spot_simple_chry.mpeg}
\end{figure}
\end{center}
%\bf\color{blue} Movie 1: Chrysocolla Volume Fraction}

\addtocounter{figure}{2}

\begin{center}
\begin{figure}[ht]
\includemovie[
text={\bf\color{blue} Movie 2: pH}
]{6in}{6in}{./figs/5spot_simple_ph.mpeg}
\label{fph}
\end{figure}
\end{center}


\end{document}

\movie[showcontrols, scale=0.15]
{\includegraphics[scale=0.15]{./figs/5spot-chry.jpeg}}
{./figs/5spot_simple_chry.mpeg}
