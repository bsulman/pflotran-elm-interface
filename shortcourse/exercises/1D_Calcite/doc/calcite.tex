\documentclass{beamer}

\usepackage{comment}
\usepackage{color}
\usepackage{listings}
\usepackage{verbatim}
\usepackage{multicol}
\usepackage{booktabs}
\definecolor{green}{RGB}{0,128,0}

\def\EQ#1\EN{\begin{equation*}#1\end{equation*}}
\def\BA#1\EA{\begin{align*}#1\end{align*}}
\def\BS#1\ES{\begin{split*}#1\end{split*}}
\newcommand{\bc}{\begin{center}}
\newcommand{\ec}{\end{center}}
\newcommand{\eq}{\ =\ }
\newcommand{\degc}{$^\circ$C}

\def\p{\partial}
\def\qbs{\boldsymbol{q}}
\def\Dbs{\boldsymbol{D}}
\def\A{\mathcal A}
\def\gC{\mathcal C}
\def\gD{\mathcal D}
\def\gL{\mathcal L}
\def\M{\mathcal M}
\def\P{\mathcal P}
\def\Q{\mathcal Q}
\def\gR{\mathcal R}
\def\gS{\mathcal S}
\def\X{\mathcal X}
\def\bnabla{\boldsymbol{\nabla}}
\def\bnu{\boldsymbol{\nu}}
\renewcommand{\a}{{\alpha}}
%\renewcommand{\a}{{}}
\newcommand{\s}{{\sigma}}
\newcommand{\bq}{\boldsymbol{q}}
\newcommand{\bz}{\boldsymbol{z}}
\def\bPsi{\boldsymbol{\Psi}}

\def\Li{\textit{L}}
\def\Fb{\textbf{f}}
\def\Jb{\textbf{J}}
\def\cb{\textbf{c}}

\def\Dt{\Delta t}
\def\tpdt{{t + \Delta t}}
\def\bpsi{\boldsymbol{\psi}}
\def\dbpsi{\delta \boldsymbol{\psi}}
\def\bc{\textbf{c}}
\def\dbc{\delta \textbf{c}}
\def\arrows{\rightleftharpoons}

\newcommand{\bGamma}{\boldsymbol{\Gamma}}
\newcommand{\bOmega}{\boldsymbol{\Omega}}
%\newcommand{\bPsi}{\boldsymbol{\Psi}}
%\newcommand{\bpsi}{\boldsymbol{\psi}}
\newcommand{\bO}{\boldsymbol{O}}
%\newcommand{\bnu}{\boldsymbol{\nu}}
\newcommand{\bdS}{\boldsymbol{dS}}
\newcommand{\bg}{\boldsymbol{g}}
\newcommand{\bk}{\boldsymbol{k}}
%\newcommand{\bq}{\boldsymbol{q}}
\newcommand{\br}{\boldsymbol{r}}
\newcommand{\bR}{\boldsymbol{R}}
\newcommand{\bS}{\boldsymbol{S}}
\newcommand{\bu}{\boldsymbol{u}}
\newcommand{\bv}{\boldsymbol{v}}
%\newcommand{\bz}{\boldsymbol{z}}
\newcommand{\pressure}{P}

\def\water{H$_2$O}
\def\calcium{Ca$^{2+}$}
\def\copper{Cu$^{2+}$}
\def\magnesium{Mg$^{2+}$}
\def\sodium{Na$^+$}
\def\potassium{K$^+$}
\def\uranium{UO$_2^{2+}$}
\def\hion{H$^+$}
\def\hydroxide{0H$^-$}
\def\bicarbonate{HCO$_3^-$}
\def\carbonate{CO$_3^{2-}$}
\def\cotwo{CO$_2$(aq)}
\def\chloride{Cl$^-$}
\def\fluoride{F$^-$}
\def\phosphoricacid{HPO$_4^{2-}$}
\def\nitrate{NO$_3^-$}
\def\sulfate{SO$_4^{2-}$}
\def\souotwooh{$>$SOUO$_2$OH}
\def\sohuotwocothree{$>$SOHUO$_2$CO$_3$}
\def\soh{$>$SOH}

\newcommand\gehcomment[1]{{{\color{orange} #1}}}
\newcommand\add[1]{{{\color{blue} #1}}}
\newcommand\remove[1]{\sout{{\color{red} #1}}}
\newcommand\codecomment[1]{{{\color{green} #1}}}
\newcommand\redcomment[1]{{{\color{red} #1}}}
\newcommand\bluecomment[1]{{{\color{blue} #1}}}
\newcommand\greencomment[1]{{{\color{green} #1}}}
\newcommand\magentacomment[1]{{{\color{magenta} #1}}}

\begin{comment}
\tiny
\scriptsize
\footnotesize
\small
\normalsize
\large
\Large
\LARGE
\huge
\Huge
\end{comment}

\begin{document}
\title{1D Calcite Scenario\ldots in a Nutshell}
\author{Glenn Hammond}
\date{\today}

%\frame{\titlepage}

%-----------------------------------------------------------------------------
\section{Description of 1D Calcite Scenario}

\subsection{1D Calcite Conceptual Model}

\frame{\frametitle{Description of 1D Calcite Scenario}
The ``1D Calcite Scenario'' simulates solute transport (without flow) along a horizontal, 1D domain measuring 100 meters in length with unit cross sectional area.  Assumptions with regard to transport include:
\begin{itemize}
  \item Problem domain: $100 \times 1 \times 1$ m (x $\times$ y $\times$ z)
  \item Grid resolution $1 \times 1 \times 1$ m
  \item Darcy flow velocity: 1 m/y (left to right or west to east)
  \item Porosity: 0.25
  \item Maximum time step size: 0.25 y (CFL = 1.)
  \item Total simulation time: 25 y
\end{itemize}
}

%-----------------------------------------------------------------------------
\subsection{Governing Equations
}
\frame{\frametitle{Governing Reactive Transport Equations}

\Large

\EQ\label{trans}
\frac{\p}{\p t} \left(\varphi s \Psi_j + \Psi_{S,j}\right) + \bnabla\cdot\bOmega_j \eq
\gR\left(c_1,c_2,\ldots,c_n\right)
\EN

\EQ\label{flux}
\bOmega_j \eq \big(\bq - \varphi s \Dbs\bnabla\big) \Psi_j
\EN

%\bigskip
%\normalsize
\footnotesize
\begin{align*}
\varphi &\eq \text{porosity}; \quad
s \eq \text{liquid saturation}\\
\Psi_j &\eq \text{total component concentration for aqueous species } j\\
c_j &\eq \text{free ion concentration for aqueous species } j\\
\bOmega &\eq \text{solute flux}; \quad
\bq \eq \text{Darcy velocity} \\
\Dbs &\eq \text{hydrodynamic dispersion} \eq\gD^\ast + \alpha_L|\bnu|\\
\gD^\ast &\eq \text{species {\color{red} independent} coefficient of diffusion}\\
\alpha_L &\eq \text{longitudinal dispersity}\\
\bnu &\eq \text{pore water velocity} \eq \bq/\varphi
\end{align*}

}

%-----------------------------------------------------------------------------
\frame{\frametitle{Reaction Equations \\{\large 1D Calcite}}

\large

\begin{align*}
\chi_i &\eq \frac{K_i}{\gamma_i} \prod_j \big(\gamma_j c_j\big)^{\nu_{ji}^{\rm aq}} \\
\Psi_j &\eq c_j + \sum_i \nu_{ji}^{\rm aq} \chi_i \eq c_j + \sum_i \nu_{ji}^{\rm aq} \frac{K_i}{\gamma_i} \prod_j \big(\gamma_j c_j\big)^{\nu_{ji}^{\rm aq}}\\
I_j &\eq -\sum_m\nu_{jm}^{\rm min} A_m k_m \left[1-K_m \prod_j \big(\gamma_j c_j\big)^{\nu_{jm}^{\rm min}}\right]
\end{align*}

%\footnotesize
\scriptsize
\begin{align*}
c_j &\eq \text{free ion concentration for aqueous species } j\\
\gamma_j &\eq \text{activity coefficient for aqueous species } j\\
\chi_i &\eq \text{secondary aqueous complex concentration } i\\
\nu_{ji} &\eq \text{stoichiometry of component $j$ in reaction } i\\
I_j &\eq \text{mineral reaction rate for species } j
\end{align*}

}

%-----------------------------------------------------------------------------
\section{Description of Input Deck: Transport Only}


\subsection{GRID}
\begin{frame}[fragile,containsverbatim]\frametitle{GRID}

\begin{itemize}
  \item Problem domain: $100 \times 1 \times 1$ m (x $\times$ y $\times$ z)
  \item Grid resolution $1 \times 1 \times 1$ m
\end{itemize}

\begin{semiverbatim}
GRID
  TYPE structured     \bluecomment{! structured grid}
  NXYZ 100 1 1        \bluecomment{! NX, NY, NZ}
  BOUNDS              \bluecomment{! define the rectangular domain}
    0.d0 0.d0 0.d0    \bluecomment{! xmin ymin zmin}
    100.d0 1.d0 1.d0  \bluecomment{! xmax ymax zmax}
  /  \bluecomment{! <-- closes out BOUNDS card}
END  \bluecomment{! <-- closes out GRID card}
\end{semiverbatim}

\end{frame}

%-----------------------------------------------------------------------------
\subsection{REGION}

\begin{frame}[fragile,containsverbatim,allowframebreaks]\frametitle{REGION}

\begin{itemize}
  \item Delineate regions in the 1D domain for:
  \begin{itemize}
    \item west boundary face
    \item east boundary face
    \item entire domain (all)
  \end{itemize}
\end{itemize}

\begin{semiverbatim}
REGION all            \bluecomment{! define a region and name it: \greencomment{all}}
  COORDINATES         \bluecomment{! using \redcomment{volumetric} coordinates}
    0.d0 0.d0 0.d0    \bluecomment{! lower coordinate: xmin ymin zmin}
    100.d0 1.d0 1.d0  \bluecomment{! upper coordinate: xmax ymax zmax}
  /   \bluecomment{! <-- closes out COORDINATES card}
END   \bluecomment{! <-- closes out REGION card}

\newpage
REGION west           \bluecomment{! define region:} \greencomment{west}
  FACE west           \bluecomment{! define the face of the grid cell}
  COORDINATES         \bluecomment{! using \redcomment{surface} coordinates}
    0.d0 0.d0 0.d0
    0.d0 1.d0 1.d0
  /
END

REGION east           \bluecomment{! define region:} \greencomment{east}
  FACE east           \redcomment{! west, east, south, north,}
  COORDINATES         \redcomment{!   bottom, top} \bluecomment{ are keywords}
    100.d0 0.d0 0.d0  \bluecomment{!   in PFLOTRAN.}
    100.d0 1.d0 1.d0
  /
END

\end{semiverbatim}

\end{frame}

%-----------------------------------------------------------------------------
\subsection{MATERIAL\_PROPERTY}

\begin{frame}[fragile,containsverbatim]\frametitle{MATERIAL\_PROPERTY}

\begin{itemize}
  \item Define a soil with :
  \begin{itemize}
    \item Material id = 1
    \item Porosity = 0.25
    \item Tortuosity = 1.
  \end{itemize}
\end{itemize}

\begin{semiverbatim}
MATERIAL_PROPERTY soil1  \bluecomment{! define a material:} \greencomment{soil1}
  ID 1                   \bluecomment{! All grid cells of this}
  POROSITY 0.25d0        \bluecomment{!   material type will have}
  TORTUOSITY 1.d0        \bluecomment{!   a material \redcomment{ID = 1}.}
END  \bluecomment{! <-- closes out MATERIAL\_PROPERTY card}
\end{semiverbatim}

\end{frame}

%-----------------------------------------------------------------------------
\subsection{FLUID\_PROPERTY}

\begin{frame}[fragile,containsverbatim]\frametitle{FLUID\_PROPERTY}

\begin{itemize}
  \item Assign a molecular diffusion coefficient of $10^{-9}$ m$^2$/s to all aqueous species
\end{itemize}

\begin{semiverbatim}

FLUID_PROPERTY                  \bluecomment{! fluid is water}
  DIFFUSION_COEFFICIENT 1.d-9   \bluecomment{! [m^2/s]}
END  \bluecomment{! <-- closes out FLUID\_PROPERTY card}
\end{semiverbatim}

\end{frame}

%-----------------------------------------------------------------------------
\subsection{CHEMISTRY}

\begin{frame}[fragile,allowframebreaks]\frametitle{CHEMISTRY}

\begin{itemize}
\item For Calcite dissolution geochemistry, specify:
  \begin{itemize}
    \item Three primary species: \hion, \bicarbonate and \calcium
    \item Six secondary aqueous complexes: \hydroxide, \carbonate, \cotwo, CaCO$_3$(aq), CaHCO$_3^+$, and CaOH$^+$
    \item One mineral: Calcite
    \item A kinetic rate for Calcite dissolution: 1.e-12 [1/s]
    \item The name of a database for species parameters (e.g. molecular weight, charge) and equilibrium constants
  \end{itemize}
\end{itemize}

\begin{semiverbatim}
CHEMISTRY
  PRIMARY_SPECIES
    H+
    HCO3-
    Ca++
  /
\end{semiverbatim}
\newpage

\begin{semiverbatim}


  SECONDARY_SPECIES
    OH-
    CO3--
    CO2(aq)
    CaCO3(aq)
    CaHCO3+
    CaOH+
  /
  GAS_SPECIES
    CO2(g)
  /
\end{semiverbatim}
\newpage

\begin{semiverbatim}


  MINERALS
    Calcite
  /
  MINERAL_KINETICS
    Calcite       \bluecomment{! Note: mineral name opens block}
      RATE_CONSTANT 1.d-12  \bluecomment{! SI units}
    /
  /
\end{semiverbatim}
\newpage

\begin{semiverbatim}


  DATABASE ./hanford.dat  \bluecomment{! path to database file}
  LOG_FORMULATION         \bluecomment{! logarithmic derivatives}
:  OPERATOR_SPLITTING     \bluecomment{! operator splitting flag}
  ACTIVITY_COEFFICIENTS TIMESTEP  \bluecomment{! time step lagged}
  OUTPUT                          \bluecomment{!   activity}
    PH                            \bluecomment{!   coefficients}
    all   \bluecomment{! print primary species (defaults to total}
  /       \bluecomment{!   component concentrations)}
END

\end{semiverbatim}

\end{frame}

%-----------------------------------------------------------------------------
\subsection{CONSTRAINT}

\begin{frame}[fragile,allowframebreaks]\frametitle{CONSTRAINT}

\begin{itemize}
  \item Set up solute concentrations and geochemical constraints
\end{itemize}

\begin{semiverbatim}

CONSTRAINT initial      \bluecomment{! define a constraint: \greencomment{initial}}
  CONCENTRATIONS           \bluecomment{! concentration block}
    H+     1.d-8      F    \bluecomment{! \redcomment{F} = free ion}
    HCO3-  1.d-3      G  CO2(g) \bluecomment{! \redcomment{G} = equil. w/ gas}
    Ca++   5.d-4      M  Calcite \bluecomment{! \redcomment{M} = equil. w/ mineral}
  /  \bluecomment{    ! ^^^^^ \redcomment{5.e-4} is a guess}
  MINERALS               \bluecomment{! mineral block}
    Calcite 1.d-5 1.d0   \bluecomment{! vol. frac. = \redcomment{1.e-5}; area = \redcomment{1.}}
  /  \bluecomment{! <-- closes out MINERAL block}
END  \bluecomment{! <-- closes out CONSTRAINT card}

\newpage
CONSTRAINT inlet         \bluecomment{! define a constraint: \greencomment{inlet}}
  CONCENTRATIONS
    H+     5.         P  \bluecomment{! \redcomment{P} = pH; analogous to -log(F)}
    HCO3-  1.d-3      T  \bluecomment{! \redcomment{T} = total component}
    Ca++   1.d-6      Z  \bluecomment{! \redcomment{Z} = charge balance}
  /  \bluecomment{! <-- closes out block; \redcomment{note no MINERAL block}}
END

\end{semiverbatim}

\end{frame}

%-----------------------------------------------------------------------------
\subsection{TRANSPORT\_CONDITION}

\begin{frame}[fragile,allowframebreaks]\frametitle{TRANSPORT\_CONDITION}

\begin{itemize}
  \item Couple transport constraints with transport conditions
\end{itemize}
\begin{semiverbatim}
TRANSPORT_CONDITION initial  \bluecomment{! named \greencomment{initial}}
  TYPE zero_gradient       \bluecomment{! this is a \redcomment{zero gradient} bc}
  CONSTRAINT_LIST          \bluecomment{! list of constraints}
    0.d0 initial  \bluecomment{! start constraint \redcomment{initial} @ time = \redcomment{0.}}
  /  \bluecomment{! <-- closes out CONSTRAINT\_LIST block}
END  \bluecomment{! <-- closes out TRANSPORT\_CONDITION block}

TRANSPORT_CONDITION inlet    \bluecomment{! named \greencomment{inlet}}
  TYPE dirichlet_zero_gradient  \bluecomment{! \redcomment{dirichlet_zero_gradient}}
  CONSTRAINT_LIST                 \bluecomment{! inflow = Dirichlet}
    0.d0 inlet                    \bluecomment{! outflow = zero grad.}
  /    \bluecomment{! ^^^^^ constraint and transport condition}
END    \bluecomment{!         names need not match}

\end{semiverbatim}

\end{frame}

%-----------------------------------------------------------------------------
\subsection{STRATA}

\begin{frame}[fragile]\frametitle{STRATA}

\begin{itemize}
\item Couple \greencomment{soil1} rock/soil type with region \greencomment{all} to define a stratigraphic unit
\end{itemize}

\begin{semiverbatim}

STRATA
  REGION all
  MATERIAL soil1
END


\end{semiverbatim}

\end{frame}

%-----------------------------------------------------------------------------
\subsection{INITIAL\_CONDITION}

\begin{frame}[fragile]\frametitle{INITIAL\_CONDITION}

\begin{itemize}
\item Couple the greencomment{initial} transport condition with region \greencomment{all} for the initial condition
\end{itemize}

\begin{semiverbatim}

INITIAL_CONDITION               \bluecomment{! notice no name}
  TRANSPORT_CONDITION initial
  REGION all
END

\end{semiverbatim}

\end{frame}

%-----------------------------------------------------------------------------
\subsection{BOUNDARY\_CONDITION}

\begin{frame}[fragile]\frametitle{BOUNDARY\_CONDITION}

\begin{itemize}
\item Couple the \greencomment{inlet} transport condition with region \greencomment{west} for the \redcomment{inlet} boundary condition.
\item Couple the \greencomment{initial} transport condition with region \greencomment{east} for the \redcomment{outlet} boundary condition.
\end{itemize}

\begin{semiverbatim}

BOUNDARY_CONDITION outlet      \bluecomment{! name is optional,}
  TRANSPORT_CONDITION initial  \bluecomment{!   but recommended}
  REGION east
END

BOUNDARY_CONDITION inlet
  TRANSPORT_CONDITION inlet
  REGION west
END

\end{semiverbatim}

\end{frame}

%-----------------------------------------------------------------------------
\subsection{LINEAR\_SOLVER}

\begin{frame}[fragile]\frametitle{LINEAR\_SOLVER}

\begin{itemize}
\item Due to the small problem size, request a direct solver.
\item PETSc provides a direct solve through LU matrix factorization as a preconditioner with no outer Krylov solve.
\end{itemize}

\begin{semiverbatim}

LINEAR_SOLVER TRANSPORT   \bluecomment{! transport solver}
  SOLVER DIRECT           \bluecomment{! sets both the below}
END

\bluecomment{! could also use:}
LINEAR_SOLVER TRANSPORT    
  KSP_TYPE PREONLY        \bluecomment{! turn off Krylov solver}
  PC_TYPE PCLU            \bluecomment{! LU preconditioner}
END

\end{semiverbatim}

\end{frame}

%-----------------------------------------------------------------------------
\subsection{TIME}

\begin{frame}[fragile]\frametitle{TIME}

\begin{itemize}
\item Set final simulation time to 25 years
\item Set initial time step size to 1.e-6 years
\item Set maximum time step size to 0.25 years
\end{itemize}


\begin{semiverbatim}

TIME
  FINAL_TIME 25.d0 \redcomment{y}            \bluecomment{! Within TIME, time units}
  INITIAL_TIMESTEP_SIZE 1.d0 \redcomment{h}    \bluecomment{! are recognized and}
  MAXIMUM_TIMESTEP_SIZE 2.5d-1 \redcomment{y}  \bluecomment{! converted to SI}
END                               \bluecomment{! (i.e. \redcomment{y}, \redcomment{h} --> \greencomment{s}).}
\end{semiverbatim}

\end{frame}

%-----------------------------------------------------------------------------
\subsection{OUTPUT}

\begin{frame}[fragile]\frametitle{OUTPUT}

\begin{itemize}
\item Specify output times (5, 10, 15, 20 years) and format (Tecplot point datapacking).
\item The initial and final simulation times are automatically added to the list of output times.
\item Note that additional output options (e.g. species names) are specified within CHEMISTRY.
\end{itemize}


\begin{semiverbatim}

OUTPUT
  TIMES \redcomment{y} 5. 10. 15. 20.  \bluecomment{! Within OUTPUT, time units}
  FORMAT TECPLOT POINT    \bluecomment{! are recognized and}
END                       \bluecomment{! converted to SI}
                          \bluecomment{! (i.e. \redcomment{y} --> \greencomment{s}).}
\end{semiverbatim}

\end{frame}

%-----------------------------------------------------------------------------
\subsection{Miscellaneous}

\begin{frame}[fragile]\frametitle{Miscellaneous}

\begin{itemize}
\item Specify a uniform Darcy velocity of 1 m/y in x-direction
\end{itemize}


\begin{semiverbatim}

UNIFORM_VELOCITY 3.170979d-8 0.d0 0.d0  \bluecomment{! SI units [m/s]}
\end{semiverbatim}

\end{frame}

%-----------------------------------------------------------------------------
\section{Description of Input Deck: Adding a Flow Solution}

\subsection{Input File Modifications}

\begin{frame}[fragile]\frametitle{Input File Modifications}

\begin{itemize}
\item Add cards:
  \begin{itemize}
    \item MODE
    \item LINEAR\_SOLVER (for flow)
    \item SATURATION\_FUNCTION
    \item FLOW\_CONDITION
  \end{itemize}
\item Modify cards:
  \begin{itemize}
    \item MATERIAL\_PROPERTY
    \item INITIAL\_CONDITION
    \item BOUNDARY\_CONDITION
   \end{itemize}
\item Remove cards:
  \begin{itemize}
    \item UNIFORM\_VELOCITY
  \end{itemize}  
\end{itemize}

\end{frame}

%-----------------------------------------------------------------------------
\subsection{MODE}

\begin{frame}[fragile]\frametitle{MODE}

\begin{itemize}
\item Specify Richards flow mode
\item Best to add card at top of file
\end{itemize}


\begin{semiverbatim}

MODE RICHARDS
\end{semiverbatim}

\end{frame}

%-----------------------------------------------------------------------------
\subsection{LINEAR\_SOLVER}

\begin{frame}[fragile]\frametitle{LINEAR\_SOLVER}

\begin{itemize}
\item As with transport, use the direct solver for this small problem.
\end{itemize}


\begin{semiverbatim}

\magentacomment{LINEAR_SOLVER FLOW}
  \magentacomment{SOLVER DIRECT}
\magentacomment{END}

LINEAR_SOLVER TRANSPORT  \bluecomment{! the original transport solver}
  SOLVER DIRECT
END
\end{semiverbatim}

\end{frame}

%-----------------------------------------------------------------------------
\subsection{SATURATION\_FUNCTION}

\begin{frame}[fragile]\frametitle{SATURATION\_FUNCTION}

\begin{itemize}
\item Assume saturated flow
\item Add default (dummy) saturation function
\item For variably-saturated flow, one would include parameters such as the air entry pressure, residual saturation, van Genuchten ``n'', etc. in the saturation function definition, \gehcomment{but ignore this for now}.
\end{itemize}

\begin{semiverbatim}

SATURATION_FUNCTION default
END
\end{semiverbatim}

\end{frame}

%-----------------------------------------------------------------------------
\subsection{FLOW\_CONDITION}

\begin{frame}[fragile]\frametitle{FLOW\_CONDITION}

\begin{itemize}
\item Specify an initial pressure
\item Specify a boundary flux at inlet
\end{itemize}

\begin{semiverbatim}
FLOW_CONDITION initial   \bluecomment{! named \greencomment{initial}}
  TYPE
    PRESSURE dirichlet   \bluecomment{! type is \redcomment{dirichlet} for pressure}
  /                      \bluecomment{!   (constant pressure [Pa])}
  PRESSURE 201325.d0     \bluecomment{! greater than atmospheric}
END                      \bluecomment{!   to ensure saturation}

FLOW_CONDITION inlet      \bluecomment{! named \greencomment{inlet}}
  TYPE
    FLUX neumann          \bluecomment{! type is \redcomment{neumann} for a flux}
  /
  FLUX 3.170979d-8        \bluecomment{! Darcy velocity of 1 m/y}
END
\end{semiverbatim}

\end{frame}

%-----------------------------------------------------------------------------
\subsection{MATERIAL\_PROPERTY}

\begin{frame}[fragile]\frametitle{MATERIAL\_PROPERTY}

\begin{itemize}
\item Add isotropic permeability of 1 Darcy (1.e-12 m$^2$)
\end{itemize}

\begin{semiverbatim}
MATERIAL_PROPERTY soil1
  ID 1
  POROSITY 0.25d0
  TORTUOSITY 1.d0
  \magentacomment{PERMEABILITY}     \bluecomment{! permeability is defined in a block}
    \magentacomment{PERM_ISO 1.d-12}  \bluecomment{! isotropic permeability}
  \magentacomment{/}
  \magentacomment{SATURATION_FUNCTION default}  \bluecomment{! couple the saturation}
END                            \bluecomment{!   function}
\end{semiverbatim}

\end{frame}

%-----------------------------------------------------------------------------
\subsection{INITIAL\_CONDITION}

\begin{frame}[fragile]\frametitle{INITIAL\_CONDITION}

\begin{itemize}
\item Add the \greencomment{initial} flow condition
\end{itemize}

\begin{semiverbatim}

INITIAL_CONDITION              \bluecomment{! flow and transport} 
  \magentacomment{FLOW_CONDITION initial}       \bluecomment{!   conditions can share}
  TRANSPORT_CONDITION initial  \bluecomment{!   the same name, but}
  REGION all                   \bluecomment{!   are considered}
END                            \bluecomment{!   independent.}
 
\end{semiverbatim}

\end{frame}

%-----------------------------------------------------------------------------
\subsection{BOUNDARY\_CONDITION}

\begin{frame}[fragile]\frametitle{BOUNDARY\_CONDITION}

\begin{itemize}
\item Add the \greencomment{inlet} flow condition to the \redcomment{inlet} boundary condition.
\item Add the \greencomment{initial} flow condition to the \redcomment{outlet} boundary condition.
\end{itemize}

\begin{semiverbatim}

BOUNDARY_CONDITION outlet      
  \magentacomment{FLOW_CONDITION initial}
  TRANSPORT_CONDITION initial 
  REGION east
END

BOUNDARY_CONDITION inlet
  \magentacomment{FLOW_CONDITION inlet}
  TRANSPORT_CONDITION inlet
  REGION west
END

\end{semiverbatim}

\end{frame}

%-----------------------------------------------------------------------------
\section{Description of Input Deck: Switch to Infiltration}

\subsection{Input File Modifications}

\begin{frame}[fragile]\frametitle{Input File Modifications}

\begin{itemize}
\item Add cards:
  \begin{itemize}
    \item
  \end{itemize}
\item Modify cards:
  \begin{itemize}
    \item GRID
    \item MATERIAL\_PROPERTY
    \item SATURATION\_FUNCTION
    \item FLOW\_CONDITION
    \item BOUNDARY\_CONDITION
   \end{itemize}
\end{itemize}

\end{frame}

%-----------------------------------------------------------------------------
\subsection{GRID}
\begin{frame}[fragile,containsverbatim]\frametitle{GRID}

\begin{itemize}
  \item Problem domain: $1 \times 1 \times 10$ m (x $\times$ y $\times$ z)
  \item Grid resolution $1 \times 1 \times 0.1$ m
\end{itemize}

\begin{semiverbatim}
GRID
  TYPE structured
  NXYZ \magentacomment{1} 1 \magentacomment{100}     \bluecomment{! Switch to Z}
  BOUNDS
    0.d0 0.d0 0.d0
    \magentacomment{1.d0} 1.d0 \magentacomment{10.d0}  \bluecomment{! 1 m in X, 10 m in Z}
  /
END
\end{semiverbatim}

\end{frame}

%-----------------------------------------------------------------------------
\subsection{MATERIAL\_PROPERTY}

\begin{frame}[fragile]\frametitle{MATERIAL\_PROPERTY}

\begin{itemize}
\item Change name of saturation function
\end{itemize}

\begin{semiverbatim}
MATERIAL_PROPERTY soil1
  ID 1
  POROSITY 0.25d0
  TORTUOSITY 1.d0
  PERMEABILITY
    PERM_ISO 1.d-12
  /
  SATURATION_FUNCTION \magentacomment{sf1}
END
\end{semiverbatim}

\end{frame}

%-----------------------------------------------------------------------------
\subsection{SATURATION\_FUNCTION}

\begin{frame}[fragile]\frametitle{SATURATION\_FUNCTION}

\begin{itemize}
\item Add van Genuchten parameters
\item Assume Mualem permeability (default)
\end{itemize}

\begin{semiverbatim}
SATURATION_FUNCTION \magentacomment{sf1
  SATURATION_FUNCTION_TYPE VAN_GENUCHTEN
  RESIDUAL_SATURATION 0.1d0
  LAMBDA 0.5d0     \bluecomment{! van Genuchten m in n = 1/(1-m)}
  ALPHA 1.d-4      \bluecomment{! [Pa^-1]}
  MAX_CAPILLARY_PRESSURE 1.d8}
END
\end{semiverbatim}

\end{frame}

%-----------------------------------------------------------------------------
\subsection{REGION}

\begin{frame}[fragile,containsverbatim,allowframebreaks]\frametitle{REGION}

\begin{itemize}
  \item Change coordinates for new domain
  \item Change faces for top and bottom, instead of west/east
\end{itemize}

\begin{semiverbatim}

REGION all
  COORDINATES
    0.d0 0.d0 0.d0
    \magentacomment{1.d0} 1.d0 \magentacomment{100.d0}
  /
END

\newpage\magentacomment{REGION top
  FACE top
  COORDINATES
    0.d0 0.d0 10.d0
    1.d0 1.d0 10.d0
  /
END}

\magentacomment{REGION bottom
  FACE bottom
  COORDINATES
    0.d0 0.d0 0.d0
    1.d0 1.d0 0.d0
  /
END}

\end{semiverbatim}

\end{frame}

%-----------------------------------------------------------------------------
\subsection{FLOW\_CONDITION}

\begin{frame}[fragile]\frametitle{FLOW\_CONDITION}

\begin{itemize}
\item Change initial condition to type hydrostatic
\item Specify the datum for the water table
\item Modify Darcy flux to reflect recharge
\end{itemize}

\begin{semiverbatim}
FLOW_CONDITION initial
  TYPE
    PRESSURE \magentacomment{hydrostatic}    \bluecomment{! pressure as a function}
  /                         \bluecomment{!   of elevation}
  \magentacomment{DATUM 0.d0 0.d0 1.d0}      \bluecomment{! location of water table}
  PRESSURE \magentacomment{101325.d0}        \bluecomment{! atmospheric pressure} 
END

FLOW_CONDITION \magentacomment{recharge}
  TYPE
    FLUX neumann
  /
  FLUX \magentacomment{3.170979d-9  ! 10 cm/y}
END

\end{semiverbatim}

\end{frame}

%-----------------------------------------------------------------------------
\subsection{BOUNDARY\_CONDITION}

\begin{frame}[fragile]\frametitle{BOUNDARY\_CONDITION}

\begin{itemize}
\item Change names of regions
\item Change name of flow condition in \redcomment{inlet} boundary condition to \greencomment{recharge}
\end{itemize}

\begin{semiverbatim}

BOUNDARY_CONDITION outlet
  FLOW_CONDITION initial
  TRANSPORT_CONDITION initial
  REGION \magentacomment{bottom}
END

BOUNDARY_CONDITION inlet
  FLOW_CONDITION \magentacomment{recharge}
  TRANSPORT_CONDITION inlet
  REGION \magentacomment{top}
END
\end{semiverbatim}

\end{frame}

\end{document} 
