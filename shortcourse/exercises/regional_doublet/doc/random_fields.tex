\documentclass{beamer}

\usepackage{comment}
\usepackage{color}
\usepackage{listings}
\usepackage{verbatim}
\usepackage{multicol}
\usepackage{booktabs}
\definecolor{green}{RGB}{0,128,0}

\def\EQ#1\EN{\begin{equation*}#1\end{equation*}}
\def\BA#1\EA{\begin{align*}#1\end{align*}}
\def\BS#1\ES{\begin{split*}#1\end{split*}}
\newcommand{\bc}{\begin{center}}
\newcommand{\ec}{\end{center}}
\newcommand{\eq}{\ =\ }
\newcommand{\degc}{$^\circ$C}

\def\p{\partial}
\def\qbs{\boldsymbol{q}}
\def\Dbs{\boldsymbol{D}}
\def\A{\mathcal A}
\def\gC{\mathcal C}
\def\gD{\mathcal D}
\def\gL{\mathcal L}
\def\M{\mathcal M}
\def\P{\mathcal P}
\def\Q{\mathcal Q}
\def\gR{\mathcal R}
\def\gS{\mathcal S}
\def\X{\mathcal X}
\def\bnabla{\boldsymbol{\nabla}}
\def\bnu{\boldsymbol{\nu}}
\renewcommand{\a}{{\alpha}}
%\renewcommand{\a}{{}}
\newcommand{\s}{{\sigma}}
\newcommand{\bq}{\boldsymbol{q}}
\newcommand{\bz}{\boldsymbol{z}}
\def\bPsi{\boldsymbol{\Psi}}

\def\Li{\textit{L}}
\def\Fb{\textbf{f}}
\def\Jb{\textbf{J}}
\def\cb{\textbf{c}}

\def\Dt{\Delta t}
\def\tpdt{{t + \Delta t}}
\def\bpsi{\boldsymbol{\psi}}
\def\dbpsi{\delta \boldsymbol{\psi}}
\def\bc{\textbf{c}}
\def\dbc{\delta \textbf{c}}
\def\arrows{\rightleftharpoons}

\newcommand{\bGamma}{\boldsymbol{\Gamma}}
\newcommand{\bOmega}{\boldsymbol{\Omega}}
%\newcommand{\bPsi}{\boldsymbol{\Psi}}
%\newcommand{\bpsi}{\boldsymbol{\psi}}
\newcommand{\bO}{\boldsymbol{O}}
%\newcommand{\bnu}{\boldsymbol{\nu}}
\newcommand{\bdS}{\boldsymbol{dS}}
\newcommand{\bg}{\boldsymbol{g}}
\newcommand{\bk}{\boldsymbol{k}}
%\newcommand{\bq}{\boldsymbol{q}}
\newcommand{\br}{\boldsymbol{r}}
\newcommand{\bR}{\boldsymbol{R}}
\newcommand{\bS}{\boldsymbol{S}}
\newcommand{\bu}{\boldsymbol{u}}
\newcommand{\bv}{\boldsymbol{v}}
%\newcommand{\bz}{\boldsymbol{z}}
\newcommand{\pressure}{P}

\def\water{H$_2$O}
\def\calcium{Ca$^{2+}$}
\def\copper{Cu$^{2+}$}
\def\magnesium{Mg$^{2+}$}
\def\sodium{Na$^+$}
\def\potassium{K$^+$}
\def\uranium{UO$_2^{2+}$}
\def\hion{H$^+$}
\def\hydroxide{0H$^-$}
\def\bicarbonate{HCO$_3^-$}
\def\carbonate{CO$_3^{2-}$}
\def\cotwo{CO$_2$(aq)}
\def\chloride{Cl$^-$}
\def\fluoride{F$^-$}
\def\phosphoricacid{HPO$_4^{2-}$}
\def\nitrate{NO$_3^-$}
\def\sulfate{SO$_4^{2-}$}
\def\souotwooh{$>$SOUO$_2$OH}
\def\sohuotwocothree{$>$SOHUO$_2$CO$_3$}
\def\soh{$>$SOH}

\newcommand\gehcomment[1]{{{\color{orange} #1}}}
\newcommand\add[1]{{{\color{blue} #1}}}
\newcommand\remove[1]{\sout{{\color{red} #1}}}
\newcommand\codecomment[1]{{{\color{green} #1}}}
\newcommand\redcomment[1]{{{\color{red} #1}}}
\newcommand\bluecomment[1]{{{\color{blue} #1}}}
\newcommand\greencomment[1]{{{\color{green} #1}}}
\newcommand\magentacomment[1]{{{\color{magenta} #1}}}

\begin{comment}
\tiny
\scriptsize
\footnotesize
\small
\normalsize
\large
\Large
\LARGE
\huge
\Huge
\end{comment}

\begin{document}
\title{3D Regional Flow and Transport \ldots in a Nutshell}
\author{Glenn Hammond}
\date{\today}

%\frame{\titlepage}

%-----------------------------------------------------------------------------
\section{Random Fields}

\begin{frame}[fragile,containsverbatim]\frametitle{Random Fields}

Steps for incorporating random fields:
  \begin{itemize}
    \item Generate random fields
    \item Store fields in 1D HDF5 Datasets. PFLOTRAN assumes the following regarding HDF5 Dataset names:
      \begin{itemize}
        \item Permeability is stored in one of the following top-level HDF5 Datasets:
        \begin{itemize}
          \item `\verb=Permeability=' -- isotropic or vertical anisotropy ratio
          \item `\verb=PermeabilityX=', `\verb=PermeabilityY=', or `\verb=PermeabilityZ=' -- anisotropic (default)
        \end{itemize}
        \item Porosity stored in top-level HDF5 Dataset named `\verb=Porosity='
        \item Realization ID (integer) is appended to Dataset name if a multirealization simulation (e.g. \verb=Permeability199=)
      \end{itemize}
    \item Store cell ids in single 1D HDF5 Dataset named `\verb=Cell Ids='. Datatype = integer
    \item Read in through DATASET card in input deck
    \item Couple DATASET with appropriate property under MATERIAL\_PROPERTY
  \end{itemize}

\end{frame}

%-----------------------------------------------------------------------------
\subsection{DATASET}

\begin{frame}[fragile,containsverbatim]\frametitle{DATASET}

\begin{itemize}
  \item \verb=perm_field.h5= contains HDF5 Datasets \verb=Permeability1=, \dots, \verb=Permeability10=
\end{itemize}

\begin{semiverbatim}

DATASET perm_field
  FILENAME ./perm_fields.h5   \bluecomment{! name of hdf5 file}
  REALIZATION_DEPENDENT       \bluecomment{! append realization id}
/
\end{semiverbatim}

\end{frame}

\subsection{MATERIAL\_PROPERTY}

\begin{frame}[fragile]\frametitle{MATERIAL\_PROPERTY}

\begin{itemize}
  \item Assuming anisotropic with a vertical anisotropy ratio for now.
\end{itemize}

\begin{semiverbatim}
MATERIAL_PROPERTY soil3
  ID 3
  POROSITY 0.25d0
  TORTUOSITY 0.5d0
  SATURATION_FUNCTION cc1
  PERMEABILITY
    \magentacomment{DATASET perm_field}      \bluecomment{! name of data set}
    \magentacomment{VERTICAL_ANISOTROPY_RATIO 0.1}
  /
END
\end{semiverbatim}

\end{frame}

\subsection{Launching the Job}

\begin{frame}[fragile]\frametitle{Launching the Job}

\begin{itemize}
  \item Launch the job based on the following assumptions:
  \begin{itemize}
    \item 2 processor groups (simultaneous simulations)
    \item 4 core per simulation
  \end{itemize}

\end{itemize}

\begin{semiverbatim}
mpirun -n 8 pflotran
  -pflotranin stochastic_regional_model.in
  -stochastic
  -num_realizations 10
  -num_groups 2
  -screen_output=no
\end{semiverbatim}

or to run a single realization (e.g. \#5)

\begin{semiverbatim}
mpirun -n 8 pflotran
  -pflotranin stochastic_regional_model.in
  -realization_id 5
\end{semiverbatim}

\end{frame}

\end{document}
