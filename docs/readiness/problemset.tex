\documentclass[12pt]{article} 
%\usepackage{times,helvet}
\usepackage{palatino}
%\usepackage[utf8]{inputenc} %useful to type directly diacritic characters
%\usepackage{ssss}
\usepackage{amsmath,amsbsy,amssymb}
\usepackage{sectsty,hangcaption}
%\usepackage{deflist}
\usepackage{fancyhdr}
\usepackage{tabularx}
\usepackage{verbatim}
\usepackage{moreverb}
\usepackage{float,comment}
\usepackage{graphicx}
\usepackage{longtable}
%\usepackage{portland}
\usepackage{booktabs}

\usepackage[super,comma,sort]{natbib} 
\bibpunct[, ]{(}{)}{;}{a}{,}{,}

%must be last package
%\usepackage{hyperref}
\usepackage[debug=false, colorlinks=true, pdfstartview=FitV, linkcolor=blue, citecolor=blue, urlcolor=blue, pdfpagelabels=true]{hyperref}

\textwidth 6.5in
\textheight 9.5in
%\topmargin -.1in
\topmargin -.75in
\newlength{\boxwidth}
\setlength{\boxwidth}{5.8in}
\oddsidemargin -0in
\evensidemargin -0in
\headheight 0.25in

\lhead{{\sl PFLOTRAN Readiness \ldots}}
\chead{\rm - \thepage\ -}
\rhead{\sl\today}
\cfoot{}

\newcommand\flotran{{\sl FloTran}}
\renewcommand{\baselinestretch}{1.0}
\def\EQ#1\EN{\begin{equation}#1\end{equation}}
\def\BA#1\EA{\begin{align}#1\end{align}}
\def\BS#1\ES{\begin{split}#1\end{split}}
%\newcommand{\EQ}{\begin{equation}}
%\newcommand{\EN}{\end{equation}}
\newcommand{\bcr}{\begin{center}}
\newcommand{\ecr}{\end{center}}
\newcommand{\eq}{\ =\ }
\newcommand{\degc}{$^\circ$C}
\renewcommand{\c}{{\rm CO_2}}
\newcommand{\im}{{\rm imb}}
\newcommand{\m}{{\rm mb}}
\newcommand{\ecm}{{\rm ecm}}
\newcommand{\eff}{{\rm eff}}
\newcommand{\eqr}{{\rm le}}
\newcommand{\equ}{{\rm eq}}
\newcommand{\kin}{{\rm kin}}
\newcommand{\rdx}{{\rm rdx}}
\newcommand{\ind}{{\rm id}}
\newcommand{\dep}{{\rm dp}}
\newcommand{\e}{{\rm{e}}}
\newcommand{\erf}{{\rm{erf}}}
\newcommand{\erfc}{{\rm{erfc}}}
\newcommand{\p}{{\partial}}
\newcommand{\A}{{\mathcal A}}
\newcommand{\B}{{\mathcal B}}
\newcommand{\C}{{\mathcal C}}
\newcommand{\D}{{\mathcal D}}
\newcommand{\E}{{\mathcal E}}
\newcommand{\F}{{\mathcal F}}
\newcommand{\G}{{\mathcal G}}
\newcommand{\I}{{\mathcal I}}
\newcommand{\J}{{\mathcal J}}
\newcommand{\M}{{\mathcal M}}
\newcommand{\cO}{{\mathcal O}}
\renewcommand{\P}{{{\mathcal P}}}
\newcommand{\Q}{{\mathcal Q}}
\newcommand{\R}{{{\mathcal R}}}
\renewcommand{\S}{{\mathcal S}}
\newcommand{\T}{{\mathcal T}}
\newcommand{\W}{{\mathcal W}}
\newcommand{\Y}{{\mathcal Y}}
\newcommand{\Z}{{\mathcal Z}}
\newcommand{\rev}{{\rm rev}}
\newcommand{\irr}{{\rm irr}}
\renewcommand{\a}{{\alpha}}
\newcommand{\abar}{{\bar \alpha}}
\renewcommand{\b}{{\beta}}
\renewcommand{\e}{{\epsilon}}
\newcommand{\s}{{\sigma}}
\newcommand{\w}{{\rm H_2O}}
\newcommand{\air}{{\rm N_2}}
\newcommand{\pe}{{\rm Pe}}
\newcommand{\da}{{\rm Da}}
\renewcommand{\k}{{\dot R}^0}
\renewcommand{\L}{\widehat{\mathcal L}}
%\renewcommand{\bar}{\overline}
\newcommand{\dsty}{{\displaystyle}}
\newcommand{\diff}{{\mathcal D}}
\newcommand{\surf}{\equiv \!\!\!}
\newcommand{\bnabla}{\boldsymbol{\nabla}}
\newcommand{\ba}{\boldsymbol{a}}

\newcommand{\balpha}{\boldsymbol{\alpha}}
\newcommand{\bbeta}{\boldsymbol{\beta}}
\newcommand{\bgamma}{\boldsymbol{\gamma}}
\renewcommand{\d}{{\delta}}

\newcommand{\bA}{\boldsymbol{A}}
\newcommand{\bB}{\boldsymbol{B}}
\newcommand{\bb}{\boldsymbol{b}}
\newcommand{\bC}{\boldsymbol{C}}
\newcommand{\bc}{\boldsymbol{c}}
\newcommand{\bcolon}{\boldsymbol{:}}
\newcommand{\bdot}{\boldsymbol{\cdot}}
\newcommand{\bD}{\boldsymbol{D}}
\newcommand{\bE}{\boldsymbol{E}}
\newcommand{\bF}{\boldsymbol{F}}
\newcommand{\bG}{\boldsymbol{G}}
\newcommand{\bg}{\boldsymbol{g}}
\newcommand{\bi}{\boldsymbol{i}}
\newcommand{\bI}{\boldsymbol{I}}
\newcommand{\bJ}{\boldsymbol{J}}
\newcommand{\bK}{\boldsymbol{K}}
\newcommand{\bL}{\boldsymbol{L}}
\newcommand{\bM}{\boldsymbol{M}}
\newcommand{\bn}{\boldsymbol{n}}
\newcommand{\bdelta}{\boldsymbol{\delta}}
\newcommand{\bGamma}{\boldsymbol{\Gamma}}
\newcommand{\bomega}{\boldsymbol{\omega}}
\newcommand{\bOmega}{\boldsymbol{\Omega}}
\newcommand{\bPsi}{\boldsymbol{\Psi}}
\newcommand{\bO}{\boldsymbol{O}}
\newcommand{\bnu}{\boldsymbol{\nu}}
\newcommand{\bdS}{\boldsymbol{dS}}
\newcommand{\bP}{\boldsymbol{P}}
\newcommand{\bq}{\boldsymbol{q}}
\newcommand{\br}{\boldsymbol{r}}
\newcommand{\bR}{\boldsymbol{R}}
\newcommand{\bS}{\boldsymbol{S}}
\newcommand{\bU}{\boldsymbol{U}}
\newcommand{\bu}{\boldsymbol{u}}
\newcommand{\bv}{\boldsymbol{v}}
\newcommand{\bw}{\boldsymbol{w}}
\newcommand{\bx}{\boldsymbol{x}}
\newcommand{\by}{\boldsymbol{y}}
\newcommand{\bY}{\boldsymbol{Y}}
\newcommand{\bz}{\boldsymbol{z}}
\newcommand{\bzero}{\boldsymbol{0}}

\newcommand{\arrows}{~\rightleftharpoons~}
\newcommand{\arrowstab}{\!\!\!\rightleftharpoons\!\!\!}
\newcommand{\longline}{\noindent\rule[-0.1in]{\textwidth}{0.01in}}

\def\water{H$_2$O}
\def\calcium{Ca$^{2+}$}
\def\copper{Cu$^{2+}$}
\def\magnesium{Mg$^{2+}$}
\def\sodium{Na$^+$}
\def\potassium{K$^+$}
\def\uranium{UO$_2^{2+}$}
\def\hion{H$^+$}
\def\bicarbonate{HCO$_3^-$}
\def\cotwo{CO$_2$}
\def\chloride{Cl$^-$}
\def\fluoride{F$^-$}
\def\phosphoricacid{HPO$_4^{2-}$}
\def\nitrate{NO$_3^-$}
\def\sulfate{SO$_4^{2-}$}
\def\souotwooh{$>$SOUO$_2$OH}
\def\sohuotwocothree{$>$SOHUO$_2$CO$_3$}
\def\soh{$>$SOH}

%\renewcommand{\thepage}{\roman{page}}
%\renewcommand{\thepage}{\arabic{page}}
%\renewcommand{\theequation}{\arabic{section}.\arabic{subsection}-\arabic{equation}}
%\renewcommand{\theequation}{\arabic{section}-\arabic{equation}}
%\setcounter{page}{1}

\setlength{\parindent}{0.3125in}
\setlength{\parskip}{2ex plus 0.2ex minus 0.2ex}

\setcounter{secnumdepth}{4}
\setcounter{tocdepth}{4}
\renewcommand{\contentsname}{CONTENTS}

\setlongtables

\pagestyle{fancy}

\thispagestyle{empty}

\begin{document}

{\bf\Large Readiness Review \hfill \today}

\longline

\noindent
\begin{tabular}{llll}
{\bf Contacts:} &Peter C. Lichtner & 505-667-3420 & lichtner@lanl.gov\\
&Glenn E. Hammond & 509-375-3875 & glenn.hammond@pnl.gov\\
&Bobby Philip & & philipb@ornl.gov
\end{tabular}

\longline

\tableofcontents

\longline

\section{Readiness Review: Problem Set 1}

Readiness test for PFLOTRAN will consist of several problem sets. The first problem set described below applies to a partially saturated groundwater problem with infiltration at the surface. A second dataset in preparation will apply to injection of supercritical CO$_2$ in a deep geologic formation.

\subsection{Problem Description}

The first problem involves solving Richards equation for an isothermal, variably saturated porous medium with infiltration at the top boundary surface. In the Richards formulation the gas pressure is kept constant fixed at atmospheric pressure (101,325 Pa). The porous media is layered consisting of eight homogeneous stratigraphic units from the Hanford 200 Area.

A structured grid is used in the calculation.
A domain size is 80 meters in the vertical. The lateral domain size is determined by the {\tt GRID} keyword through the number of nodes specified by {\tt NXYZ} for {\tt nx} and {\tt ny} and the grid spacing specified by {\tt DXYZ} for {\tt dx} and {\tt dy}. The vertical grid spacing should be fixed at {\tt dz} = 1 meter to accurately determine the boundaries between stratigraphic units. The watertable is located at an elevation of 6 meters from the bottom. 
Either a constant infiltration rate or variable infiltration may be specified at the top boundary. Variable infiltration is read from a file and based on transient rainfall events.
The computational domain is divided into eight stratigraphic units with the material properties listed in Table~\ref{tstrata}.

\begin{table}[h]\centering
\caption{Material properties for each stratigraphic unit taken from Last et al. (2009) and recently updated (Last private comm.).}\label{tstrata}
\vspace{3mm}
\begin{tabular}{lccccccc}
\toprule
Unit & Elevation [m] & $\varphi$ & $\tau$ [---] & $k$ [m$^2$] & $\a$ [Pa$^{-1}$] & $\lambda$ & $s_r$ [---] \\
\midrule
HD   & 77--80 & 0.262 & 1 & 5.43d-13 & 1.9401d-4 & 0.286 & 0.115 \\ 
H1   & 66--77 & 0.317 & 1 & 9.91d-13 & 3.8801d-4 & 0.486 & 0.110\\
H2   & 51--66 & 0.356 & 1 & 3.34d-14 & 1.0211d-4 & 0.541 & 0.118\\
H3   & 45--51 & 0.398 & 1 & 1.74d-14 & 5.1054d-5 & 0.527 & 0.143\\
CCUz & 41--45 & 0.419 & 1 & 5.06d-14 & 5.1054d-5 & 0.555 & 0.095\\
CCUc & 37--41 & 0.281 & 1 & 7.68d-13 & 1.1232d-4 & 0.425 & 0.192\\
Rtf  & 35--37 & 0.419 & 1 & 5.06d-14 & 5.1054d-5 & 0.555 & 0.095\\
Rwi  & 0--35 & 0.294 & 1 & 9.63d-14 & 1.4295d-4 & 0.402 & 0.139\\
\bottomrule
\end{tabular}
\end{table}

Boundary conditions imposed on the system are specified infiltration at the surface $\bq_0(t)$, fixed pressure at the bottom surface, and no flow conditions at the sides of the domain.

\subsection{Governing Equations}

The governing equation is Richards equation given by
\EQ
\frac{\p}{\p t} \varphi s \rho + \bnabla\cdot\bq\rho \eq Q,
\EN
where $\varphi$ denotes the porosity of the porous medium, liquid saturation is denoted by $s$, $\rho$ designates the molar fluid density taken as pure water, and $Q$ denotes a source/sink term. The Darcy velocity $\bq$ is computed from the expression
$\bq_\a$ denotes the Darcy flow rate defined by
\EQ
\bq \eq -\frac{kk_r}{\mu} \bnabla \big(P-W\rho g \bz\big),
\EN
where $P$ denotes fluid pressure, $k$ refers to the intrinsic permeability, $k_r$ denotes the relative permeability, $\mu$ denotes the fluid viscosity, $W$ denotes the formula weight of water, $g$ denotes the acceleration of gravity, and $z$ designates the vertical of the position vector. 

The relative permeability $k_r$ is computed from the van Genuchten correlation (van Genuchten, 1980) given by
\EQ\label{krl} 
k_{r} \eq \sqrt{s_e} \left\{1-\left[1-s_e^{1/\lambda} \right]^\lambda \right\}^2, 
\EN 
where the effective saturation $s_e$ is defined as
\EQ 
s_e \eq \frac{s - s_r}{1 - s_r}, 
\EN 
where $s_r$ denotes the residual saturation. Note that $k_{r}$ = 0 for saturations below the residual saturation. 
The liquid saturation is related to capillary pressure $P_c$ [Pa] by the expression
\EQ\label{sat}
s_e \eq \left[1+\left( \alpha |P_c| \right)^n \right]^{-\lambda}. 
\EN 
The constants $n$ and $\lambda$ are related by the expressions 
\EQ\label{lambda} 
\lambda \eq 1-\frac{1}{n}, \ \ \ \ \ n \eq \frac{1}{1-\lambda}. 
\EN 
The inverse relation is given by
\EQ
P_c \eq \frac{1}{\alpha} \left[\left(s_e\right)^{-1/\lambda} -1 \right]^{1/n}.
\EN
For $P\!>\!P_{\rm ref}$, $s\!=\!1$ and the porous medium is fully saturated. 
For $P\!<\!P_{\rm ref}$, the capillary pressure is equal to $P_c \!=\! P_{\rm ref} \!-\!P$ and the saturation is calculated from Eqn.\eqref{sat}.

Boundary conditions of type Dirichlet or Neumann can be imposed on the system of specified pressure or flux, respectively.

\subsection{Numerical Discretization}

A fully implicit backward Euler time stepping method is used for Richards mode. Richards equation is discretized spatially using integrated finite volume. The residual equation for a time step $\Delta t$ thus has the form
\EQ
R_{n} \eq \frac{(\varphi s \rho)_n^{t+\Delta t} - (\varphi s \rho)_n^{t}}{\Delta t} V_n + \sum_{n'} q_{nn'}^{t+\Delta t} \rho_{nn'}^{t+\Delta t} A_{nn'} - Q_n^{t+\Delta t} V_n,
\EN
where $V_n$ denotes the volume of the $n$th control volume, and $q_{nn'}$, $\rho_{nn'}$, and $A_{nn'}$ refer to the Darcy flow velocity, density, and interfacial area at the interface between control volumes $n$ and $n'$. Upstream weighting is used to calculate the density $\rho_{nn'}$. The velocity $q_{nn'}$ is obtained from Darcy's law
\EQ
q_{nn'} \eq -\left(\frac{kk_r}{\mu}\right)_{nn'} \left(\frac{P_n - P_{n'} -W\rho g z_{nn'}}{d_n+d_{n'}}\right),
\EN
where $d_n$, $d_{n'}$ refer to the distances from the control volume center to the interface, and $z_{nn'}$ denotes the vertical distance between control volumes $n$ and $n'$. Harmonic averaging is used to calculate $kk_r/\mu$ at the interface.

\subsection{Input Files}

\subsubsection{PFLOTRAN Input File}

The PFLOTRAN input file for this problem is listed below.

\tiny

\verbatiminput{pflotran_1.in}

\normalsize

\subsubsection{SAMRAI Input File}

The associated SAMRAI input file is given by:

\tiny

\verbatiminput{pflotran_amr_1.in}


\normalsize

\section{Readiness Review: Problem Set 2}

\subsection{Problem Description}

This problem involves flow and transport in which transport of a tracer is combined with the flow problem of Problem Set 1. In this problem a tracer is injected at a depth of 15 meters with a flow rate of 0.187 kg/s over a duration of two weeks representing a 60,000 gallon leak from one of the storage tanks at the Hanford tank farm.

\subsection{Governing Equations}

The same equations for flow are used as in Problem Set 1. The transport equations for a tracer are coupled to the flow equations through the flow velocity $\bq$ and saturation $s$. The transport equation for a non-reactive tracer has the form
\EQ\label{rxntp}
\frac{\p}{\p t} \varphi s \Psi_j + \bnabla\cdot\big(\bq \Psi_j - \varphi s D \bnabla \Psi_j\big) \eq Q_j - \sum_m\nu_{jm}I_m,
\EN
In this equation $D$ denotes the diffusion/dispersion coefficient taken as a constant, and $Q_j$ denotes the source term from the leaking tank. The quantity $\Psi_j$ denotes the total concentration of the $j$th primary species defined as
\EQ
\Psi_j \eq C_j + \sum_i\nu_{ji} C_i,
\EN
for primary species concentration $C_j$ and secondary species $C_i$ with
\EQ
C_i \eq \gamma_i^{-1} K_i \prod_j\big(\gamma_j C_j\big)^{\nu_{ji}},
\EN
with activity coefficients $\gamma_j$, $\gamma_i$ and equilibrium constant $K_i$. The coefficients $\nu_{ji}$ refer to the stoichiometric reaction coefficients. The second term on the right-hand side of Eqn.\eqref{rxntp} describes mineral reactions with rate $I_m$ provided by transition state theory
\EQ
I_m \eq -k_m a_m \big(1-K_m Q_m\big) \zeta_m,
\EN
where $k_m$ denotes the kinetic rate constant, $a_m$ refers to the specific surface area, $K_m$ denotes the equilibrium constant and $Q_m$ designates the ion activity product defined by
\EQ
Q_m \eq \prod\big(\gamma_j C_j\big)^{\nu_{jm}}.
\EN
The factor $\zeta_m$ takes on the values 0 or 1 depending on whether the mineral is present or supersaturated
\EQ
\zeta_m \eq \left\{
\begin{array}{ll}
1, & K_m Q_m >0 \ \text{or} \ \varphi_m>0\\
0, & \text{otherwise}
\end{array} \right.,
\EN
where $\varphi_m$ denotes the mineral volume fraction. The change in mineral concentration is provided by the equation
\EQ
\frac{\p\varphi_m}{\p t} \eq \overline V_m I_m,
\EN
with molar volume $\overline V_m$.

\subsection{Numerical Discretization}

Either fully implicit or operator splitting may be used in the reactive transport mode. For operator splitting the fully coupled transport equations are split into non-reactive and reactive components according to
\EQ
\frac{\p}{\p t} \varphi s \Psi_j + \bnabla\cdot\big(\bq \Psi_j - \varphi s D \bnabla \Psi_j\big) \eq Q_j,
\EN
followed by
\EQ
\frac{d}{d t} \varphi s \Psi_j \eq - \sum_m\nu_{jm}I_m.
\EN

\subsection{Input Files}

\section{References}

Last, G. V., P. D. Thorne, J. A. Horner, K. R. Parker, B. N. Bjornstad, R.D. Mackley, D. C. Lanigan and B. A. Williams. 2009. Hydrogeology of the Hanford Site Central Plateau---A Status Report for the 200 West Area. PNNL-17913, Rev. 1, Pacific Northwest National Laboratory, Richland, Washington.

\end{document}
