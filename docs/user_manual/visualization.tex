
\section{Visualization}

Visualization of the results produced by PFLOTRAN can be achieved using several different utilities: {\bf Tecplot}, {\bf VisIt}, and {\bf gnuplot}. Plotting 2D or 3D output files can be done using the commercial package {\bf Tecplot}, or the opensource packages {\bf VisIt} and {\bf ParaView}. {\bf Paraview} is similar to {\bf visit} but has not been tested with PFLOTRAN. For 1D problems the opensource software packages {\bf gnuplot} and {\bf matplotlib} are recommended.

The plot package {\bf gnuplot} is useful for plotting observation and mass balance output. With {\bf gnuplot} it is possible to plot data from several files in the same plot.
To do this it is necessary that the files have the same number of rows, e.g. time history points. The files can be merged during input by using the {\tt paste} command as a pipe: e.g.

\verb|plot '< paste file1.dat file2.dat' using 1:($n1*$n2)|

\noindent
plots the product of variable in file 1 in column {\tt n1} times the variable in file 2 in column {\tt n2} of the merged file.
