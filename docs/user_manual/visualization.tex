
\section{Visualization}

Visualization of the results produced by PFLOTRAN can be achieved using several different utilities including commercial and open source software. 
Plotting 2D or 3D output files can be done using the commercial package {\bf Tecplot}, or the opensource packages \href{https://wci.llnl.gov/codes/visit/}{VisIt} and \href{http://www.paraview.org/}{ParaView}. {\bf Paraview} is similar to {\bf VisIt} and both are capable of remote visualization on parallel architectures. 

For 1D problems or for plotting PFLOTRAN observation and mass balance output, the opensource software packages \href{http://www.gnuplot.info/}{gnuplot} and \href{http://matplotlib.org/}{matplotlib} are recommended.
With {\bf gnuplot} and {\bf matplotlib} it is possible to plot data from several files in the same plot.
To do this with {\bf gnuplot} it is necessary that the files have the same number of rows, e.g. time history points. The files can be merged during input by using the {\tt paste} command as a pipe: e.g.

\verb|plot '< paste file1.dat file2.dat' using 1:($n1*$n2)|

\noindent
plots the product of variable in file 1 in column {\tt n1} times the variable in file 2 in column {\tt n2} of the merged file.
