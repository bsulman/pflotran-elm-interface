\documentclass[12pt]{article}
%\usepackage{times}
%\usepackage{palatino}
\usepackage{fourier}
\usepackage{amsmath,amsbsy,amssymb}
\usepackage{deflist}
%\usepackage{fancyheadings}
\usepackage{fancyhdr}
\usepackage{tabularx}
\usepackage{verbatim}
\usepackage{moreverb}
\usepackage{float}
\usepackage{fancybox}
\usepackage{graphicx}
\usepackage{longtable}
\usepackage{subfigure}
%\usepackage{ssss}
\usepackage{fancybox}
\usepackage{booktabs}
%\usepackage{toc_entr}
\usepackage[debug=false, colorlinks=true, pdfstartview=FitV, linkcolor=blue, citecolor=black, urlcolor=blue]{hyperref}

\textwidth 6.5in
\textheight 9.5in
\topmargin -1.0in
%\topmargin -3.280cm
\newlength{\boxwidth}
\setlength{\boxwidth}{5.8in}
\oddsidemargin 0.in
\evensidemargin 0.in
\headheight 0.25in
\lhead{\sl PFLOTRAN: Quick Reference User Guide}
\chead{\sl DRAFT}
%\chead{\rm - \thepage\ -}
%\rhead{DRAFT \qquad \qquad \ \today}
\rhead{\today}
%\rfoot{\today--[\thepage]}
\rfoot{}
\cfoot{\rm - \thepage\ -}
%\cfoot{}
%\footrulewidth 0.4pt
\newcommand{\negsp}{\vspace{-4mm}}
\newcommand\secsp{\vspace{-3 mm}}
\newcommand\secspb{\vspace{-2 mm}}
\newcommand\subsecsp{\vspace{-0.0 mm}}
\newcommand\pflotran{{\bf\sl PFLOTRAN}}
\newcommand\pflow{{\bf\sl PFLOW}}
\newcommand\ptran{{\bf\sl PTRAN}}
\renewcommand{\baselinestretch}{1.01}
%\setlength{\baselineskip}{13pt}
\def\EQ#1\EN{\begin{equation}#1\end{equation}}
\def\BA#1\EA{\begin{align}#1\end{align}}
\def\BS#1\ES{\begin{split}#1\end{split}}
%\newcommand{\EQ}{\begin{equation}}
%\newcommand{\EN}{\end{equation}}
\newcommand{\longline}{\noindent\rule[-0.1in]{\textwidth}{0.01in}}
\newcommand{\bc}{\begin{center}}
\newcommand{\ec}{\end{center}}
\newcommand{\degc}{$^\circ$C}
\newcommand{\eq}{\ =\ }
\newcommand{\eff}{{\rm eff}}
\newcommand{\eqr}{{\rm le}}
\newcommand{\equ}{{\rm eq}}
\newcommand{\kin}{{\rm kin}}
\newcommand{\rdx}{{\rm rdx}}
\newcommand{\ind}{{\rm id}}
\newcommand{\dep}{{\rm dp}}
\newcommand{\e}{{\rm{e}}}
\newcommand{\erf}{{\rm{erf}}}
\newcommand{\erfc}{{\rm{erfc}}}
\renewcommand{\sc}{{\rm sc}}
\newcommand{\sign}{{\rm{sign}}}
\newcommand{\p}{{\partial}}
\newcommand{\A}{{\mathcal A}}
\newcommand{\B}{{\mathcal B}}
\newcommand{\C}{{\mathcal C}}
\newcommand{\D}{{\mathcal D}}
\newcommand{\E}{{\mathcal E}}
\newcommand{\F}{{\mathcal F}}
\newcommand{\G}{{\mathcal G}}
\newcommand{\J}{{\mathcal J}}
\newcommand{\jo}{{j_o}}
\newcommand{\M}{{\mathcal M}}
\newcommand{\mO}{{\mathcal O}}
\renewcommand{\P}{{{\mathcal P}}}
\newcommand{\Q}{{\mathcal Q}}
\newcommand{\R}{{{\mathcal R}}}
\renewcommand{\S}{{\mathcal S}}
\newcommand{\T}{{\mathcal T}}
\newcommand{\W}{{\mathcal W}}
\newcommand{\X}{{\mathcal X}}
\newcommand{\Y}{{\mathcal Y}}
\newcommand{\Z}{{\mathcal Z}}
\newcommand{\rev}{{\rm rev}}
\newcommand{\irr}{{\rm irr}}
\renewcommand{\a}{{\alpha}}
\newcommand{\abar}{{\bar \alpha}}
\renewcommand{\b}{{\beta}}
\renewcommand{\c}{{\rm CO_2}}
\newcommand{\w}{{\rm H_2O}}
\newcommand{\air}{{\rm N_2}}
\newcommand{\pe}{{\rm Pe}}
\newcommand{\da}{{\rm Da}}
\renewcommand{\k}{{\dot R}^0}
\renewcommand{\L}{\widehat{\mathcal L}}
\renewcommand{\bar}{\overline}
\newcommand{\dsty}{{\displaystyle}}
\newcommand{\diff}{{\mathcal D}}
\newcommand{\surf}{\equiv \!\!\!}
\newcommand{\bnabla}{\boldsymbol{\nabla}}
\newcommand{\bA}{\boldsymbol{A}}
\newcommand{\ba}{\boldsymbol{a}}
\newcommand{\bB}{\boldsymbol{B}}
\newcommand{\bC}{\boldsymbol{C}}
\newcommand{\bD}{\boldsymbol{D}}
\newcommand{\bE}{\boldsymbol{E}}
\newcommand{\bF}{\boldsymbol{F}}
\newcommand{\bi}{\boldsymbol{i}}
\newcommand{\bI}{\boldsymbol{I}}
\newcommand{\bJ}{\boldsymbol{J}}
\newcommand{\bK}{\boldsymbol{K}}
\newcommand{\bM}{\boldsymbol{M}}
\newcommand{\bg}{\boldsymbol{g}}
\newcommand{\bGamma}{\boldsymbol{\Gamma}}
\newcommand{\bOmega}{\boldsymbol{\Omega}}
\newcommand{\bPsi}{\boldsymbol{\Psi}}
\newcommand{\bO}{\boldsymbol{O}}
\newcommand{\bnu}{\boldsymbol{\nu}}
\newcommand{\bdS}{\boldsymbol{dS}}
\newcommand{\bq}{\boldsymbol{q}}
\newcommand{\br}{\boldsymbol{r}}
\newcommand{\bR}{\boldsymbol{R}}
\newcommand{\bS}{\boldsymbol{S}}
\newcommand{\bu}{\boldsymbol{u}}
\newcommand{\bv}{\boldsymbol{v}}
\newcommand{\bz}{\boldsymbol{z}}
\newcommand{\arrows}{~\rightleftharpoons~}
\newcommand{\arrowstab}{\!\!\!\rightleftharpoons\!\!\!}
\newcommand{\CA}{C_{\rm A}}
\newcommand{\CB}{C_{\rm B}}
\newcommand{\CC}{C_{\rm C}}
\newcommand{\CAB}{C_{\rm AB}}
\newcommand{\CBC}{C_{\rm BC}}
\newcommand{\RA}{\R_{\rm A}}
\newcommand{\RB}{\R_{\rm B}}
\newcommand{\RC}{\R_{\rm C}}
\newcommand{\RAB}{\R_{\rm AB}}
\newcommand{\RBC}{\R_{\rm BC}}
%\newcommand{\kdsc}{K^{DS}}
%\newcommand{\kdec}{K^{DE}}
%\newcommand{\kdsm}{\overline K^{DS}}
%\newcommand{\kdem}{\overline K^{DE}}
\newcommand{\kdsc}{K^{sc}}
\newcommand{\kdec}{K^{ec}}
\newcommand{\kdsm}{K^{sm}}
\newcommand{\kdem}{K^{em}}
\newcommand{\csm}{S}
\renewcommand{\csc}{S}
\newcommand{\cem}{\E}
\newcommand{\cec}{\E}
\renewcommand{\min}{{\rm min}}
\newcommand{\coll}{{\rm coll}}
\newcommand{\ex}{{\rm ex}}
\newcommand{\srf}{{\rm srf}}
\def\dbar{{\mkern2mu\mathchar'26\mkern-11mu\mathrm{d}}}

\newcounter{saveeqn}%
\newcommand{\alpheqn}{\setcounter{saveeqn}{\value{equation}}%
\stepcounter{saveeqn}\setcounter{equation}{0}%
\renewcommand{\theequation}
      {\mbox{\arabic{saveeqn}\alph{equation}}}}%
\newcommand{\reseteqn}{\setcounter{equation}{\value{saveeqn}}%
\renewcommand{\theequation}{\arabic{equation}}}%

\renewcommand{\contentsname}{TABLE OF CONTENTS}
\renewcommand{\listfigurename}{{LIST OF FIGURES}}
\renewcommand{\listtablename}{{LIST OF TABLES}}

\renewcommand{\arraystretch}{1.3}

\setlength{\parindent}{0.3125in}
\setlength{\parskip}{2ex plus 0.2ex minus 0.2ex}

\setcounter{secnumdepth}{4}
\setcounter{tocdepth}{4}

\pagestyle{fancy}
%\pagestyle{empty}

\thispagestyle{empty}

\renewcommand{\theequation}{\arabic{equation}}
%\renewcommand{\thetable}{{\protect\bf\arabic{table}}}
%\renewcommand{\thefigure}{{\protect\bf\arabic{figure}}}
%\renewcommand{\thepage}{\roman{page}}

\setlongtables

\begin{document}

%begin title page
\noindent
{\large\sffamily LA-UR-06-7048}

\medskip

\noindent
\scriptsize
{\em Approved for public release;}\\
{\em distribution is unlimited.}

\normalsize

\bc
\begin{tabular}{r|l}
~ & ~\\
%{\em Title:} & {\sl Quick Reference Guide: PFLOTRAN 1.0 (LA-CC 06-093)}\\
{\em Title:} & {\sl Quick Reference Guide: PFLOTRAN 2.0 (LA-CC-09-047)}\\
~ & {\sl Multiphase-Multicomponent-Multiscale Massively Parallel} \\
~ & {\sl Reactive Transport Code}\\
~ & ~\\
~ & ~\\
{\em Author(s):} & SciDAC-2 Project (PI: Peter C. Lichtner, lichtner@lanl.gov)\\
~ & ~\\
~ & ~\\
{\em Contacts:}  & Glenn Hammond (glenn.hammond@pnnl.gov)\\
%~ & Chuan Lu (clu@egi.utah.edu)\\
~ & Richard Mills (rmills@ornl.gov)\\
%~ & %David Moulton (moulton@lanl.gov)\\
%~ & Bobby Philip (bphilip@lanl.gov)\\
~ & \\%Barry Smith (bsmith@anl.gov)\\
~ & \\%Albert Valocchi (clu@lanl.gov)\\
~ & ~\\
%{\em Submitted to:} & \\
~ & ~\\
{\em Date:} & \today \\
~ & ~\\
~ & ~\\
~ & {\bf \huge DRAFT}\\
~ & ~\\
\end{tabular}
\ec

\vfill

\noindent
{\Huge\sffamily Los Alamos}

\vspace{-8pt}

\noindent
{\sffamily NATIONAL LABORATORY}

\vspace{-6pt}

\noindent
\scriptsize
Los Alamos National Laboratory, an affirmative action/equal opportunity employer, is operated by the Los Alamos National Security, LLC
for the National Nuclear Security Administration of the U.S. Department of Energy under contract DE-AC52-06NA25396. By acceptance
of this article, the publisher recognizes that the U.S. Government retains a nonexclusive, royalty-free license to publish or reproduce the
published form of this contribution, or to allow others to do so, for U.S. Government purposes. Los Alamos National Laboratory requests
that the publisher identify this article as work performed under the auspices of the U.S. Department of Energy. Los Alamos National
Laboratory strongly supports academic freedom and a researcher�s right to publish; as an institution, however, the Laboratory does not
endorse the viewpoint of a publication or guarantee its technical correctness.

\hfill Form 836 (7/06)

\normalsize

\clearpage

\tableofcontents
%\clearpage

%\listoffigures

%\listoftables

\clearpage

\section{Introduction}

PFLOTRAN solves a system of generally nonlinear partial differential equations describing multiphase, multicomponent and multiscale reactive flow and transport in porous materials. The code is designed to run on massively parallel computing architectures as well as workstations and labtops. Parallelization is achieved through domain decomposition using the PETSc (Port\-a\-ble Extensible Toolkit for Scientific Computation) libraries for the parallelization framework  (Balay et al., 1997).

\section{Governing Equations}

\subsection{Mode: {\tt RICHARDS}}

{\tt RICHARDS} Mode applies to single phase, variably saturated, isothermal systems.
The governing mass conservation equation is given by
\EQ
\frac{\p}{\p t}\left(\varphi s\rho\right) + \bnabla\cdot\left(\rho\bq\right) = Q_w,
\EN
and
\EQ
\bq = -\frac{kk_r}{\mu}\bnabla\left(P-\rho g z\right).
\EN
Here, $\varphi$ denotes porosity, $s$ saturation, $\rho$ water density, $\bq$ Darcy flux, $k$ intrinsic permeability, $k_r$ relative permeability, $\mu$ viscosity, $P$ pressure, $g$ gravity, and $z$ the vertical component of the position vector.  Supported relative permeability functions $k_r$ for Richards' equation include van Genuchten, Books-Corey and Thomeer-Corey, while the saturation functions include Burdine and Mualem.  Water density and viscosity are computed as a function of temperature and pressure through an equation of state for water.

\subsection{Mode: {\tt MPHASE}}
Local equilibrium is assumed between phases for modeling multiphase systems with PFLOTRAN. The multiphase partial differential equations for mass and energy conservation solved by PFLOTRAN have the general form:
\begin{subequations}
\BA\label{mass_conservation_equation}
&\frac{\p}{\p t} \bigg(\varphi \sum_\a s_\a^{}\rho_\a^{} X_i^\a \bigg)
+ \bnabla\cdot\sum_\a\bigg[\bq_\a^{}\rho_\a^{} X_i^\a \bigg.\nonumber\\
& \qquad - \varphi s_\a^{} D_\a^{} \rho_\a^{} \bnabla X_i^\a \bigg] \eq Q_i,
\EA
for the $i$th component, and
\BA\label{energy_equation}
&\frac{\p}{\p t} \bigg(\varphi \sum_\a s_\a\rho_\a U_\a + (1-\varphi) \rho_r c_r T\bigg)\nonumber\\
&+ \bnabla\cdot\sum_\a\bigg[\bq_\a\rho_\a H_\a - \kappa\bnabla T\bigg] \eq Q_e.
\EA
\end{subequations}
for energy. 
In these equations $\a$ designates a fluid phase (e.g. H$_\textrm{2}$O, supercritical CO$_\textrm{2}$) at temperature $T$ and pressure $P_\a$ with the sums over all fluid phases present in the system; 
species are designated by the subscript $i$ 
(e.g. $w\!=\!\textrm{H}_\textrm{2}\textrm{O}$, $c\!=\!$~CO$_\textrm{2}$); 
$\varphi$ denotes the porosity of the geologic formation; 
$s_\a$ denotes the phase saturation state; 
$X_i^\a$ denotes the mole fraction of species $i$ ($\sum_i X_i^\alpha=1$); 
$\rho_\a$, $H_\a$, $U_\a$ refer to the molar density, enthalpy, and internal energy of each fluid phase, respectively; and 
$\bq_\a$ denotes the Darcy flow rate defined by
\EQ
\bq_\a \eq -\frac{kk_\a}{\mu_\a} \bnabla \big(P_\a-W_\a^{}\rho_\a g \bz\big),
\EN
where $k$ refers to the intrinsic permeability, $k_\a$ denotes the relative permeability, $\mu_\a$ denotes the fluid viscosity, $W_\a^{}$ denotes the formula weight, $g$ denotes the acceleration of gravity, and $z$ designates the vertical of the position vector. The source/sink terms, $Q_i$ and $Q_e$, describe injection and extraction of mass and heat at wells, respectively. The quantities $\rho_r$, $c_r$, and $\kappa$ refer to the density, heat capacity, and thermal conductivity of the porous rock.

Additional constitutive relations are needed to account for capillary pressure, and changes in phase which are not discussed in detail here (see \cite{lichtner:pflotran07}).
In PFLOTRAN a variable switching approach is used to account for phase changes enforcing local equilibrium. According to the Gibbs phase rule there are a total of $N_C\!+\!1$ degrees of freedom where $N_C$ denotes the number of independent components. This can be seen by noting that the
intensive
degrees of freedom are equal to $N_{\rm int}\!=\!N_C \!-\! N_P \!+\!2$, where $N_P$ denotes the number of phases. The 
extensive
degrees of freedom equals $N_{\rm ext}\!=\!N_P\!-\!1.$ This gives a total number of degrees of freedom $N_{\rm dof}\!=\!N_{\rm int}\!+\!N_{\rm ext}\!=\!N_C\!+\!1$, independent of the number of phases $N_P$ in the system.

\subsection{Mode: {\tt IMMIS}}

The {\tt IMMIS} mode applies to multiple immiscible phases.

\subsection{Mode: {\tt THC}}

The {\tt THC} mode involves two phase liquid-gas flow coupled to the reactive transport mode.

\subsection{Reactive Transport (Keyword {\tt CHEMISTRY})}

The governing mass conservation equation for Pthe geochemical transport mode for a multiphase system written in terms of a set of independent aqueous primary or basis species with the form
\BA
&\frac{\p}{\p t}\big(\varphi \sum_\a s_\a \Psi_j^\a\big) +
\nabla\cdot\sum_\a\big(\bq_\a - \varphi s_\a \bD_\a\bnabla\big)\Psi_j^\a \nonumber\\
&\qquad\eq Q_j - \sum_m\nu_{jm} I_m -\frac{\p S_j}{\p t},
\EA
where the sums over $\a$ are over all fluid phases in the system, and where $\Psi_j^\a$ denotes the total concentration in the $\a$th fluid phase for primary species $\A_j^{\rm pri}$ defined by
\EQ
\Psi_j^\a = \delta_{l\a}^{} C_j^l + \sum_{i=1}^{N_{\rm sec}}\nu_{ji}^{} C_i^\a.
\EN
In this equation the subscript $l$ represents the aqueous electrolyte phase from which the primary species are chosen.

\section{Installation}

\subsection{Mac OS X}

This installation requires compiler versions 4.4.2 for gfortran, gcc, and g++. The compiler g++ is needed for installing SAMRAI.
Define environment variables: {\tt MPI\_HOME}, {\tt PKGS}, \linebreak {\tt HYPRE\_INSTALL\_DIR}

\subsubsection{OpenMPI}

\begin{verbatim}
setenv F90 gfortran
setenv F77 gfortran
setenv FC gfortran
setenv CC gcc-4
setenv CXX g++-4

./configure \
--prefix=$PKGS/openmpi/openmpi-1.4.1-gcc-4.4.2-gfortran \
--disable-debug \
--enable-static \
--disable-shared

make
make install
\end{verbatim}

\subsubsection{Hypre}

\begin{verbatim}
./configure \
--with-MPI \
--enable-debug \
--disable-opt \
--prefix=${HYPRE_INSTALL_DIR} \
--with-MPI-include=${MPI_HOME}/include \
--with-MPI-libs=openmpi \
--with-MPI-lib-dirs=${MPI_HOME}/lib \
CC=mpicc \
CXX=mpicxx \
F77=mpif90

make
make install
\end{verbatim}

\subsubsection{PETSc}

The PETSc {\tt ./configure} script can be used not only to install PETSc but also HDF5, Parmetis, and various direct solvers such as Mumps and Trilinos solvers. To use PETSc with PFLOTRAN the PETSc developer version petsc-dev is necessary which can be downloaded from the PETSc web page (http://www.mcs.anl.gov/petsc/petsc-as/developers/index.html). The recommended way to do this is to use the version control program Mercurial \linebreak (http://mercurial.selenic.com/wiki).

Define environment variables: $\verb|PETSC_DIR|$ and $\verb|PETSC_ARCH|$ giving the location of the petsc-dev source and the architecture.

\paragraph{MacOSX 10.5.8 (Leopard)}

\begin{verbatim}
./config/configure.py \
--with-blas-lapack-lib="-framework vecLib" \
--with-mpi-dir=$MPI_HOME \
--download-openmpi=1 \
--with-debugging=1 \
--with-shared=0 \
--download-mumps=1 \
--download-parmetis=1 \
--download-scalapack=1 \
--download-blacs=1 \
--download-ml=1 \
--download-hdf5=1 

make all
make test
\end{verbatim}

\paragraph{MacOSX 10.6.3 (Snow Leopard)}

\subsubsection{ORNL's Jaguar XT4/5}

\footnotesize
\begin{verbatim}
./config/configure.py PETSC_ARCH=cray-xt4-pgi_fast \
--configModules=PETSc.Configure \
--optionsModule=PETSc.compilerOptions \
--known-level1-dcache-size=65536 \
--known-level1-dcache-linesize=64 \
--known-level1-dcache-assoc=2 \
--known-memcmp-ok=1 \
--known-sizeof-char=1 \
--known-sizeof-void-p=8 \
--known-sizeof-short=2 \
--known-sizeof-int=4
--known-sizeof-long=8 \
--known-sizeof-long-long=8 \
--known-sizeof-float=4 \
--known-sizeof-double=8 \
--known-sizeof-size_t=8 \
--known-bits-per-byte=8 \
--known-sizeof-MPI_Comm=4 \
--known-sizeof-MPI_Fint=4 \
--known-mpi-long-double=0 \
--with-batch=1 \
--with-shared=0 \
--with-dynamic=0 \
--with-cc=cc \
--with-cxx=CC \
--with-fc=ftn \
--COPTFLAGS="-tp barcelona-64 -fastsse -Mipa=fast" \
--CXXOPTFLAGS="-tp barcelona-64 -fastsse -Mipa=fast" \
--FOPTFLAGS="-tp barcelona-64 -fastsse" \
--with-debugging=0 \
--with-blas-lib=sci \
--with-lapack-lib=sci \
--with-x=0 \
--with-mpi-dir=$MPICH_DIR \
--download-hypre=1 \
--download-parmetis=1 \
--with-hdf5=1 \
--with-hdf5-dir=$HDF5_DIR \
--known-mpi-shared=0
\end{verbatim}
\normalsize
The user will need to load the following HDF5 module beforehand: \ \verb|module load hdf5-parallel|.

\subsection{Building PFLOTRAN}

The source code for PFLOTRAN can be downloaded from the LANL web site using Mercurial: \verb|hg clone https://software.lanl.gov/pflotran/hg/pflotran|.

\verb|make [chuan_co2=1 hdf5=1 coll=1] pflotran|

\subsection{Building SAMRAI Version 2.4.4}

\noindent
To build SAMRAI follow the instructions listed below.

{\tt mkdir SAMRAI-v2.4.4}

{\tt cd SAMRAI-v2.4.4}

{\tt hg clone http://software.lanl.gov/pflotran/hg/samrai SAMRAI}

{\tt mkdir samrai-objs in directory SAMRAI-v2.4.4}

{\tt cd samrai-objs}

Define the environment variables $\verb|SAMRAI_INSTALL_DIR|$, $\verb|MPI_HOME|$, $\verb|HDF5_HOME|$, \linebreak $\verb|PETSC_DIR|$ to be the top level directories of the appropriate packages

\noindent
With MPICH:
\begin{verbatim}
../SAMRAI/configure --prefix=${SAMRAI_INSTALL_DIR} \
--with-CC=mpicc \
--with-CXX=mpicxx \
--with-F77=mpif90 \
--with-mpi \
--with-mpi-include=${MPI_HOME}/include \
--with-mpi-lib-dirs=${MPI_HOME}/lib \
--with-MPICC=mpicc \
--with-x \
--with-hdf5=${HDF5_HOME} \
--with-hypre=${HYPRE_HOME} \
--with-petsc=${PETSC_DIR} \
--with-blaslapack \
--enable-opt \
 --enable-debug \
--enable-char \
--enable-bool \
CXXFLAGS="-DMPICH_IGNORE_CXX_SEEK -DMPICH_SKIP_MPICXX" \
CPPFLAGS="-DMPICH_IGNORE_CXX_SEEK -DMPICH_SKIP_MPICXX"
\end{verbatim}

\noindent
With OPENMPI:
\begin{verbatim}
../${SAMRAI_SRC_DIR}/configure \
--prefix=${SAMRAI_INSTALL_DIR} \
--with-CC=mpicc \
--with-CXX=mpicxx \
--with-F77=mpif90 \
--with-mpi \
--with-mpi-include=${MPI_HOME}/include 
--with-mpi-lib-dirs=${MPI_HOME}/lib 
--with-MPICC=mpicc \
--with-x \
--with-hdf5=${HDF5_INSTALL_DIR} \
--with-hypre=${HYPRE_INSTALL_DIR} \
--with-petsc=${PETSC_DIR} \
--with-blaslapack \
--disable-opt \
--enable-debug \
--enable-char \
--enable-bool \
CXXFLAGS="-DOMPI_IGNORE_CXX_SEEK -DOMPI_SKIP_MPICXX \
  -I${PETSC_DIR}/${PETSC_ARCH}/include -I${PETSC_DIR}/include" \
CPPFLAGS="-DOMPI_IGNORE_CXX_SEEK -DOMPI_SKIP_MPICXX"
\end{verbatim}

{\tt make}

{\tt make install}

Define the environment variable $\verb|SAMRAI|$ to point to $\verb|SAMRAI_INSTALL_DIR|$ (for example in your .cshrc)

Instructions for SAMR utils package

{\tt hg clone  http://software.lanl.gov/pflotran/hg/samrutils}

{\tt cd samrutils}

Define the environment variable $\verb|AMRUTILITIES_HOME|$ to be the directory where you want the libraries and headers installed, it can point to the top level src dir if you like.

\verb|make prefix=${AMRUTILITIES_HOME} lib3d|

Instructions for SAMR solvers package

{\tt hg clone  http://software.lanl.gov/pflotran/hg/samrsolvers}

{\tt cd samrsolvers}

Define the environment variable $\verb|SAMRSOLVERS_HOME|$ to be the directory where you want the libraries and headers installed, it can point to the top level src dir if you like.

\verb|make prefix=${SAMRSOLVERS_HOME} lib3d|

\noindent
Assumptions:
\begin{enumerate}
\item External packages needed by PFLOTRAN are built: petsc-dev, hypre,  
hdf5, \linebreak mpich2/openmpi/some mpi
\item External packages needed by PFLOTRAN AMR interface are built:  
SAMRAI, SAMRUtils, SAMRSolvers
\end{enumerate}

\noindent
Define the environment variables:
\begin{center}
\begin{tabularx}{\linewidth}{lX}
\bf SAMRAI &--points to where SAMRAI is installed\\
\bf AMRUTILITIES\_HOME &--points to where SAMRUtils is installed\\
\bf SAMRSOLVERS\_HOME &--points to where SAMRSolvers is installed
\end{tabularx}
\end{center}

\noindent
Instructions for building $\verb|SAMRAIDriver|$:
\begin{center}
\begin{verbatim}
cd pflotran/src/pflotran
make hdf5=1 pflotranamr

cd samr/src
make
\end{verbatim}
\end{center}

\noindent
At this point there should exist an executable named SAMRAIDriver.

\begin{comment}
The following instructions should aid in installing openmpi, PETSc, HDF5 and PFLOTRAN on a UNIX or Mac computer running MacOSX 10.4 or later.

\subsection{Openmpi}

Set environment variables \verb|PKGS| and \verb|MPI_HOME| and the appropriate \verb|PATH|:

\verb|setenv PKGS /Users/lichtner/petsc/packages|

\verb|setenv MPI_HOME $PKGS/openmpi/openmpi-1.2.5-gcc-4.0.1-absoft-10.1|

\verb|setenv PATH \$PKGS/openmpi/openmpi-1.2.5-gcc-4.0.1-absoft-10.1:\$PATH|

\verb|setenv F90 f90|

\verb|setenv F77 'f90 -YEXT_NAMES=LCS -YEXT_SFX= -f'|

\verb|setenv FC 'f90 -YEXT_NAMES=LCS -YEXT_SFX= -f'|

\verb|setenv CC gcc|

\noindent
Configure using:

\verb|./configure --prefix=$PKGS/openmpi/openmpi-1.2.6-gcc-4.0.1-absoft-10.1|

\noindent
Finally, compile, check installation and install:

\verb|make|

\verb|make check|

\verb|make install|

\subsection{PETSc}

PFLOTRAN uses the Developer version of PETSc. To install PETSc first set the enviroment variables \verb|PETSC_DIR| and \verb|PETSC_ARCH|:

\verb|setenv PETSC_DIR /Users/lichtner/petsc/petsc-dev|

\verb|setenv PETSC_ARCH Intel_MacOSX10.5|

\noindent
Configure PETSc on a Mac using openmpi and Fortran 90 Absoft 10.1:
\begin{verbatim}
./config/configure.py
--with-blas-lapack-lib="-framework vecLib"
--with-mpi-dir=$PKGS/openmpi/openmpi-1.2.6-gcc-4.0.1-absoft-10.1
--with-debugging=0
--with-shared=0
\end{verbatim}

\noindent
Compile and test the PESTc installation with:

\verb|make all test|

\noindent
Optionally install PETSc:

\verb|make install|


\subsection{HDF5}

To install HDF5 set the following environment variables:

\verb|setenv HDF5_INCLUDE $PKGS/hdf/hdf5-1.6.7-gcc-4.0.1-absoft-10.1/include|

\verb|setenv HDF5_LIB $PKGS/hdf/hdf5-1.6.7-gcc-4.0.1-absoft-10.1/lib|

\verb|setenv CC $MPI_HOME/bin/mpicc|

\verb|setenv F9X $MPI_HOME/bin/mpif90|

\verb|setenv CFLAGS -fno-strict-aliasing|

\verb|setenv FFLAGS ""|

\begin{verbatim}
./configure --enable-fortran
--prefix=$PKGS/hdf/hdf5-1.6.7-gcc-4.0.1-absoft-10.1
--disable-debug --enable-production --enable-parallel
--enable-static --disable-shared
\end{verbatim}

\verb|make|

\verb|make check|

\verb|make install|

\subsection{PFLOTRAN}

Compile PFLOTRAN using the command

\verb|make [hdf5=1] pflotran|

\noindent
Create input file \verb|pflotran.in| and run PFLOTRAN with the command:

\verb|mpirun -n #proc pflotran|

\noindent
where \verb|#proc| is the desired number of processor cores.

%%%%%%%%%%%%%%%%%%%%%%%%%%%%%%%%%%%%%%%%%%%%%%%%%%%%%%%%%%%%%%%%%%%%%

\subsection{Direct Solvers}

To implement direct solvers in PETSc with PFLOTRAN first recompile PETSc with the options (petsc-dev, MacOSX 10.5):

\begin{verbatim}
./config/configure.py --with-blas-lapack-lib="-framework vecLib"
--with-mpi-dir=\$PKGS/openmpi/openmpi-1.2.7-gcc-4.0.1-absoft-10.1
--with-debugging=0 --with-shared=0
--download-mumps=1
--download-parmetis=1 --with-parmetis
--download-scalapack=1 --with-scalapack
--download-blacs=1 --with-blacs
\end{verbatim}

\noindent
Then run PFLOTRAN with the command-line options:
\begin{verbatim}
-flow\_mat\_type mpiaij
-flow\_ksp\_type preonly
-flow\_pc\_type lu
-flow\_pc\_factor\_mat\_solver\_package mumps
\end{verbatim}

\subsection{Trilinos}

The Trilinos package for multi-level solvers (ML) may be used with PETSc by configuring petsc-dev with the flag

\begin{verbatim}
./config/configure.py ... --download-ml
\end{verbatim}

When running PFLOTRAN add \verb|-flow_pc_type ml| and/or \verb|-trans_pc_type ml|.
ML options are given in the manual page for PCML (not tested).

\subsection{Condition Number of the Preconditioned Jacobian Matrix}

Since the preconditioned matrix is not explicitly computed, it is necessary to estimate the condition number of the preconditioned Jacobian matrix using PETSc during a PFLOTRAN run. To do this specify a KSP type of GMRES (the Hessenberg matrix is needed that is constructed as part of the Arnoldi process), and then specify

 {\tt -ksp\_monitor\_singular\_value}

The ratio of the largest to smallest singular values gives the condition number estimate for the preconditioned operator.  Note that this flag will also cause the 2-norm of the true residual (as opposed to the preconditioned residual) to be printed.

If you are doing this in the current version of PFLOTRAN, you need

 {\tt -flow\_ksp\_type gmres -flow\_ksp\_monitor\_singular\_value}

The estimates will get better the closer one is to the GMRES restart.  (When restart occurs, the Hessenberg matrix from which the eigenvalue estimates are obtained gets discarded along with everything else.)  The frequency can be changed via the option

 {\tt -flow\_ksp\_gmres\_restart <positive integer>}

The default restart frequency is 30.
\end{comment}

\section{Creating the Input File: PFLOTRAN Keywords}

The PFLOTRAN input file construction is based on keywords. Lines beginning with a colon (:) are treated as comments. Each entry to the input file must begin in the first column. Keywords {\tt SKIP} and {\tt NOSKIP} are used to skip over sections of the input file. Blank lines may occur in input file. Alternate keyword spelling is indicated in round brackets (~). Input options are indicated in square brackets [~], as well as default values. Curly brackets \{~\} indicate the result of invoking the corresponding keyword. Always refer to source code when in doubt!

Initial and boundary conditions and material properties are assigned to spatial regions using a novel {\em coupler} approach. In this approach, initial and boundary conditions (keyword CONDITION) are assigned to regions (keyword REGION) using keywords INITIAL\_CONDITION and BOUNDARY\_CONDITION. Material properties (keyword MATERIAL) are assigned to regions using the keyword STRATIGRAPHY.

\begin{longtable}{ll}%[h]%\centering
%\caption{PFLOTRAN Keywords}\label{flkeywd}

%\vspace{4mm}

%\begin{tabular}{lll}
\toprule[1.5pt]
\multicolumn{1}{c}{\bf Keyword} & \multicolumn{1}{c}{\bf Description}\\
\midrule[1pt]
\hyperlink{target_bcon}{\bf BOUNDARY\_CONDITION} & \\
\hyperlink{target_brk}{\bf BREAKTHROUGH} & \\
\hyperlink{target_brine}{\bf BRINE (BRIN)} & \\
\hyperlink{target_ckpt}{\bf CHECKPOINT} & \\
\hyperlink{target_chem}{\bf CHEMISTRY} & \\
\hyperlink{target_stat}{\bf COMPUTE\_STATISTICS} & \\
\hyperlink{target_constraint}{\bf CONSTRAINT} & transport (optional)\\
\hyperlink{target_datset}{\bf DATASET} & \\
\hyperlink{target_dbg}{\bf DEBUG} & \\
\hyperlink{target_flow_cond}{\bf FLOW\_CONDITION} & \\
\hyperlink{target_fluid_property}{\bf FLUID\_PROPERTY} & \\
\hyperlink{target_grid}{\bf GRID} & (required)\\
\hyperlink{target_init}{\bf INITIAL\_CONDITION} & \\
\hyperlink{target_linsolv}{\bf LINEAR\_SOLVER} & \\
\hyperlink{target_mat}{\bf MATERIAL\_PROPERTY} & \\
\hyperlink{target_mode}{\bf MODE} & \\
\hyperlink{target_newt}{\bf NEWTON\_SOLVER} & \\
\hyperlink{target_numjac}{\bf NUMERICAL\_JACOBIAN} & \\
\hyperlink{target_observation}{\bf OBSERVATION} & \\
\hyperlink{target_orig}{\bf ORIG, ORIGIN} & \\
\hyperlink{target_output}{\bf OUTPUT} & \\
\hyperlink{target_overwrite}{\bf OVERWRITE\_RESTART\_TRANSPORT} & \\
\hyperlink{target_proc}{\bf PROC} & (optional)\\
\hyperlink{target_region}{\bf REGION} & \\
\hyperlink{target_restart}{\bf RESTART} & \\
\hyperlink{target_sat}{\bf SATURATION\_FUNCTION} & \\
\hyperlink{target_src}{\bf SOURCE\_SINK} & \\
\hyperlink{target_strata}{\bf STRATIGRAPHY (STRATA)} & \\
\hyperlink{target_time}{\bf TIME} & \\
\hyperlink{target_timestep}{\bf TIMESTEPPER} & \\
\hyperlink{target_trans_cond}{\bf TRANSPORT\_CONDITION} & \\
\hyperlink{target_unifvel}{\bf UNIFORM\_VELOCITY} & \\
\hyperlink{target_touch}{\bf USE\_TOUCH\_OPTIONS} & \\
\hyperlink{target_veldata}{\bf VELOCITY\_DATASET} & (optional) \\
\hyperlink{target_wallclk}{\bf WALLCLOCK\_STOP} & \\
\bottomrule[1.5pt]
%\end{tabular}
\end{longtable}


\noindent{\bf Conventions and Notation:} Keywords are in boldface with optional modifying keywords in square brackets [\ldots], and user entries in typewriter font.

\begin{comment}
\begin{verbatim}
\protect\hypertarget{target_XXX}{}
\subsection{Keyword: XXX}
{\noindent\bf Description:}
{\noindent\bf Input:}
\begin{deflist}{0000000000}
\item[]
\end{deflist}
{\noindent\bf Explanation:}
\begin{description}
\item[Keyword XXX]
\end{description}
{\noindent\bf Examples:}
\end{verbatim}
\end{comment}

\newpage
\protect\hypertarget{target_bcon}{}

\subsection{Keyword: BOUNDARY\_CONDITION}
{\noindent\bf Description:}
The BOUNDARY\_CONDITION keyword couples conditions specified under the FLOW\_\-CONDITION and/or TRANSPORT\_CONDITION keywords to a REGION in the problem domain.  The use of this keyword enables the use/reuse of flow and transport conditions and regions within multiple boundary and initial conditions and source/sinks in the input deck.

{\noindent\bf Input:}

\begin{deflist}{000}
\item[BOUNDARY\_CONDITION] {\tt boundary\_condition\_name}
\begin{deflist}{000}
\item[FLOW\_CONDITION] {\tt flow\_condition\_name}
\item[TRANSPORT\_CONDITION] {\tt transport\_condition\_name}
\item[REGION] {\tt region\_name}
\end{deflist}
\item[(.,/,END)]
\end{deflist}

{\noindent\bf Explanation:}

\begin{center}
\begin{tabularx}{\linewidth}{lX}
\toprule[1.5pt]
\bf Keyword & \bf Description\\
\midrule
\bf BOUNDARY\_CONDITION & Defines the beginning of a boundary condition entry and the name of the boundary condition.\\
\midrule
\bf FLOW\_CONDITION & Defines the name of the flow condition to be linked to this boundary condition.\\
\midrule
\bf TRANSPORT\_CONDITION & Defines the name of the transport condition to be linked to this boundary condition.\\
\midrule
\bf REGION & Defines the name of the region to which the conditions are linked.\\
\midrule
\bf END & Terminates the boundary condition entry.\\
\midrule\midrule
\bottomrule[1.5pt]
\end{tabularx}
\end{center}

\bigskip

{\noindent\bf Examples:}
\begin{verbatim}
BOUNDARY_CONDITION river
FLOW_CONDITION river_stage
TRANSPORT_CONDITION river_chemistry
REGION river_bank
END

BOUNDARY_CONDITION recharge
FLOW_CONDITION infiltration_flux
TRANSPORT_CONDITION infiltration_chemistry
REGION ground_surface
END
\end{verbatim}

%========================================================================

\newpage
\protect\hypertarget{target_brine}{}

\subsection{Keyword: BRINE (BRIN)}

\begin{deflist}{000}
\item[BRINE, BRIN] {\tt Value m\_nacl} [MOLAL, MASS, MOLE]
\end{deflist}

%========================================================================

\newpage
\protect\hypertarget{target_ckpt}{}

\subsection{Keyword: CHECKPOINT}
{\noindent\bf Description:}
Checkpoint files enable the restart of a simulation at any discrete point in simulation where a checkpoint file has been printed.  When the CHECKPOINT card is included in the input deck, checkpoint files are printed every N time steps, where N is the checkpoint frequency, and at the end of the simulation, should the simulation finish or the be shut down properly mid-simulation using the WALL\_CLOCK\_STOP card.  Checkpoint files are named 'pflotran.chkN', where N is the number of the timestep when the checkpoint file was printed, or 'restart.chk' when printed as PFLOTRAN properly terminates execution.

{\noindent\bf Input:}

\begin{deflist}{0000000000}
\item [CHECKPOINT] checkpoint\_frequency
\end{deflist}

{\noindent\bf Explanation:}

\begin{center}
\begin{tabularx}{\linewidth}{lX}
\toprule[1.5pt]
\bf Keyword & \bf Description\\
\midrule
\bf CHECKPOINT & Toggles on checkpointing and defines the frequency at which checkpoint files will be printed.\\
\bottomrule
\end{tabularx}
\end{center}

\bigskip

{\noindent\bf Examples:}
\begin{verbatim}
CHECKPOINT 1000

CHECKPOINT 5
\end{verbatim}

%========================================================================
\newpage
\protect\hypertarget{target_chem}{}

\subsection{Keyword: CHEMISTRY}

\noindent
{\bf Description:}
The {\bf CHEMISTRY} keyword invokes the reactive transport mode and provides input for primary species, secondary species, minerals, gases, colloids and colloid-facilitated transport, and sorption including ion exchange and surface complexation. Mineral reactions are described through a kinetic rate law based on transition state theory and surface complexation reactions may involve equilibrium, kinetic (reversible or irreversible) or a multirate formulation.

%\shadowbox{
%\begin{minipage}{6.5in}

\noindent {\bf Input:}

\begin{deflist}{000}
\item [CHEMISTRY] ~

\begin{deflist}{000}

\item [PRIMARY\_SPECIES] ~
\item {\tt Name}
\item [(.,\,/,\,END)]

\item [SECONDARY\_SPECIES] ~
\item {\tt Name}
\item [(.,\,/,\,END)]

\item[GAS\_SPECIES] ~
\item {\tt Name}
\item [(.,\,/,\,END)]

\item[MINERALS]
\item {\tt Name}
\item [(.,\,/,\,END)]

\item[COLLOIDS]
\item {\tt Name} \ \ \ \ {\tt Mobile\_Fraction} [---]
\item [(.,\,/,\,END)]

\item[MINERAL\_KINETICS] ~

\begin{deflist}{000}
\item [{\tt Mineral Name}]
\item [RATE\_CONSTANT] \ {\tt Value} [mol/cm$^2$/s]
\item [(.,\,/,\,END)]
\end{deflist}

\item [(.,\,/,\,END)]

\item[SORPTION] ~
\begin{deflist}{000}
\item[SURFACE\_COMPLEXATION\_RXN] ~

\begin{deflist}{000}
\item[EQUILIBRIUM]

\item[MULTIRATE\_KINETIC]

\item[KINETIC]

\item[COMPLEX\_KINETICS] ~
\begin{deflist}{000}
\item[FORWARD\_RATE\_CONSTANT] {\tt Value} [mol/cm$^3$/s???]
\item[BACKWARD\_RATE\_CONSTANT] {\tt Value} [mol/cm$^3$/s???]
\end{deflist}

\item [SITE\_FRACTION] {\tt Value}[---] \ \ \ \ \ (Continuation line `$\backslash$')
\item [RATE, RATES] {\tt Value} [mol/cm$^3$/s?] (Continuation line `$\backslash$')
\item [MULTIRATE\_SCALE\_FACTOR] {\tt Value} [---]

\item [MINERAL] {\tt Mineral Name}
\begin{deflist}{000}
\item[SITE] {\tt Name} \ \ \ {\tt Site Density} [mol/m$^3$]
\item[COMPLEXES] ~
\begin{deflist}{000}
\item[\tt Complex Name]
\end{deflist}
\item [(.,\,/,\,END)] ~
\end{deflist}
\item [(.,\,/,\,END)] ~

\item [COLLOID] {\tt Name}
\begin{deflist}{000}
\item[SITE] {\tt Name} \ \ \ {\tt Site Density} [mol/m$^3$]
\item[COMPLEXES] ~
\begin{deflist}{000}
\item[\tt Complex Name]
\end{deflist}
\item [(.,\,/,\,END)] ~
\end{deflist}
\item [(.,\,/,\,END)] ~
\end{deflist}

\item [(.,\,/,\,END)]
\end{deflist}

\begin{deflist}{000}
\item [ION\_EXCHANGE\_RXN] ~
\begin{deflist}{000}
%\item [EQUILIBRIUM]
\item [MINERAL] {\tt Mineral Name}
\begin{deflist}{000}
\item [CEC] {\tt Value}[mol/m$^3$]
%\item[SITE/CATION] {\tt Name} \ \ \ {\tt Site Density} [mol/m$^3$]
\item[CATIONS] ~
\begin{deflist}{000}
\item[\tt Name]
\end{deflist}
\item [(.,\,/,\,END)] ~
\end{deflist}
\item [(.,\,/,\,END)] ~
\end{deflist}
\item [(.,\,/,\,END)] ~

\item[DISTRIBUTION\_COEF] \ (not implemented)
\item[JUMPSTART\_KINETIC\_SORPTION]
\item[NO\_CHECKPOINT\_KINETIC\_SORPTION]
\item[NO\_RESTART\_KINETIC\_SORPTION]
\end{deflist}

\item [(.,\,/,\,END)]

\item[DATABASE] {\tt Path/Database\_Name}
\item[LOG\_FORMULATION]
\item[NO\_CHECKPOINT\_ACT\_COEFS]
\item[ACTIVITY\_COEFFICIENTS] [\bf LAG, NEWTON, TIMESTEP, NEWTON\_ITERATION]
\item[ACTIVITY\_H2O, ACTIVITY\_WATER]
\item[MOLAL, MOLALITY]
\item[NO\_BDOT]
\item[UPDATE\_POROSITY]
\item[UPDATE\_TORTUOSITY]
\item[UPDATE\_PERMEABILITY]
\item[UPDATE\_MINERAL\_SURFACE\_AREA]
\item[MAX\_DLNC]
\item[OUTPUT] ~
\begin{deflist}{000}
\item[MOLALITY]
\item[MOLARITY]
\item[All]
\item[\tt Species Name]
\item[FREE\_ION]
\item[pH]
\item[TOTAL\_SORBED]
\item[TOTAL\_SORBED\_MOBILE]
\item[COLLOIDS]
\item[KD]
\end{deflist}

\item [(.,\,/,\,END)]

\end{deflist}

\item [(.,\,/,\,END)]
\end{deflist}
%\end{minipage}
%}

\noindent
{\bf Explanation:}

\begin{center}
\begin{tabularx}{\linewidth}{lX}
\toprule
\bf Keyword & \bf Description\\
\midrule
\bf Primary\_Species & List of primary species that fully describe the chemical composition of the fluid. The set of primary species must form an independent set of species in terms of which all homogeneous aqueous equilibrium reactions can be expressed.\\
\midrule
\bf Secondary\_Species & List of aqueous species in equilibrium with primary species.\\
\midrule
\bf Gas\_Species & List of gas species.\\
\midrule
\ldots & \\
\bottomrule
\end{tabularx}
\end{center}

\bigskip

\noindent
{\bf Examples:}

%========================================================================

\newpage
\protect\hypertarget{target_stat}{}

\subsection{Keyword: COMPUTE\_STATISTICS}

\noindent
{\bf Description:}
COMPUTE\_STATISTICS enables the calculation statistical analysis of flow velocities during a simulation.  The average, maximum, minimum, and standard deviations velocities are computed.

\noindent{\bf Input:}
\begin{deflist}{000}
\item [COMPUTE\_STATISTICS] \{compute\_statistics = .true.\}
\end{deflist}

\noindent {\bf Explanation:}

\noindent
{\bf Example:}
\begin{verbatim}
COMPUTE_STATISTICS
\end{verbatim}

%========================================================================

\newpage
\protect\hypertarget{target_constraint}{}

\subsection{Keyword: CONSTRAINT}

\noindent
{\bf Description:}
The keyword {\bf CONSTRAINT} sets up fluid compositions based on various constraint conditions chosen by the user.

\noindent{\bf Input:}
\begin{deflist}{000}
\item [CONSTRAINT] \ {\tt constrain\_name}
\begin{deflist}{000}
\item[CONC, CONCENTRATIONS]
\begin{deflist}{000}
\item[{\tt Name, Concentration\_Value, Constraint}] ~\\
The variable {\tt Constraint} is chosen from the following list:
\item[F, FREE]
\item[T, TOTAL] 
\item[TOTAL\_SORB]
\item[P, PH]
\item[L, LOG] 
\item[M, MINERAL, MNRL]
\item[G, GAS]
\item[SC, CONSTRAINT\_SUPERCRIT\_CO2] 
\item[Z, CHG]
\end{deflist}
\item[(., /, END)]
\item[MNRL, MINERALS] ~
\begin{deflist}{000}
\item[{\tt mineral\_name, volume\_fraction, surface\_area}]
\end{deflist}
\item[(., /, END)]
\end{deflist}
\item[(., /, END)]
\end{deflist}

\noindent {\bf Explanation:}

\noindent
{\bf Example:}
\begin{verbatim}
CONSTRAINT initial
CONCENTRATIONS
H+       7.3             pH
O2(aq)   1.78132e-4       T
Al+++    1.e-9            M K-Feldspar
Ca++     1.20644e-3       M Calcite
Cu++     1.e-6            T
Fe++     1.e-9            M Ferrihydrite
Mg++     5.09772e-4       T
UO2++    2.34845e-7       T
K+       1.54789e-4       T
Na+      2.03498e-3       T
HCO3-    2.57305e-3       T
Cl-      6.97741e-4       T
F-       2.09491e-5       T
HPO4--   1.e-6            T
NO3-     4.69979e-3       T
SO4--    6.37961e-4       T
SiO2(aq) 5.36989e-4       T
Tracer   2.34845e-7       F
/
MINERALS
Quartz         0.35  1.
Calcite        0.    1.
Metatorbernite 0.    1.
/
/
\end{verbatim}

%========================================================================

\newpage
\protect\hypertarget{target_datset}{}

\subsection{Keyword: DATASET}

%\noindent{\bf Description:} Input datafile for thermodynamic properties of aqueous species, minerals, gases, colloids, and surface complexes.

\noindent{\bf Description:} Dataset for permeability.

\noindent{\bf Input:}
\begin{deflist}{000}
\item[DATASET] [permx, permy, permz] [permx\_filename, permy\_filename, permz\_filename]
\end{deflist}

%========================================================================

\newpage
\protect\hypertarget{target_dbg}{}

{\noindent\bf Input:}

\subsection{Keyword: DEBUG}
\begin{deflist}{00}
\item[DEBUG]~
\begin{deflist}{00}
\item[PRINT\_SOLUTION] [\bf VECVIEW\_SOLUTION, VIEW\_SOLUTION]
\item[PRINT\_RESIDUAL] [VECVIEW\_RESIDUAL,VIEW\_RESIDUAL]
\item[PRINT\_JACOBIAN] [MATVIEW\_JACOBIAN, VIEW\_JACOBIAN]
\item[PRINT\_JACOBIAN\_NORM] [NORM\_JACOBIAN]
\item[PRINT\_COUPLERS] [PRINT\_COUPLER]
\item[PRINT\_JACOBIAN\_DETAILED] [MATVIEW\_JACOBIAN\_DETAILED, VIEW\_JACOBIAN\_DETAILED]
\item  PRINT\_NUMERICAL\_DERIVATIVES] [VIEW\_NUMERICAL\_DERIVATIVES]
\end{deflist}
\item[(.,/,END)]
\end{deflist}

{\noindent\bf Explanation:}

\bigskip

{\noindent\bf Examples:}

%================================================================

\newpage
\protect\hypertarget{target_flow_cond}{}

\subsection{Keyword: FLOW\_CONDITION}

{\noindent\bf Description:}
The {\bf FLOW\_CONDITION} keyword specifies scalar or vector data sets to be associated with a given boundary or initial condition.  For instance, to specify a hydrostatic boundary condition, the use would specify a condition with a pressure associated with a point in space (i.e. datum) in space and a gradient, both vector quantities.  Note that in the case of a hydrostatic boundary condition, the vertical gradient specified in the input deck must be zero in order to enable the hydrostatic pressure calculation.  Otherwise, the specified vertical gradient overrides the hydrostatic pressure.  Transient pressures, temperatures, concentrations, datums, gradients, etc. are specified using the {\bf FILE} {\tt filename} combination for the name of the data set.

{\noindent\bf Input:}
\begin{deflist}{000}
\item [FLOW\_CONDITION] \ {\tt flow\_condition\_name}
\begin{deflist}{000}
\item [UNITS] {\tt Value} (not currently supported)
\begin{deflist}{000000000}
{\tt Value} is one of the following entries:
\item[s, sec, min, hr, d, day, w, week, mo, month, y, yr] (time)
\item[mm, cm, m, met, meter, dm, km] (length)
\item[kg/s, kg/yr] (rate)
\item[Pa, KPa] (pressure)
\item[m/s, m/yr] (velocity)
\item[C, K] (temperature)
\item[M, mol/L] (concentration)
\item[KJ/mol] (enthalpy)
\end{deflist}

%\item[(., /, END)]

\item[CYCLIC] 

\item[INTERPOLATION] ~
\begin{deflist}{000}
\item[step]
\item[linear]
\end{deflist}

\item[TYPE] ~

\begin{deflist}{000}
\item[PRESSURE] [{\bf dirichlet, hydrostatic, zero\_gradient, conductance, seepage}]
\item[RATE] [{\bf mass\_rate}]
\item[FLUX] [{\bf dirichlet, neumann, mass\_rate, hydrostatic, conductance, zero\_gradient, production\_well, seepage, volumetric, volumetric\_rate, equilibrium}]
\item[TEMPERATURE] [{\bf dirichlet, hydrostatic, zero\_gradient}]
\item[CONCENTRATION] [{\bf dirichlet, hydrostatic, zero\_gradient}]
\item[ENTHALPY (H)] [{\bf dirichlet, hydrostatic, zero\_gradient}]
\end{deflist}
\item[(., /, END)]
\item[TIME] (not currently supported)

\item[IPHASE] {\tt Value[integer]}

\item[DATUM] ~
\begin{deflist}{000}
\item[{\tt x \ y \ z}]
\item[{\bf FILE} \ {\tt file\_name}]
\end{deflist}
\item[GRADIENT, GRAD] ~
\begin{deflist}{000}
\item [PRES, PRESS, PRESSURE] ~
\begin{deflist}{000}
\item[$d_{dx}$ $d_{dy}$ $d_{dz}$]
\item[{\bf FILE} \ {\tt file\_name}]
\end{deflist}
\item [FLUX]
\item [TEMP, TEMPERATURE]
\item [CONC, CONCENTRATION]
\item [H, ENTHALPY]
\end{deflist}
\item[(., /, END)]
\item[TEMPERATURE, TEMP] \ {\tt Value}
\item[ENTHALPY, H] \ {\tt Value}
\item[PRESSURE, PRES, PRESS] \ {\tt Value}
\item[RATE] \ {\tt Value}
\item[FLUX, VELOCITY, VEL] \ {\tt Value}
\item[CONC, CONCENTRATION] \ {\tt Value}
\item[CONDUCTANCE] \ {\tt Value}
%\item[CONSTRAINT\_LIST]
\end{deflist}
\item[(., /, END)] ~
\end{deflist}

{\noindent\bf Explanation:}

\begin{center}
\begin{tabularx}{\linewidth}{lX}
\toprule[1.5pt]
\bf Keyword & \bf Description\\
\midrule
\bf FLOW/TRANSPORT\_CONDITION & Initiates a condition entry and defines its name.\\
\midrule
\bf CYCLIC & Instructs PFLOTRAN to cycle the transient data set should the simulation time exceed the last time in the data set.\\
\midrule
\bf INTERPOLATION & Defines the method for interpolating between data set times.\\
\midrule
\bf DATUM & Location is space where prescribed scalar (e.g. pressure, temperature concentration, etc.) is defined.\\
\midrule
\bf TYPE & Specifies the type of condition.\\
\midrule
\bf PRESSURE & Specifies the type of pressure condition.\\
 \bf TEMPERATURE & Specifies the type of temperature condition.\\
    \bf CONCENTRATION & Specifies the type of concentration condition.\\
    \bf ENTHALPY & Specifies the type of enthalpy condition.\\
    \bf END & Terminates type entry.\\
\midrule\midrule
\bf GRADIENT & Gradient of the scalar field in 3D space.\\
\midrule
\bf PRESSURE & Pressure gradient in $x$-, $y$-, and $z$-directions.\\
\bf TEMPERATURE & Temperature gradient in $x$-, $y$-, and $z$-directions.\\
\bf CONCENTRATION & Concentration gradient in $x$-, $y$-, and $z$-directions.\\
\bf ENTHALPY & Enthalpy gradient in $x$-, $y$-, and $z$-directions.\\
\bf END & Terminates gradient entry.\\
\midrule\midrule
\bf PRESSURE & Absolute fluid pressure at the datum.\\
\bf FLUX & Darcy velocity of fluid defining flux across a boundary.\\
\bf TEMPERATURE & Temperature in $^\circ$C at the datum.\\
\bf CONCENTRATION & Solute concentration at the datum.\\
\bf ENTHALPY & Enthalpy at the datum.\\
\bf CONSTRAINT\_LIST & Specifies a list of concentration constraints for solute transport.\\
\bf END & Terminates the condition entry.\\
\bottomrule[1.5pt]
\end{tabularx}
\end{center}

\bigskip

{\noindent\bf Examples:}
\begin{verbatim}
FLOW_CONDITION initial
TYPE
PRESSURE hydrostatic
/
PRESSURE 1956741.84 ! 200 meter piezometric head (200*997.32*9.81)
/

FLOW_CONDITION source
TYPE
RATE volumetric_rate
/
RATE 2.77777778d-3  ! 10 m^3/hr
/

TRANSPORT_CONDITION initial
TYPE zero_gradient
CONSTRAINT_LIST
0.d0 initial
/
END

TRANSPORT_CONDITION source
TYPE dirichlet
CONSTRAINT_LIST
0.d0 well
/

FLOW_CONDITION East
TYPE
:PRESSURE seepage
PRESSURE conductance
/
CYCLIC
DATUM file ../../river_scope3.datum
GRADIENT
PRESSURE file ../../river_scope3.gradient
/
CONDUCTANCE 1.d-12
PRESSURE 101325.d0
/
\end{verbatim}

%========================================================================

\newpage
\protect\hypertarget{target_fluid_property}{}

\subsection{Keyword: FLUID\_PROPERTY}

\noindent{\bf Description:}

\noindent{\bf Input:}
\begin{deflist}{000}
\item[FLUID\_PROPERTY]~
\begin{deflist}{000}
\item[PHASE] \ {\tt Name}
\item[DIFFUSION\_COEFFICIENT] \ {\tt Value} [m$^2$/s]
\end{deflist}
\item[(., /, END)]
\end{deflist}

\noindent{\bf Explanation:}

\noindent{\bf Example:}
\begin{verbatim}
FLUID_PROPERTY
DIFFUSION_COEFFICIENT 1.d-9
/
\end{verbatim}

%========================================================================

\newpage
\protect\hypertarget{target_grid}{}

\subsection{Keyword: GRID \hfill Required}

\noindent{\bf Description: this keyword defines a spatial grid.}

\noindent{\bf Input:}
\begin{deflist}{000}
\item[GRID]~
\begin{deflist}{000}
\item[TYPE] {\bf structured \ [cartesian]}
\item[TYPE] {\bf structured \ cylindrical}
\item[TYPE] {\bf structured \ spherical}

\item[TYPE] {\bf unstructured} \ {\tt file\_name}
\item[TYPE] {\bf amr}

\item[ORIG, ORIGIN] {\tt X, Y, Z}
\item[NXYZ] \ {\tt nx ny nz}
\item[FILE]
\item[GRAVITY]
\item[INVERT\_Z]
\item[DXYZ]~
\begin{deflist}{000}
\item[{\tt dx}]
\item[{\tt dy}]
\item[{\tt dz}]
\end{deflist}
\item[(., /, END)]~

\item[CARTESIAN coordinates input:]~
\begin{deflist}{000}
\item[BOUNDS]~
\begin{deflist}{000}
\item[{\tt x\_min, x\_max}]
\item[{\tt y\_min, y\_max}]
\item[{\tt z\_min, z\_max}]
\end{deflist}
\item[(., /, END)]~
\end{deflist}

\item[CYLINDRICAL coordinates input:]~
\begin{deflist}{000}
\item[BOUNDS]~
\begin{deflist}{000}
\item[{\tt r\_min, r\_max}]
\item[{\tt z\_min, z\_max}]
\end{deflist}
\item[(., /, END)]~
\end{deflist}

\item[SPHERICAL coordinates input:]~
\begin{deflist}{000}
\item[BOUNDS]~
\begin{deflist}{000}
\item[{\tt r\_min, r\_max}]
\end{deflist}
\item[(., /, END)]~
\end{deflist}

%\item UNSTRUCTURED
\item[AMR input:] ~
\begin{deflist}{000}
\item[BOUNDS]~
\begin{deflist}{000}
\item[{\tt dx}]
\item[{\tt dy}]
\item[{\tt dz}]
\end{deflist}
\item[(., /, END)]
\end{deflist}
\end{deflist}
\item[(., /, END)]
\end{deflist}

\noindent{\bf Explanation:}

\begin{center}
\begin{tabularx}{\linewidth}{lX}
GRAVITY <\# \# \#>: &Specifies gravity vector\\

ORIGIN <\# \# \#>: &Coorindate of grid origin (default: 0. 0. 0.)\\

BOUNDS: &Specifies bounds of structured cartesian grid (see examples below)
DXYZ: Specifies grid spacing of structured cartesian grid (see examples below)\\
INVERT\_Z: &Inverts the z axis (positive Z is down)
\end{tabularx}
\end{center}

\noindent
{\bf Example:}
\begin{verbatim}
GRID
TYPE structured cylindrical
NXYZ 512  1  32
DXYZ
2.d0
1.d0
2.d0
END

BOUNDS
0. 1000.
0. 64.
END
\end{verbatim}

%========================================================================

\newpage
\protect\hypertarget{target_init}{}

\subsection{Keyword: INITIAL\_CONDITION}

\noindent{\bf Description:}

\noindent{\bf Input:}
\begin{deflist}{000}
\item[INITIAL\_CONDITION] \ [{\tt Name}]
\begin{deflist}{000}
\item[REGION] {\tt region\_name}
\item[FLOW\_CONDITION] {\tt condition\_name}
\item[TRANSPORT\_CONDITION] {\tt condition\_name}
%\item[TYPE] [{\tt initial, boundary, source\_sink}]
%\item[FACE] \ {\bf [WEST, EAST, NORTH, SOUTH, BOTTOM, TOP]}
\end{deflist}
\item[(., /, END)] ~
\end{deflist}

\noindent{\bf Explanation:}

\noindent{\bf Example:}

%========================================================================

\newpage
\protect\hypertarget{target_linsolv}{}

\subsection{Keyword: LINEAR\_SOLVER}

\noindent{\bf Description:}

\noindent{\bf Input:}
\begin{deflist}{000}
\item[LINEAR\_SOLVER] \ [{\bf TRAN, TRANSPORT / FLOW}]

\begin{deflist}{000}
\item[SOLVER\_TYPE (SOLVER, KRYLOV\_TYPE, KRYLOV, KSP, KSP\_TYPE)] ~

\begin{deflist}{000}
\item[NONE (PREONLY)]
\item[GMRES]
\item[FGMRES]
\item[BCGS] \ (BICGSTAB, BI-CGSTAB)
\item[IBCGS] \ (IBICGSTAB, IBI-CGSTAB)
\item[RICHARDSON]
\item[CG]
\end{deflist}

\item[PRECONDITIONER\_TYPE (PRECONDITIONER, PC, PC\_TYPE)] ~

\begin{deflist}{000}
\item[NONE (PCNONE)]
\item[ILU (PCILU)]
\item[LU (PCLU)]
\item[BJACOBI (BLOCK\_JACOBI)]
\item[ASM (ADDITIVE\_SCHWARTZ)]
\item[PCASM]
\item[HYPRE]
\item[SHELL]
\end{deflist}

\item[HYPRE\_OPTIONS]
\begin{deflist}{000}
\item[TYPE] \ [{\tt pilut, parasails, boomeramg, euclid}]
\item[BOOMERAMG\_CYCLE\_TYPE] / [{\tt V, W}]
\item[BOOMERAMG\_MAX\_LEVELS] \ {\tt Value}
\item[BOOMERAMG\_MAX\_ITER] \ {\tt Value}
\item[BOOMERAMG\_TOL] \ {\tt Value}
\item[BOOMERAMG\_TRUNCFACTOR] \ {\tt Value}
\item[BOOMERAMG\_AGG\_NL] \ {\tt Value}
\item[BOOMERAMG\_AGG\_NUM\_PATHS] \ {\tt Value}
\item[BOOMERAMG\_STRONG\_THRESHOLD] \ {\tt Value}
\item[BOOMERAMG\_GRID\_SWEEPS\_ALL] \ {\tt Value}
\item[BOOMERAMG\_GRID\_SWEEPS\_DOWN] \ {\tt Value}
\item[BOOMERAMG\_GRID\_SWEEPS\_UP] \ {\tt Value}
\item[BOOMERAMG\_GRID\_SWEEPS\_COARSE] \ {\tt Value}
\item[BOOMERAMG\_RELAX\_TYPE\_ALL] \ {\tt Value}
\item[BOOMERAMG\_RELAX\_TYPE\_DOWN] \ {\tt Value}
\item[BOOMERAMG\_RELAX\_TYPE\_UP] \ {\tt Value}
\item[BOOMERAMG\_RELAX\_TYPE\_COARSE] \ {\tt Value}
\item[BOOMERAMG\_RELAX\_WEIGHT\_ALL] \ {\tt Value}
\item[BOOMERAMG\_RELAX\_WEIGHT\_LEVEL] \ {\tt Value}
\item[BOOMERAMG\_OUTER\_RELAX\_WEIGHT\_ALL] \ {\tt Value}
\item[BOOMERAMG\_OUTER\_RELAX\_WEIGHT\_LEVEL] \ {\tt Value}
\item[BOOMERAMG\_NO\_CF] \ {\tt Value}
\item[BOOMERAMG\_MEASURE\_TYPE] \ {\tt Value}
\item[BOOMERAMG\_COARSEN\_TYPE] \ {\tt Value}
\item[BOOMERAMG\_INTERPOLATION\_TYPE, BOOMERAMG\_INTERP\_TYPE] \ {\tt Value}
\item[BOOMERAMG\_NODAL\_COARSEN] \ {\tt Value}
\item[BOOMERAMG\_NODAL\_RELAXATION] \ {\tt Value}
\end{deflist}

\item[ATOL] \ {\tt Value}
\item[RTOL] \ {\tt Value}
\item[DTOL] \ {\tt Value}
\item[MAXIT] \ {\tt Value}

\end{deflist}
\item[(., /, END)]
\end{deflist}

\noindent{\bf Explanation:}

\noindent{\bf Example:}

%========================================================================

\newpage
\protect\hypertarget{target_mat}{}

\subsection{Keyword: MATERIAL\_PROPERTY}

\noindent{\bf Description:}

\noindent{\bf Input:}
\begin{deflist}{000}
\item[MATERIAL\_PROPERTY] \ {\tt Name}
\begin{deflist}{000}
\item[ID] \ {\tt Value}
\item[SATURATION\_FUNCTION] \ {\tt Name}
\item[ROCK\_DENSITY] \ {\tt Value}
\item[SPECIFIC\_HEAT] \ {\tt Value}
\item[LONGITUDINAL\_DISPERSIVITY] \ {\tt Value}
\item[TRANSVERSE\_DISPERSIVITY] \ (not implemented)
\item[THERMAL\_CONDUCTIVITY\_DRY] \ {\tt Value}
\item[THERMAL\_CONDUCTIVITY\_WET] \ {\tt Value}
\item[PORE\_COMPRESSIBILITY] \ {\tt Value}
\item[THERMAL\_EXPANSITIVITY] \ {\tt Value}
\item[POROSITY] \ {\tt Value}
\item[POROSITY] \ {\tt porosity\_filename}
\item[TORTUOSITY] \ {\tt Value}
\item[PERMEABILITY] ~
\begin{deflist}{000}
\item[ANISOTROPIC]
\item[VERTICAL\_ANISOTROPY\_RATIO] \ {\tt Value}
\item[ISOTROPIC]
\item[PERM\_X] \ {\tt Value}
\item[PERM\_Y] \ {\tt Value}
\item[PERM\_Z] \ {\tt Value}
\item[PERM\_ISO] \ {\tt Value}
\end{deflist}
\item[(., /, END)]
\item[PERMEABILITY\_POWER] \ {\tt Value}
\item[TORTUOSITY\_POWER] \ {\tt Value}
\item[MINERAL\_SURFACE\_AREA\_POWER] \ {\bf VOLUME FRACTION} \ {\tt Value}
\item[MINERAL\_SURFACE\_AREA\_POWER] \ {\bf POROSITY} \ {\tt Value}
\item[RANDOM\_DATASET] \ {\tt permeability\_filename}
\end{deflist}
\item[(., /, END)]
\end{deflist}

\noindent{\bf Explanation:}

\noindent{\bf Example:}
\begin{verbatim}
MATERIAL_PROPERTY  Hanford
ID 1
SATURATION_FUNCTION sf1
POROSITY 0.332
TORTUOSITY 1.
PERMEABILITY
PERM_X 1.d-12
PERM_Y 1.d-12
PERM_Z 1.d-12
/
/
\end{verbatim}

%========================================================================

\newpage
\protect\hypertarget{target_mode}{}

\subsection{Keyword: MODE}

\noindent{\bf Description:} determines the mode of operation: IMMIS (Immisible), MPH (CO$_2$ + H$_2$O), Richards (vadose-saturated zones), THC (Thermal-Hydrologic-Chemical, not implemented).

\noindent{\bf Input:}
\begin{deflist}{000}
\item[MODE] ~
\begin{deflist}{000}
\item[IMS, IMMIS, THS]
\item[MPH, MPHASE]
\item[RICHARDS]
%\item[RICHARDS\_LITE]
\item[THC (\rm not implemented)]
\end{deflist}
\end{deflist}

\noindent{\bf Explanation:}

\begin{center}
\begin{tabularx}{\linewidth}{lX}
HC: &Thermo-Hydro-Chemical Coupled groundwater flow, thermal and solute transport\\
Richards: &Single-phase variable saturated groundwater flow using Richards Equation\\
MPH, MPHASE: &Multiphase CO$_2$-water-energy\\
IMS, IMMIS, THS: &Immissible CO$_2$-water-energy
\end{tabularx}
\end{center}

\noindent{\bf Example:}
\begin{verbatim}
MODE THC
\end{verbatim}

%========================================================================

\newpage
\protect\hypertarget{target_newt}{}

\subsection{Keyword: NEWTON\_SOLVER}

\noindent{\bf Description:} 

\noindent{\bf Input:}
\begin{deflist}{000}
\item[NEWTON\_SOLVER] ~
\begin{deflist}{000}
\item[TRAN, TRANSPORT (tran\_solver) / DEFAULT (flow\_solver)]
\item[INEXACT\_NEWTON]
\item[NO\_PRINT\_CONVERGENCE]
\item[NO\_INF\_NORM (NO\_INFINITY\_NORM)]
\item[NO\_FORCE\_ITERATION]
\item[PRINT\_DETAILED\_CONVERGENCE]
\item[ATOL]
\item[RTOL]
\item[STOL]
\item[DTOL]
\item[ITOL (INF\_TOL, ITOL\_RES, INF\_TOL\_RES)]
\item[ITOL\_UPDATE (INF\_TOL\_UPDATE)]
\item[MAXIT]
\item[MAXF]
\end{deflist}
\item[(., /, END)]
\end{deflist}

\noindent{\bf Explanation:}

\noindent{\bf Example:}

%========================================================================

\newpage
\protect\hypertarget{target_numjac}{}

\subsection{Keyword: NUMERICAL\_JACOBIAN}
\begin{deflist}{0000000000}
\item[NUMERICAL\_JACOBIAN] \{numerical\_derivatives = .true.\}
\end{deflist}

%========================================================================

\newpage
\protect\hypertarget{target_observation}{}

\subsection{Keyword: OBSERVATION}

{\noindent\bf Description:}
The OBSERVATION card specifies a location (REGION) at which flow and transport results (e.g. pressure, saturation, flow velocities, solute concentrations, etc.) will be monitored in the output.
The user must specify either a region or boundary condition to which the observation object is linked.  The velocity keyword toggles on the printing of velocities at a point in space.

\noindent{\bf Input:}
\begin{deflist}{000}
\item[OBSERVATION] ~
\begin{deflist}{000}
\item[BOUNDARY\_CONDITION] \ {\tt boundary condition name}
\item[REGION] \ {\tt region name}
\item[VELOCITY]
\item[AT\_CELL\_CENTER]
\end{deflist}
\item[(., /, END)]
\end{deflist}

{\noindent\bf Explanation:}
\begin{description}
\item[Keyword OBSERVATION] initiates an observation point entry.
\item[Keyword REGION] (optional) defines the name of the region (usually a point in space) to which the observation point is linked.
\item[Keyword BOUNDARY\_CONDITION] (optional) specifies the name of a boundary condition to which the observation point is tied (e.g. to monitor fluxes across a boundary face).
\item[Keyword VELOCITY] (optional) toggles on the printing of Darcy velocities at the observation point.
\end{description}

{\noindent\bf Examples:}
\begin{verbatim}
OBSERVATION
REGION well1
VELOCITY
END

OBSERVATION
BOUNDARY_CONDITION river
END
\end{verbatim}

%========================================================================

\newpage
\protect\hypertarget{target_orig}{}

\subsection{Keyword: ORIGIN (ORIG)}
\begin{deflist}{0000000000}
\item[ORIGIN (ORIG)] X\_DIRECTION Y\_DIRECTION Z\_DIRECTION
\end{deflist}

%========================================================================


\newpage
\protect\hypertarget{target_output}{}

\subsection{Keyword: OUTPUT}

\noindent {\bf Description:} The {\bf OUTPUT} keyword controls formatting and time of output.

\noindent {\bf Input:}

\begin{deflist}{000}
\item[OUTPUT] ~
\begin{deflist}{000}
\item[TIMES] {\tt Unit (s, h, y)} \ \ \ {\tt Value} 
\item[SCREEN OFF]
\item[SCREEN PERIODIC] {\tt Value}
\item[PERIODIC TIME] {\tt Value} \ {\tt Unit}
\item[PERIODIC TIMESTEP] {\tt Value} \ {\tt Unit}
\item[PERIODIC\_OBSERVATION TIME] {\tt Value} \ {\tt Unit}
\item[PERIODIC\_OBSERVATION TIMESTEP] {\tt Value}
\item[NO\_FINAL, NO\_PRINT\_FINAL]
\item[FORMAT TECPLOT POINT]
\item[FORMAT TECPLOT BLOCK]
\item[FORMAT HDF5]
\item[FORMAT MAD]
\item[FORMAT VTK]
\item[PERMEABILITY]
\item[POROSITY]
\item[FLUXES]
\item[VELOCITIES]
\item[MASS\_BALANCE]
\end{deflist}
\item[(., /, END)]
\end{deflist}

\noindent {\bf Explanation:}

\begin{center}
\begin{tabularx}{\linewidth}{lX}
OUTPUT: & keyword to control output.\\
TIMES: & list of output times.\\
SCREEN OFF: & turns off screen output\\
SCREEN PERIODIC: & controls screen output frequency.\\
PERIODIC TIME: & controls frequency of output times.\\
PERIODIC TIMESTEP: & controls frequency of output time steps.\\
PERIODIC\_OBSERVATION TIME: & frequency of output time.\\
PERIODIC\_OBSERVATION TIMESTEP: & frequency of output time step.\\
NO\_FINAL, NO\_PRINT\_FINAL: & \\
FORMAT TECPLOT POINT: & Tecplot POINT output, valid for 1D and 2D problems.\\
FORMAT TECPLOT BLOCK: & Tecplot BLOCK output.\\
FORMAT HDF5: & HDF5 output format written to a .h5 file which can be read by Visit.\\
FORMAT MAD: & MAD (Method of Anchored Distributions) format.\\
FORMAT VTK: & VTK format.\\
UNIT: & time units of seconds (s), days (d), and years (y).\\
PERMEABILITY: & \\
POROSITY: & \\
FLUXES: & \\
VELOCITIES: & keyword to output velocities.\\
MASS\_BALANCE: & keyword to output global mass balances and boundary fluxes.
\end{tabularx}
\end{center}

\noindent {\bf Examples:}

\begin{verbatim}
OUTPUT
:SCREEN PERIODIC 10
:PERIODIC TIME 10 h
PERIODIC_OBSERVATION TIMESTEP 1
:times h 1.
:PERIODIC_OBSERVATION TIME 50 h
FORMAT TECPLOT POINT
FORMAT HDF5
VELOCITIES
MASS_BALANCE
/
\end{verbatim}

%========================================================================

\newpage
\protect\hypertarget{target_overwrite}{}

\subsection{Keyword: OVERWRITE\_RESTART\_TRANSPORT}
%\shadowbox{
%\begin{minipage}{6.5in}
\begin{deflist}{0000000000}
\item[OVERWRITE\_RESTART\_TRANSPORT] \{overwrite\_restart\_transport = .true.\}
\end{deflist}
%\end{minipage}
%}

%================================================================

\newpage
\protect\hypertarget{target_region}{}


\subsection{Keyword: REGION}

{\noindent\bf Description:}
The {\bf REGION} keyword defines a set of grid cells encompassed by a volume or intersected by a plane or point, or a list of grid cell ids.  The {\bf REGION} name can then be used to link this set of grid cells to material properties, strata, boundary and initial conditions, source sinks, observation points, etc.  Although a region may be defined though the use of (I, J, K) indices using the {\bf BLOCK} keyword, the user is encouraged to define regions either through {\bf COORDINATES} or lists read in from an HDF5 file in order to minimize the dependence of the input file on grid resolution.  In the case of the {\bf FILE} keyword, a list of grid cell ids is read from an HDF5 file where the {\tt region\_name} defines the HDF5 data set.  It should be noted that given a region defined by a plane or point shared by two grid cells (e.g. a plane defining the surface between two grid cells), {\bf PFLOTRAN} will select the upwind cell(s) as the region.

{\noindent\bf Input:}

\begin{deflist}{000}
\item[REGION] {\tt region\_name}
\begin{deflist}{000}
\item[FILE] \ {\tt file\_name}
\item[LIST] \ (to be implemented)
\item[FACE] \ {\tt face\_name}
\item[BLOCK] \ {\tt i1 \ i2 \ j1 \ j2 \ k1 \ k2}
\item[COORDINATE] \ {\tt x \ y \ z}
\item[COORDINATES] ~
\begin{deflist}{000}
\item[\tt x1 y1 z1]
\item[\tt x2 y2 z2]
\item[\tt x3 y3 z3]
\end{deflist}
\item[(., /, END)]
\end{deflist}
\item[(., /, END)]
\end{deflist}

{\noindent\bf Explanation:}
\begin{description}
\item[Keyword REGION] begins a region entry with name {\tt region\_name}.
\item[Keyword BLOCK] defines a volumetric, planar, or point region through IJK indices: i1 i2 j1 j2 k1 k2.
\item[Keyword COORDINATE] defines a point region through coordinates in 3D space.
\item[Keyword COORDINATES] Defines a volumetric, planar, or point region between two coordinates in space.
\item[Keyword FILE] Defines an HDF5 file within which a dataset named
region\_name contains a list of grid cells corresponding to a region.
\item [Keyword FACE] Defines the face of the grid cell to which boundary conditions are connected where face\_name is one of WEST, EAST, NORTH, SOUTH, BOTTOM, TOP (structured grids only).
\item [Keyword END] Ends the region entry (can be one of . \ END).
\end{description}

{\noindent\bf
Examples:}

\begin{verbatim}
REGION source_zone
BLOCK 3 5 15 16 2 3
END

REGION source_zone
BLOCK
3 5 15 16 2 3
END

REGION west_boundary
BLOCK 1 1 1 30 1 50
FACE WEST
END

REGION source_zone
COORDINATES
50. 10. 10.
60. 15. 15.
/
END

REGION river_boundary
FILE ./regions.h5
FACE EAST
END

REGION well
COORDINATE 50. 10. 10.
END

REGION well
COORDINATE
50. 10. 10.
END

REGION west_boundary
COORDINATES
0. 0. 0.
0. 10. 10.
/
FACE WEST
END
\end{verbatim}

%========================================================================

\newpage
\protect\hypertarget{target_restart}{}

\subsection{Keyword: RESTART}
{\noindent\bf Description}

The RESTART card defines a checkpoint file from which the current simulation should be restarted.  If a time is specified after the file name, the initial simulation time is set to that time.

{\noindent\bf Input:}

%\shadowbox{
%\begin{minipage}{6.5in}
\begin{deflist}{000}
\item[RESTART] restart\_file restart\_time time\_units
\end{deflist}
%\end{minipage}
%}

{\noindent\bf Explanation:}
\begin{description}
\item[Keyword RESTART] defines the checkpoint filename to be read in to restart a simulation at the specified time.
\end{description}

{\noindent\bf
Examples:}

\begin{verbatim}
RESTART restart.chk 0. y

RESTART restart.chk 

\end{verbatim}

%========================================================================

\newpage
\protect\hypertarget{target_sat}{}

\subsection{Keyword: SATURATION\_FUNCTION}

\noindent{\bf Description:}

\noindent{\bf Input:}
\begin{deflist}{000}
\item[SATURATION\_FUNCTION] \ {\tt Name}

\begin{deflist}{000}
\item[SATURATION\_FUNCTION\_TYPE] \ {\bf VAN\_GENUCHTEN}
\item[SATURATION\_FUNCTION\_TYPE] \ {\bf MUALEM}
\item[PERMEABILITY\_FUNCTION\_TYPE] \ {\bf VAN\_GENUCHTEN}
\item[PERMEABILITY\_FUNCTION\_TYPE] \ {\bf MUALEM}
\item[RESIDUAL\_SATURATION] \ {\tt Value}
\item[LAMBDA] \ {\tt Value}
\item[ALPHA] \ {\tt Value}
\item[MAX\_CAPILLARY\_PRESSURE] \ {\tt Value}
\item[BETAC] \ {\tt Value}
\item[POWER] \ {\tt Value}
\end{deflist}
\item[(., /, END)]
\end{deflist}

\noindent{\bf Explanation:}

\noindent{\bf Example:}
\begin{verbatim}
SATURATION_FUNCTION sf1
SATURATION_FUNCTION_TYPE VAN_GENUCHTEN
RESIDUAL_SATURATION 0.1d0
LAMBDA 2.67d0
ALPHA 2.042d-4
MAX_CAPILLARY_PRESSURE 1d8
BETAC 0.d0
POWER 1.d0
/
\end{verbatim}

%========================================================================

\newpage
\protect\hypertarget{target_src}{}

\subsection{Keyword: SOURCE\_SINK}
\begin{deflist}{0000000000}
\item[SOURCE\_SINK]
\item  REGION region\_name
\item  CONDITION condition\_name
\item  TYPE [initial, boundary, source\_sink]
\item  FACE [WEST, EAST, NORTH, SOUTH, BOTTOM, TOP]
\item END
\end{deflist}

%========================================================================

\newpage
\protect\hypertarget{target_strata}{}

\subsection{Keyword: STRATIGRAPHY (STRATA)}
%\shadowbox{
%\begin{minipage}{6.5in}
\begin{deflist}{0000000000}
\item[STRATIGRAPHY (STRATA)]
\item  REGION region\_name
\item  MATERIAL material\_name
\item  INACTIVE
\item[(., /, END)]
\end{deflist}
%\end{minipage}
%}

%========================================================================

\newpage
\protect\hypertarget{target_time}{}

\noindent {\bf Description:} the keyword {\bf TIME} controls the simulation time.

\noindent {\bf Input:}
\subsection{Keyword: TIME}
\begin{deflist}{000}
\item[TIME] ~
\begin{deflist}{000}
\item[FINAL\_TIME] {\tt Value} \ {\tt Unit (s, m, h, d, mo, y)}
\item[INITIAL\_TIMESTEP\_SIZE] {\tt Value} \ {\tt Unit (s, m, h, d, mo, y)}
\item[MAXIMUM\_TIMESTEP\_SIZE] {\tt Value} \ {\tt Unit (s, m, h, d, mo, y)}
\item[STEADY\_STATE]
\end{deflist}
\item[(., /, END)]
\end{deflist}

\noindent{\bf Explanation:}

\noindent{\bf Example:}
\begin{verbatim}
TIME
FINAL_TIME 100. h
INITIAL_TIMESTEP_SIZE 1.d-3 h
MAXIMUM_TIMESTEP_SIZE 1.d0 h
/
\end{verbatim}
%================================================================

\newpage
\protect\hypertarget{target_timestep}{}

\subsection{Keyword: TIMESTEPPER}

\noindent{\bf Description:} the keyword {\bf TIMESTEPPER} controls time stepping.

\noindent{\bf Input:}
\begin{deflist}{000}
\item[TIMESTEPPER] [{\bf FLOW, TRAN, TRANSPORT}]
\begin{deflist}{000}
\item[NUM\_STEPS\_AFTER\_TS\_CUT] [5]
\item[MAX\_STEPS] [999999]
\item[TS\_ACCELERATION] [5]
\item[MAX\_TS\_CUTS] [16]
\item[INITIALIZE\_TO\_STEADY\_STATE]
\item[RUN\_AS\_STEADY\_STATE]
\item[MAX\_PRESSURE\_CHANGE] [5.d4]
\item[MAX\_TEMPERATURE\_CHANGE] [5.d0]
\item[MAX\_CONCENTRATION\_CHANGE] [1.d0]
\item[MAX\_SATURATION\_CHANGE] [0.5d0]
\end{deflist}
\item[(., /, END)]
\end{deflist}

\noindent{\bf Explanation:}

\noindent{\bf Example:}

%========================================================================

\newpage
\protect\hypertarget{target_trans_cond}{}

\subsection{Keyword: TRANSPORT\_CONDITION}

\noindent{\bf Description:}

\noindent{\bf Input:}
\begin{deflist}{000}
\item[TRANSPORT\_CONDITION] \ {\tt Name}

\begin{deflist}{000}
\item[TYPE] [{\bf dirichlet, dirichlet\_zero\_gradient, equilibrium, neumann, mole, mole\_rate, zero\_gradient}]
\item[TIME] \ {\tt Value}
\item[UNITS] \ {\bf s, sec, min, hr, d, day, y, yr}
\item[CONSTRAINT\_LIST]
\item {\tt time} \ {\tt constraint\_name}
\item[(., /, END)]
\item[CONSTRAINT] \ {\tt constraint\_name}
\end{deflist}
\item[(., /, END)]
\end{deflist}

\noindent{\bf Explanation:}

\noindent{\bf Example:}

%========================================================================

\newpage
\protect\hypertarget{target_unifvel}{}

\subsection{Keyword: UNIFORM\_VELOCITY \hfill Optional}
%\addcontentsline{toc}{subsection}{\thesubsection\ Keyword: UNIFORM\_VELOCITY} 

\noindent{\bf Description:}

\noindent{\bf Input:}
\begin{deflist}{000}
\item[UNIFORM\_VELOCITY] \ {\tt vlx \ vly \ vlz} \ [m/s]
\end{deflist}

\noindent{\bf Explanation:} Set uniform velocity for transport mode.

\noindent{\bf Example:}
\begin{verbatim}
UNIFORM_VELOCITY 3.84259d-6 0.d0 0.d0  ! 1.38333 cm/h
\end{verbatim}

%========================================================================

\newpage
\protect\hypertarget{target_touch}{}

\subsection{Keyword: USE\_TOUCH\_OPTIONS}

\noindent{\bf Description:}

\noindent{\bf Input:}
\begin{deflist}{000}
\item[USE\_TOUCH\_OPTIONS] \{use\_touch\_options = .true.\}
\end{deflist}

\noindent{\bf Explanation:}

\noindent{\bf Example:}

%========================================================================

\newpage
\protect\hypertarget{target_veldata}{}

\subsection{Keyword: VELOCITY\_DATASET}

\noindent{\bf Description:} Set time-dependent velocity for transport mode.

\noindent{\bf Input:}
\begin{deflist}{000}
\item[VELOCITY\_DATASET] ~
\begin{deflist}{000}
\item[UNITS] \ cm/h
\item[CYCLIC]
\item[INTERPOLATION] \ {\bf step} [default]
\item[INTERPOLATION] \ {\bf linear}
\item[VELOCITY] ~
\begin{deflist}{000}
\item[{\tt Time \ velx \ vely \ velz}]
\end{deflist}
\item[(., /, END)]
\end{deflist}
\item[(., /, END)]
\end{deflist}

\noindent{\bf Explanation:}

\noindent{\bf Example:}
\begin{verbatim}
VELOCITY_DATASET
UNITS cm/h
CYCLIC ! cycles the data set using last time as offset
:INTERPOLATION STEP ! interpolation method (step [default] or linear)
VELOCITY
:time velx vely velz
:time units = time unit in velocity units
0.d0       1.38333d0 0.d0 0.d0 
12.d0     -1.38333d0 0.d0 0.d0 
24.d0      1.38333d0 0.d0 0.d0 
/
/
\end{verbatim}

%========================================================================

\newpage
\protect\hypertarget{target_wallclk}{}

\subsection{Keyword: WALLCLOCK\_STOP}
%\shadowbox{
%\begin{minipage}{6.5in}
\begin{deflist}{0000000000}
\item[WALLCLOCK\_STOP] wallclock\_stop\_time
\end{deflist}

\noindent{\bf Explanation:}

\noindent{\bf Example:}

%========================================================================
%========================================================================
%========================================================================
\newpage
\section*{Example Input File}

%\footnotesize
\begin{verbatim}
:Description: 3D toy problem for richards equation
:
MODE RICHARDS_LITE
TRAN 1
:
CHECKPOINT 1000
RESTART steady_68_125_20_Scope3.chk 0.d0
OVERWRITE_RESTART_TRANSPORT
OVERWRITE_RESTART_FLOW_PARAMS
WALLCLOCK_STOP 3.95
:
GRID
TYPE structured
NXYZ 68 125 20
END
:
ORIGIN 0.d0 0.d0 90.d0
:
NEWTON_SOLVER FLOW
:RTOL 1.d-50
RTOL 1.d-5
:ATOL 1.d-50
ATOL 1.d-8
:STOL 1.d-50
STOL 1.d-6
:ITOL_RES 1.d-4
:ITOL_UPDATE 0.5d0  ! Pa
NO_INFINITY_NORM
NO_FORCE_ITERATION
:NO_PRINT_CONVERGENCE
:PRINT_DETAILED_CONVERGENCE
MAXIT 20
END
:
NEWTON_SOLVER TRANSPORT
:RTOL 1.d-50
RTOL 1.d-6
:ATOL 1.d-50
:STOL 1.d-50
:STOL 1.d-6
:ITOL_RES 1.d-8
NO_INFINITY_NORM
:NO_FORCE_ITERATION
:NO_PRINT_CONVERGENCE
:PRINT_DETAILED_CONVERGENCE
MAXIT 10
END
:
LINEAR_SOLVER FLOW
:KSP_TYPE gmres
:RTOL 1.d-50
:ATOL 1.d-10
END
:
TIMESTEPPER
TS_ACCELERATION 8
END
:
:HDF5 !VELO !FLUX
TECP VELO !FLUX
:
DXYZ
19.8529411765d0
20.d0
1.d0
:
:  d0[m^2/s] delhaq[kJ/mol]
DIFF  1.D-9       12.6
:
: Richards Equation Pref
RICH 101325.
:
SATURATION_FUNCTIONS
: van Genuchten
:id itype swir   m      alpha     pcwmax betac pwr
 1  1     0.1600 0.3391 7.2727d-4 1.e8   0.d0  1.d0
 2  1     0.1299 0.7479 1.4319d-4 1.e8   0.d0  1.d0
END

THERMAL_PROPERTIES
:ithm rho    cpr     ckdry cksat tau cdiff   cexp
  1   2.76e3 1000.e0 0.5   0.5   0.5 2.13d-5 1.8
END
:
MATERIALS
:name   id icap ithm por  tau permx     permy     permz     permpwr
:Hanford 1  1    1    0.20 0.5 7.387d-9  7.387d-9  7.387d-10 1.d0
Hanford 1  1    1    0.20 0.5 7.387d-10 7.387d-10 7.387d-10 1.d0
Unit2   2  2    1    0.25 0.5 0.d0      0.d0      0.d0      1.d0
Unit3   3  2    1    0.25 0.5 0.d0      0.d0      0.d0      1.d0
Unit4   4  2    1    0.25 0.5 0.d0      0.d0      0.d0      1.d0
Unit5   5  2    1    0.25 0.5 4.221d-11 4.221d-11 4.221d-12 1.d0
Unit6   6  2    1    0.25 0.5 1.052d-14 1.052d-14 1.052d-15 1.e0
Unit7   7  2    1    0.25 0.5 4.523d-11 4.523d-11 4.523d-12 1.e0
Unit8   8  2    1    0.25 0.5 5.259d-17 5.259d-17 5.259d-18 1.e0
Unit9   9  2    1    0.25 0.5 5.259d-17 5.259d-17 5.259d-18 1.e0
Unit10  10 2    1    0.25 0.5 1.d-20    1.d-20    1.d-20    1.e0
END
:
:TIME h
:100000.
TIME h every 168
7500.
:
:DTST 1.d-6
:1. 100000.
DTST 1.d-2
1. 1.d0
:
REGION all
BLOCK 1 68 1 125 1 20
END

REGION West
FILE input_68_125_20_Scope3.h5
END

REGION East
FILE input_68_125_20_Scope3.h5
END

REGION North
FILE input_68_125_20_Scope3.h5
END

REGION South
FILE input_68_125_20_Scope3.h5
END

REGION Top
FILE input_68_125_20_Scope3.h5
END

REGION North_Pond_West_Trench
FILE input_68_125_20_Scope3.h5
END

REGION North_Pond_East_Trench
FILE input_68_125_20_Scope3.h5
END

:REGION Plume
:FILE input_68_125_20_Scope3.h5
:END

REGION Plume_Source
FILE input_68_125_20_Scope3.h5
:BLOCK 60 62 70 72 15 17
END

REGION 399-1-1
COORDINATE 1208.69 1784.40 100.0
END

REGION 399-1-2
COORDINATE 876.57 1599.94 100.0
END

REGION 399-2-1
COORDINATE 1199.61 1304.67 100.0
END

REGION 399-2-2
COORDINATE 1159.62 1480.95 100.0
END

REGION 399-3-9
COORDINATE 1186.58 1098.39 100.0
END

REGION 399-3-12
COORDINATE 911.44 1196.74 100.0
END

REGION 399-4-1
COORDINATE 870.96 784.91 100.0
END

REGION 399-4-7
COORDINATE 1179.54 661.80 100.0
END

REGION 399-4-9
COORDINATE 1176.30 919.13 100.0
END

REGION 399-5-1
COORDINATE 360.22 899.80 100.0
END

REGION ifc
COORDINATE 979.57 1302.39 100.0
END

BREAKTHROUGH
REGION 399-1-1
END

BREAKTHROUGH
REGION 399-1-2
END

BREAKTHROUGH
REGION 399-2-1
END

BREAKTHROUGH
REGION 399-2-2
END

BREAKTHROUGH
REGION 399-3-9
END

BREAKTHROUGH
REGION 399-3-12
END

BREAKTHROUGH
REGION 399-4-1
END

BREAKTHROUGH
REGION 399-4-9
END

BREAKTHROUGH
REGION 399-5-1
END

BREAKTHROUGH
REGION ifc
VELOCITY
END

CONDITION initial
UNITS Pa,C,M,yr
CLASS flow
TYPE
PRESSURE hydrostatic
TEMPERATURE dirichlet
CONCENTRATION dirichlet
END
:DATUM file initial_data.datum
DATUM 1.2294770e+003 9.4611630e+002 1.0559964e+002
GRADIENT
:PRESSURE file initial_data.gradient
PRESSURE 1.9538819e-004 2.8407287e-004 0.0000000e+000
END
PRESSURE 101325.d0
TEMPERATURE 25.d0
CONCENTRATION 1.d-6
END

CONDITION river
CLASS flow
TYPE
PRESSURE seepage
TEMPERATURE dirichlet
CONCENTRATION dirichlet
END
INTERPOLATION linear
DATUM file river.datum
:DATUM  1.2294770e+003 9.4611630e+002 1.0559964e+002
GRADIENT
PRESSURE file river.gradient_adj
:PRESSURE 0.d0 0.00027389 0.d0
END
PRESSURE 101325.d0
TEMPERATURE 25.d0
CONCENTRATION 1.d-6
END

CONDITION west
CLASS flow
TYPE
PRESSURE hydrostatic
TEMPERATURE dirichlet
CONCENTRATION dirichlet
END
INTERPOLATION linear
DATUM file well_data.datum
:DATUM 2.1149088e+002 1.2272915e+003 1.0548061e+002
GRADIENT
PRESSURE file well_data.gradient
:PRESSURE -7.8414067e-004 3.4105428e-004 0.0000000e+000
END
PRESSURE 101325.d0
TEMPERATURE 25.d0
CONCENTRATION 1.d-6
END

CONDITION north
CLASS flow
TYPE
PRESSURE hydrostatic
TEMPERATURE dirichlet
CONCENTRATION dirichlet
END
INTERPOLATION linear
DATUM file north.datum
:DATUM 0. 2500. 106.0805113
GRADIENT
PRESSURE file north.gradient
:PRESSURE -4.09472e-05 0. 0.
END
PRESSURE 101325.d0
TEMPERATURE 25.d0
CONCENTRATION 1.d-6
END

CONDITION south
CLASS flow
TYPE
PRESSURE hydrostatic
TEMPERATURE dirichlet
CONCENTRATION dirichlet
END
INTERPOLATION linear
DATUM file south.datum
:DATUM 0       0       105.2278756
GRADIENT
PRESSURE file south.gradient
:PRESSURE 8.34319e-05 0. 0.
END
PRESSURE 101325.d0
TEMPERATURE 25.d0
CONCENTRATION 1.d-6
END

CONDITION recharge
CLASS flow
TYPE
PRESSURE neumann
TEMPERATURE dirichlet
CONCENTRATION dirichlet
END
FLUX file recharge.txt
:FLUX 1.756d-9
TEMPERATURE 25.d0
CONCENTRATION 1.d-6
END

CONDITION plume
CLASS flow
TYPE
PRESSURE neumann
END
FLUX 0.d0
END

CONDITION north_pond_west_trench
CLASS flow
TYPE
PRESSURE neumann
TEMPERATURE dirichlet
CONCENTRATION dirichlet
END
FLUX file north_pond_west_trench.txt
:FLUX  4.500d-5
TEMPERATURE 25.d0
CONCENTRATION 1.d-6
END

CONDITION north_pond_east_trench
CLASS flow
TYPE
PRESSURE neumann
TEMPERATURE dirichlet
CONCENTRATION dirichlet
END
FLUX file north_pond_east_trench.txt
:FLUX 0.d0
TEMPERATURE 25.d0
CONCENTRATION 1.d-6
END

CONDITION river_c
CLASS transport
CONCENTRATION 1.d-40
END

CONDITION west_c
CLASS transport
CONCENTRATION 1.d-40
END

CONDITION initial_c
CLASS transport
CONCENTRATION 1.d-40
END

CONDITION plume_c
CLASS transport
CONCENTRATION 1.d0
END

: initial condition
INITIAL_CONDITION
FLOW_CONDITION initial
TRANSPORT_CONDITION initial_c
REGION all
END

: inland boundary condition
BOUNDARY_CONDITION
FLOW_CONDITION west
TRANSPORT_CONDITION initial_c
REGION West
END

: river boundary condition
BOUNDARY_CONDITION
FLOW_CONDITION river
TRANSPORT_CONDITION initial_c
REGION East
END

: north boundary condition
BOUNDARY_CONDITION
FLOW_CONDITION north
TRANSPORT_CONDITION initial_c
REGION North
END

: south boundary condition
BOUNDARY_CONDITION
FLOW_CONDITION south
TRANSPORT_CONDITION initial_c
REGION South
END

: recharge boundary condition
BOUNDARY_CONDITION
FLOW_CONDITION recharge
TRANSPORT_CONDITION initial_c
REGION Top
END

: north pond west trench boundary condition
BOUNDARY_CONDITION
FLOW_CONDITION north_pond_west_trench
TRANSPORT_CONDITION initial_c
REGION North_Pond_West_Trench
END

: north pond east trench boundary condition
BOUNDARY_CONDITION
FLOW_CONDITION north_pond_east_trench
TRANSPORT_CONDITION initial_c
REGION North_Pond_East_Trench
END

: plume source
SOURCE_SINK
FLOW_CONDITION plume
TRANSPORT_CONDITION plume_c
REGION Plume_Source
END

STRATA
MATERIAL input_68_125_20_Scope3.h5
END

:read in permeability field-------------------------------------------------

DATASET permx perm_inv.dat
DATASET permy perm_inv.dat
DATASET permz perm_inv.dat

\end{verbatim}

%%%%%%%%%%%%%%%%%%%%%%%%%%%%%%%%%%%%%%%%%%%%%%%%%%%%%%%%%%%%%%%%%%%%%

\begin{comment}
\section{PFLOTRAN Objects}

This section gives an overview in alphabetical order of the objects and their data structures used in PFLOTRAN. The upper most object is {\bf Simulation} followed by {\bf Realization}, followed by {\bf Level}, {\bf Patch} and {\bf Grid}.

\begin{longtable}{ll}%[h]%\centering
%\caption{PFLOTRAN Keywords}\label{flkeywd}

%\vspace{4mm}

%\begin{tabular}{lll}
\toprule[1.5pt]
\multicolumn{1}{c}{Object} & \multicolumn{1}{c}{Description}\\
\midrule[1pt]
\hyperlink{target_obj_breakthrough}{Breakthrough} & \\
\hyperlink{target_obj_cond}{Condition} & \\
\hyperlink{target_obj_conn}{Connection} & \\
\hyperlink{target_obj_coupler}{Coupler} & \\
\hyperlink{target_obj_discretization}{Discretization} & \\
\hyperlink{target_obj_field}{Field} & \\
\hyperlink{target_obj_grid}{Grid} & \\
\hyperlink{target_obj_level}{Level} & \\
\hyperlink{target_obj_logging}{Logging} & \\
\hyperlink{target_obj_mat}{Material} & \\
\hyperlink{target_obj_option}{Option} & \\
\hyperlink{target_obj_patch}{Patch} & \\
\hyperlink{target_obj_real}{Realization} & \\
\hyperlink{target_obj_region}{Region} & \\
\hyperlink{target_obj_richards}{Richards} & \\
\hyperlink{target_obj_richards_lite}{Richards\_lite} & \\
\hyperlink{target_obj_simulation}{Simulation} & \\
\hyperlink{target_obj_solver}{Solver} & \\
\hyperlink{target_obj_stepper}{Stepper} & \\
\hyperlink{target_obj_strata}{Strata} & \\
\hyperlink{target_obj_structured_grid}{Structured\_grid} & \\
\hyperlink{target_obj_waypoint}{Waypoint} & \\
\bottomrule[1.5pt]
%\end{tabular}
\end{longtable}

%%%%%%%%%%%%%%%%%%%%%%%%%%%%%%%%%%%%%%%%%%%%%%%%%%%%%%%%%%%%%%%%%%%%%

\protect\hypertarget{target_obj_breakthrough}{}
\subsection{Breakthrough}

\begin{verbatim}
  type, public :: breakthrough_type
    ! all added variables must be included in BreakthroughCreateFromBreakthrough
    PetscInt :: id
    PetscTruth :: print_velocities
    character(len=MAXWORDLENGTH) :: name
    character(len=MAXWORDLENGTH) :: region_name
    type(region_type), pointer :: region
    type(breakthrough_type), pointer :: next
  end type breakthrough_type

  type, public :: breakthrough_list_type
    PetscInt :: num_breakthroughs
    type(breakthrough_type), pointer :: first
    type(breakthrough_type), pointer :: last
    type(breakthrough_type), pointer :: array(:)
  end type breakthrough_list_type
\end{verbatim}

%%%%%%%%%%%%%%%%%%%%%%%%%%%%%%%%%%%%%%%%%%%%%%%%%%%%%%%%%%%%%%%%%%%%%

\protect\hypertarget{target_obj_cond}{}
\subsection{Condition}

\begin{verbatim}
  type, public :: condition_dataset_type
    PetscInt :: rank
    logical :: is_transient
    logical :: is_cyclic
    PetscInt :: interpolation_method
    PetscReal, pointer :: times(:)
    PetscReal, pointer :: values(:,:)
    PetscReal, pointer :: cur_value(:)
    PetscInt :: cur_time_index
    PetscInt :: max_time_index
  end type condition_dataset_type

  type, public :: condition_type
    PetscInt :: id                    ! id from which condition can be referenced
    character(len=MAXWORDLENGTH) :: class ! character string describing class of
                                            condition
    PetscInt :: iclass                            ! integer id for class
    logical :: sync_time_with_update
    character(len=MAXWORDLENGTH) :: name ! name of condition (e.g. initial,
                                           recharge)
    PetscInt :: num_sub_conditions
    PetscInt :: iphase
    PetscInt, pointer :: itype(:)
    character(len=MAXWORDLENGTH) :: time_units
    character(len=MAXWORDLENGTH) :: length_units
    type(sub_condition_type), pointer :: pressure
    type(sub_condition_type), pointer :: temperature
    type(sub_condition_type), pointer :: concentration
    type(sub_condition_type), pointer :: enthalpy
    type(sub_condition_ptr_type), pointer :: sub_condition_ptr(:)
    type(condition_type), pointer :: next ! pointer to next condition_type for
                                            linked-lists
  end type condition_type

  type, public :: sub_condition_type
    PetscInt :: itype                  ! integer describing type of condition
    character(len=MAXWORDLENGTH) :: ctype ! character string describing type of
                                            condition
    character(len=MAXWORDLENGTH) :: units      ! units

    type(condition_dataset_type) :: datum
    type(condition_dataset_type) :: gradient
    type(condition_dataset_type) :: dataset

  end type sub_condition_type

  type, public :: sub_condition_ptr_type
    type(sub_condition_type), pointer :: ptr
  end type sub_condition_ptr_type

  type, public :: condition_ptr_type
    type(condition_type), pointer :: ptr
  end type condition_ptr_type

  type, public :: condition_list_type
    PetscInt :: num_conditions
    type(condition_type), pointer :: first
    type(condition_type), pointer :: last
    type(condition_ptr_type), pointer :: array(:)
  end type condition_list_type
\end{verbatim}

%%%%%%%%%%%%%%%%%%%%%%%%%%%%%%%%%%%%%%%%%%%%%%%%%%%%%%%%%%%%%%%%%%%%%

\protect\hypertarget{target_obj_conn}{}
\subsection{Connection}

\begin{verbatim}
  type, public :: connection_set_type
    PetscInt :: id
    PetscInt :: itype                  ! connection type (boundary, internal,
                                         source sink
    PetscInt :: num_connections
    PetscInt, pointer :: id_up(:)      ! list of ids of upwind cells
    PetscInt, pointer :: id_dn(:)      ! list of ids of downwind cells
    PetscReal, pointer :: dist(:,:)    ! list of distance vectors,
                                         size(-1:3,num_connections) where
                                      !   -1 = fraction upwind
                                      !   0 = magnitude of distance
                                      !   1-3 = components of unit vector
    PetscReal, pointer :: area(:)        ! list of areas of faces normal to
                                           distance vectors
!    PetscReal, pointer :: velocity(:,:)  ! velocity scalars for each phase
    type(connection_set_type), pointer :: next
  end type connection_set_type


  ! pointer data structure required for making an array of region pointers in F90
  type, public :: connection_set_ptr_type
    type(connection_set_type), pointer :: ptr           ! pointer to the
                                                          connection_set_type
  end type connection_set_ptr_type

  type, public :: connection_set_list_type
    PetscInt :: num_connection_objects
    type(connection_set_type), pointer :: first
    type(connection_set_type), pointer :: last
    type(connection_set_ptr_type), pointer :: array(:)
  end type connection_set_list_type
\end{verbatim}

%%%%%%%%%%%%%%%%%%%%%%%%%%%%%%%%%%%%%%%%%%%%%%%%%%%%%%%%%%%%%%%%%%%%%

\protect\hypertarget{target_obj_coupler}{}
\subsection{Coupler}

\begin{verbatim}
  type, public :: coupler_type
    PetscInt :: id                                       ! id of coupler
    PetscInt :: itype                                    ! integer defining type
    character(len=MAXWORDLENGTH) :: ctype        ! character string defining type
    character(len=MAXWORDLENGTH) :: condition_name      ! character string
                                         defining name of condition to be applied
    character(len=MAXWORDLENGTH) :: region_name         ! character string
                                         defining name of region to be applied
    PetscInt :: icondition          ! id of condition in condition array/list
    PetscInt :: iregion             ! id of region in region array/list
    PetscInt :: iface                               ! for structured grids only
    PetscInt, pointer :: aux_int_var(:,:) ! auxilliary array for integer value
    PetscReal, pointer :: aux_real_var(:,:) ! auxilliary array for real values
    type(condition_type), pointer :: condition          ! pointer to condition in
                                                          condition array/list
    type(region_type), pointer :: region                ! pointer to region in
                                                          region array/list
    type(connection_type), pointer :: connection        ! pointer to an array/list
                                                          of connections
    type(coupler_type), pointer :: next                 ! pointer to next coupler
  end type coupler_type

  type, public :: coupler_ptr_type
    type(coupler_type), pointer :: ptr
  end type coupler_ptr_type

  type, public :: coupler_list_type
    PetscInt :: num_couplers
    type(coupler_type), pointer :: first
    type(coupler_type), pointer :: last
    type(coupler_ptr_type), pointer :: array(:)
  end type coupler_list_type
\end{verbatim}

%%%%%%%%%%%%%%%%%%%%%%%%%%%%%%%%%%%%%%%%%%%%%%%%%%%%%%%%%%%%%%%%%%%%%

\protect\hypertarget{target_obj_discretization}{}
\subsection{Discretization}

\begin{verbatim}
  type, public :: discretization_type
    PetscInt :: itype ! type of discretization (e.g. structured, unstructured,
                        etc.)
    character(len=MAXWORDLENGTH) :: ctype
    type(grid_type), pointer :: grid  ! pointer to a grid object
    type(amrgrid_type), pointer :: amrgrid  ! pointer to an amr grid object
    DM :: dm_1_dof, dm_nflowdof, dm_ntrandof
  end type discretization_type
\end{verbatim}

%%%%%%%%%%%%%%%%%%%%%%%%%%%%%%%%%%%%%%%%%%%%%%%%%%%%%%%%%%%%%%%%%%%%%

\protect\hypertarget{target_obj_field}{}
\subsection{Field}

\begin{verbatim}
  type, public :: field_type

!geh material id
    ! 1 degree of freedom
    Vec :: porosity0, porosity_loc
    Vec :: tor_loc
    Vec :: ithrm_loc
    Vec :: icap_loc
    Vec :: iphas_loc, iphas_old_loc

    Vec :: perm_xx_loc, perm_yy_loc, perm_zz_loc
    Vec :: perm0_xx, perm0_yy, perm0_zz, perm_pow

    Vec :: saturation_loc, density_loc

    Vec :: volume

    ! residual vectors
    Vec :: flow_r
    Vec :: tran_r

    ! Solution vectors
    Vec :: flow_xx, flow_xx_loc, flow_dxx, flow_yy, flow_accum
    Vec :: tran_xx, tran_xx_loc, tran_dxx, tran_yy, tran_accum

  end type field_type
\end{verbatim}

%%%%%%%%%%%%%%%%%%%%%%%%%%%%%%%%%%%%%%%%%%%%%%%%%%%%%%%%%%%%%%%%%%%%%

\protect\hypertarget{target_obj_grid}{}
\subsection{Grid}

\begin{verbatim}
  type, public :: grid_type

    character(len=MAXWORDLENGTH) :: ctype
    PetscInt :: itype  ! type of grid (e.g. structured, unstructured, etc.)

    PetscInt :: nmax   ! Total number of nodes in global domain
    PetscInt :: nlmax  ! Total number of non-ghosted nodes in local domain.
    PetscInt :: ngmax  ! Number of ghosted & non-ghosted nodes in local domain.

    !nL2G :  not collective, local processor: local  =>  ghosted local
    !nG2L :  not collective, local processor:  ghosted local => local
    !nG2N :  collective,  ghosted local => global index , used for
    !                     matsetvaluesblocked ( not matsetvaluesblockedlocal)
    !nL2A :   collective, local => natural index, used for initialization
    !                              and source/sink setup
    PetscInt, pointer :: nL2G(:), nG2L(:), nL2A(:)
    PetscInt, pointer :: nG2A(:)

    PetscReal, pointer :: x(:), y(:), z(:)

    PetscReal :: x_min, x_max, y_min, y_max, z_min, z_max

    PetscInt, pointer :: hash(:,:,:)
    PetscInt :: num_hash_bins

    type(structured_grid_type), pointer :: structured_grid
    type(unstructured_grid_type), pointer :: unstructured_grid

    type(connection_list_type), pointer :: internal_connection_list

  end type grid_type
\end{verbatim}

%%%%%%%%%%%%%%%%%%%%%%%%%%%%%%%%%%%%%%%%%%%%%%%%%%%%%%%%%%%%%%%%%%%%%

\protect\hypertarget{target_obj_level}{}
\subsection{Level}

\begin{verbatim}
  type, public :: level_type

    PetscInt :: id
    type(patch_list_type), pointer :: patch_list
    type(level_type), pointer :: next

  end type level_type

  ! pointer data structure required for making an array of level pointers in F90
  type, public :: level_ptr_type
    type(level_type), pointer :: ptr           ! pointer to the level_type
  end type level_ptr_type

  type, public :: level_list_type
    PetscInt :: num_level_objects
    type(level_type), pointer :: first
    type(level_type), pointer :: last
    type(level_ptr_type), pointer :: array(:)
  end type level_list_type
\end{verbatim}

%%%%%%%%%%%%%%%%%%%%%%%%%%%%%%%%%%%%%%%%%%%%%%%%%%%%%%%%%%%%%%%%%%%%%

\protect\hypertarget{target_obj_logging}{}
\subsection{Logging}

\begin{verbatim}
  type, public :: logging_type

    PetscInt :: stage(10)

    PetscInt :: class_pflotran

    PetscEvent :: event_init
    PetscEvent :: event_setup

    PetscEvent :: event_restart
    PetscEvent :: event_checkpoint

    PetscEvent :: event_condition_read
    PetscEvent :: event_condition_read_values

    PetscEvent :: event_h5dread_f
    PetscEvent :: event_h5dwrite_f
    PetscEvent :: event_read_indices_hdf5
    PetscEvent :: event_map_indices_hdf5
    PetscEvent :: event_hash_create
    PetscEvent :: event_hash_map
    PetscEvent :: event_read_real_array_hdf5
    PetscEvent :: event_read_int_array_hdf5
    PetscEvent :: event_write_real_array_hdf5
    PetscEvent :: event_write_int_array_hdf5
    PetscEvent :: event_read_array_hdf5
    PetscEvent :: event_write_struct_dataset_hdf5
    PetscEvent :: event_region_read_hdf5
    PetscEvent :: event_region_read_ascii
    PetscEvent :: event_material_read_hdf5

    PetscEvent :: event_output_tecplot
    PetscEvent :: event_output_hdf5
    PetscEvent :: event_output_str_grid_tecplot
    PetscEvent :: event_output_write_tecplot
    PetscEvent :: event_output_write_flux_tecplot
    PetscEvent :: event_output_get_var_from_array
    PetscEvent :: event_output_get_cell_vel
    PetscEvent :: event_output_vec_tecplot
    PetscEvent :: event_output_breakthrough
    PetscEvent :: event_output_coordinates_hdf5

  end type logging_type
\end{verbatim}

%%%%%%%%%%%%%%%%%%%%%%%%%%%%%%%%%%%%%%%%%%%%%%%%%%%%%%%%%%%%%%%%%%%%%

\protect\hypertarget{target_obj_mat}{}
\subsection{Material}

\begin{verbatim}
  type, public :: material_type
    PetscInt :: id
    character(len=MAXWORDLENGTH) :: name
    PetscReal :: permeability(3,3)
    PetscReal :: permeability_pwr
    PetscReal :: porosity
    PetscReal :: tortuosity
    PetscInt :: ithrm
    PetscInt :: icap
    type(material_type), pointer :: next
  end type material_type

  type, public :: material_ptr_type
    type(material_type), pointer :: ptr
  end type material_ptr_type

  type, public :: thermal_property_type
    PetscInt :: id
    PetscReal :: rock_density
    PetscReal :: spec_heat
    PetscReal :: therm_cond_dry
    PetscReal :: therm_cond_wet
    PetscReal :: pore_compress
    PetscReal :: pore_expansivity
    PetscReal :: tort_bin_diff
    PetscReal :: vap_air_diff_coef
    PetscReal :: exp_binary_diff
    PetscReal :: enh_binary_diff_coef
    type(thermal_property_type), pointer :: next
  end type thermal_property_type

  type, public :: saturation_function_type
    PetscInt :: id
    character(len=MAXWORDLENGTH) :: saturation_function_ctype
    PetscInt :: saturation_function_itype
    character(len=MAXWORDLENGTH) :: permeability_function_ctype
    PetscInt :: permeability_function_itype
    PetscReal, pointer :: Sr(:)
    PetscReal :: m
    PetscReal :: lambda
    PetscReal :: alpha
    PetscReal :: pcwmax
    PetscReal :: betac
    PetscReal :: power
    PetscInt :: ihist
    PetscReal :: BC_pressure_low
    PetscReal :: BC_pressure_high
    PetscReal :: BC_spline_coefficients(4)
    type(saturation_function_type), pointer :: next
  end type saturation_function_type

  type, public :: saturation_function_ptr_type
    type(saturation_function_type), pointer :: ptr
  end type saturation_function_ptr_type
\end{verbatim}

%%%%%%%%%%%%%%%%%%%%%%%%%%%%%%%%%%%%%%%%%%%%%%%%%%%%%%%%%%%%%%%%%%%%%

\protect\hypertarget{target_obj_option}{}
\subsection{Option}

\begin{verbatim}
  type, public :: option_type

    PetscMPIInt :: myrank                    ! rank in PETSC_COMM_WORLD
    PetscMPIInt :: commsize                  ! size of PETSC_COMM_WORLD

    ! defines the mode (e.g. mph, richards, vadose, etc.
    character(len=MAXWORDLENGTH) :: flowmode
    PetscInt :: iflowmode
    character(len=MAXWORDLENGTH) :: tranmode
    PetscInt :: itranmode

    PetscInt :: nphase
    PetscInt :: nflowdof
    PetscInt :: nspec

    PetscInt :: ntrandof
    PetscInt :: ncomp

    PetscReal :: uniform_velocity(3)

    ! Program options
    PetscTruth :: use_matrix_free  ! If true, do not form the Jacobian.

    PetscInt :: imod

    PetscTruth :: use_isoth

    character(len=MAXWORDLENGTH) :: generalized_grid
    logical :: use_generalized_grid

    PetscReal :: flow_time, tran_time, time  ! The time elapsed in the simulation.
    PetscReal :: flow_dt, tran_dt, dt ! The size of the time step.

!    PetscReal, pointer :: tplot(:)
    PetscReal, pointer :: tfac(:)
      ! An array of multiplicative factors that specify how to increase time step.

    PetscInt :: iblkfmt ! blocked format

      ! Basically our target number of newton iterations per time step.
    PetscReal :: dpmxe,dtmpmxe,dsmxe,dcmxe !maximum allowed changes in field vars.
    PetscReal :: dpmax,dtmpmax,dsmax,dcmax

    PetscReal :: scale
    PetscReal, pointer :: rock_density(:),cpr(:),dencpr(:),ckdry(:),ckwet(:), &
                          tau(:),cdiff(:),cexp(:)
    PetscReal, pointer :: swir(:),lambda(:),alpha(:),pckrm(:),pcwmax(:), &
                          pcbetac(:),pwrprm(:),sir(:,:)
    PetscInt, pointer:: icaptype(:)

    PetscReal :: m_nacl
    PetscReal :: difaq, delhaq, gravity(3), fmwh2o= 18.0153D0, fmwa=28.96D0, &
              fmwco2=44.0098D0, eqkair, ret=1.d0, fc=1.d0

    PetscInt :: ideriv
    PetscReal :: tref,pref

    PetscReal :: disp

!   table lookup
    PetscInt :: itable=0

    PetscTruth :: restart_flag
    PetscReal :: restart_time
    character(len=MAXWORDLENGTH) :: restart_file
    PetscTruth :: checkpoint_flag
    PetscInt :: checkpoint_frequency

    PetscLogDouble :: start_time
    PetscTruth :: wallclock_stop_flag
    PetscLogDouble :: wallclock_stop_time

    PetscInt :: log_stage(10)

    logical :: numerical_derivatives
    logical :: compute_statistics
    logical :: use_touch_options
    logical :: overwrite_restart_transport
    PetscInt :: io_handshake_buffer_size

    character(len=MAXWORDLENGTH) :: permx_filename
    character(len=MAXWORDLENGTH) :: permy_filename
    character(len=MAXWORDLENGTH) :: permz_filename

  end type option_type

  type, public :: output_option_type

    character(len=2) :: tunit
    PetscReal :: tconv

    logical :: print_hdf5
    logical :: print_hdf5_velocities
    logical :: print_hdf5_flux_velocities

    logical :: print_tecplot
    logical :: print_tecplot_velocities
    logical :: print_tecplot_flux_velocities

    PetscInt :: plot_number
    character(len=MAXWORDLENGTH) :: plot_name

  end type output_option_type
\end{verbatim}

%%%%%%%%%%%%%%%%%%%%%%%%%%%%%%%%%%%%%%%%%%%%%%%%%%%%%%%%%%%%%%%%%%%%%

\protect\hypertarget{target_obj_patch}{}
\subsection{Patch}

\begin{verbatim}
  type, public :: patch_type

    PetscInt :: id

    ! thiese arrays will be used by all modes, mode-specific arrays should
    ! go in the auxilliary data stucture for that mode
    PetscInt, pointer :: imat(:)
    PetscReal, pointer :: internal_velocities(:,:)
    PetscReal, pointer :: boundary_velocities(:,:)

    type(grid_type), pointer :: grid

    type(region_list_type), pointer :: regions

    type(coupler_list_type), pointer :: transport_boundary_conditions
    type(coupler_list_type), pointer :: transport_initial_conditions
    type(coupler_list_type), pointer :: transport_source_sinks

    type(coupler_list_type), pointer :: flow_boundary_conditions
    type(coupler_list_type), pointer :: flow_initial_conditions
    type(coupler_list_type), pointer :: flow_source_sinks

    type(strata_list_type), pointer :: strata
    type(breakthrough_list_type), pointer :: breakthrough

    type(auxilliary_type) :: aux

    type(patch_type), pointer :: next

  end type patch_type

  ! pointer data structure required for making an array of patch pointers in F90
  type, public :: patch_ptr_type
    type(patch_type), pointer :: ptr           ! pointer to the patch_type
  end type patch_ptr_type

  type, public :: patch_list_type
    PetscInt :: num_patch_objects
    type(patch_type), pointer :: first
    type(patch_type), pointer :: last
    type(patch_ptr_type), pointer :: array(:)
  end type patch_list_type
\end{verbatim}

%%%%%%%%%%%%%%%%%%%%%%%%%%%%%%%%%%%%%%%%%%%%%%%%%%%%%%%%%%%%%%%%%%%%%

\protect\hypertarget{target_obj_real}{}
\subsection{Realization}

\begin{verbatim}
  type, public :: realization_type

    type(discretization_type), pointer :: discretization
    type(level_list_type), pointer :: level_list
    type(patch_type), pointer :: patch

    type(option_type), pointer :: option
    type(field_type), pointer :: field
    type(pflow_debug_type), pointer :: debug
    type(output_option_type), pointer :: output_option

    type(region_list_type), pointer :: regions
    type(condition_list_type), pointer :: flow_conditions
    type(condition_list_type), pointer :: transport_conditions

    type(material_type), pointer :: materials
    type(material_ptr_type), pointer :: material_array(:)
    type(thermal_property_type), pointer :: thermal_properties
    type(saturation_function_type), pointer :: saturation_functions
    type(saturation_function_ptr_type), pointer :: saturation_function_array(:)

    type(waypoint_list_type), pointer :: waypoints

  end type realization_type
\end{verbatim}

%%%%%%%%%%%%%%%%%%%%%%%%%%%%%%%%%%%%%%%%%%%%%%%%%%%%%%%%%%%%%%%%%%%%%

\protect\hypertarget{target_obj_region}{}
\subsection{Region}

\begin{verbatim}
  type, public :: block_type
    PetscInt :: i1,i2,j1,j2,k1,k2
    type(block_type), pointer :: next
  end type block_type

  type, public :: region_type
    PetscInt :: id
    character(len=MAXWORDLENGTH) :: name
    character(len=MAXWORDLENGTH) :: filename
    PetscInt :: i1,i2,j1,j2,k1,k2
    PetscReal :: coordinate(3)
    PetscInt :: iface
    PetscInt :: num_cells
    PetscInt, pointer :: cell_ids(:)
    PetscInt, pointer :: faces(:)
    type(region_type), pointer :: next
  end type region_type

  type, public :: region_ptr_type
    type(region_type), pointer :: ptr
  end type region_ptr_type

  type, public :: region_list_type
    PetscInt :: num_regions
    type(region_type), pointer :: first
    type(region_type), pointer :: last
    type(region_type), pointer :: array(:)
  end type region_list_type
\end{verbatim}

%%%%%%%%%%%%%%%%%%%%%%%%%%%%%%%%%%%%%%%%%%%%%%%%%%%%%%%%%%%%%%%%%%%%%

\protect\hypertarget{target_obj_richards}{}
\subsection{Richards}

\begin{verbatim}
  type, public :: richards_auxvar_type
    PetscReal :: pres
    PetscReal :: temp
    PetscReal :: sat
    PetscReal :: den
    PetscReal :: den_kg
    PetscReal :: avgmw
    PetscReal :: h
    PetscReal :: u
    PetscReal :: pc
!    PetscReal :: vis
!    PetscReal :: dvis_dp
!    PetscReal :: kr
!    PetscReal :: dkr_dp
    PetscReal :: kvr
    PetscReal :: dsat_dp
    PetscReal :: dden_dp
    PetscReal :: dden_dt
    PetscReal :: dkvr_dp
    PetscReal :: dkvr_dt
    PetscReal :: dh_dp
    PetscReal :: dh_dt
    PetscReal :: du_dp
    PetscReal :: du_dt
    PetscReal, pointer :: xmol(:)
    PetscReal, pointer :: diff(:)
  end type richards_auxvar_type

  type, public :: richards_type
    PetscInt :: n_zero_rows
    PetscInt, pointer :: zero_rows_local(:), zero_rows_local_ghosted(:)

    logical :: aux_vars_up_to_date
    logical :: inactive_cells_exist
    PetscInt :: num_aux, num_aux_bc
    type(richards_auxvar_type), pointer :: aux_vars(:)
    type(richards_auxvar_type), pointer :: aux_vars_bc(:)
  end type richards_type
\end{verbatim}

%%%%%%%%%%%%%%%%%%%%%%%%%%%%%%%%%%%%%%%%%%%%%%%%%%%%%%%%%%%%%%%%%%%%%

\protect\hypertarget{target_obj_richards_lite}{}
\subsection{Richards\_lite}

\begin{verbatim}
  type, public :: richards_lite_auxvar_type
    PetscReal :: pres
    PetscReal :: temp
    PetscReal :: sat
    PetscReal :: den
    PetscReal :: den_kg
    PetscReal :: avgmw
    PetscReal :: pc
!    PetscReal :: vis
!    PetscReal :: dvis_dp
!    PetscReal :: kr
!    PetscReal :: dkr_dp
    PetscReal :: kvr
    PetscReal :: dsat_dp
    PetscReal :: dden_dp
    PetscReal :: dkvr_dp
  end type richards_lite_auxvar_type

  type, public :: richards_lite_type
    PetscInt :: n_zero_rows
    PetscInt, pointer :: zero_rows_local(:), zero_rows_local_ghosted(:)

    logical :: aux_vars_up_to_date
    logical :: inactive_cells_exist
    PetscInt :: num_aux, num_aux_bc
    type(richards_lite_auxvar_type), pointer :: aux_vars(:)
    type(richards_lite_auxvar_type), pointer :: aux_vars_bc(:)
  end type richards_lite_type
\end{verbatim}

%%%%%%%%%%%%%%%%%%%%%%%%%%%%%%%%%%%%%%%%%%%%%%%%%%%%%%%%%%%%%%%%%%%%%

\protect\hypertarget{target_obj_simulation}{}
\subsection{Simulation}

\begin{verbatim}
  type, public :: simulation_type

    type(realization_type), pointer :: realization
    type(stepper_type), pointer :: flow_stepper
    type(stepper_type), pointer :: tran_stepper

  end type simulation_type
\end{verbatim}

%%%%%%%%%%%%%%%%%%%%%%%%%%%%%%%%%%%%%%%%%%%%%%%%%%%%%%%%%%%%%%%%%%%%%

\protect\hypertarget{target_obj_solver}{}
\subsection{Solver}

\begin{verbatim}
  type, public :: solver_type
    PetscReal :: linear_atol       ! absolute tolerance
    PetscReal :: linear_rtol       ! relative tolerance
    PetscReal :: linear_dtol       ! divergence tolerance
    PetscInt :: linear_maxit     ! maximum number of iterations

    PetscReal :: newton_atol       ! absolute tolerance
    PetscReal :: newton_rtol       ! relative tolerance
    PetscReal :: newton_stol       ! relative tolerance (relative to previous
                                     iteration)
    PetscReal :: newton_dtol       ! divergence tolerance
    PetscReal :: newton_inf_res_tol    ! infinity tolerance for residual
    PetscReal :: newton_inf_upd_tol    ! infinity tolerance for update
    PetscInt :: newton_maxit     ! maximum number of iterations
    PetscInt :: newton_maxf      ! maximum number of function evaluations

        ! Jacobian matrix
    Mat :: J
    MatFDColoring :: matfdcoloring
      ! Coloring used for computing the Jacobian via finite differences.

    ! PETSc nonlinear solver context
    SNES :: snes
    KSPType :: ksp_type
    PCType  :: pc_type
    KSP   ::  ksp
    PC    ::  pc

    PetscTruth :: inexact_newton

    PetscTruth :: print_convergence
    PetscTruth :: print_detailed_convergence
    PetscTruth :: check_infinity_norm
    PetscTruth :: force_at_least_1_iteration

  end type solver_type
\end{verbatim}

%%%%%%%%%%%%%%%%%%%%%%%%%%%%%%%%%%%%%%%%%%%%%%%%%%%%%%%%%%%%%%%%%%%%%

\protect\hypertarget{target_obj_stepper}{}
\subsection{Stepper}

\begin{verbatim}
  type, public :: stepper_type

    PetscInt :: steps     ! The number of time-steps taken by the code.
    PetscInt :: nstepmax  ! Maximum number of timesteps taken by the code.
    PetscInt :: icut_max  ! Maximum number of timestep cuts within one time step.
    PetscInt :: ndtcmx    ! Steps needed after cutting to increase time step
    PetscInt :: newtcum   ! Total number of Newton steps taken.
    PetscInt :: icutcum   ! Total number of cuts in the timestep taken.
    PetscInt :: iaccel    ! Accelerator index

    PetscReal :: dt_min
    PetscReal :: dt_max

    type(solver_type), pointer :: solver

    type(waypoint_type), pointer :: cur_waypoint

    type(convergence_context_type), pointer :: convergence_context

  end type stepper_type
\end{verbatim}

%%%%%%%%%%%%%%%%%%%%%%%%%%%%%%%%%%%%%%%%%%%%%%%%%%%%%%%%%%%%%%%%%%%%%

\protect\hypertarget{target_obj_strata}{}
\subsection{Strata}

\begin{verbatim}
  type, public :: strata_type
    PetscInt :: id                                 ! id of strata
    logical :: active
    character(len=MAXWORDLENGTH) :: material_name  ! character string defining
                                                   name of material to be applied
    character(len=MAXWORDLENGTH) :: region_name    ! character string defining
                                                   name of region to be applied
    PetscInt :: imaterial                ! id of material in material array/list
    PetscInt :: iregion                      ! id of region in region array/list
    type(material_type), pointer :: material       ! pointer to material in
                                                     material array/list
    type(region_type), pointer :: region           ! pointer to region in region
                                                     array/list
    type(strata_type), pointer :: next             ! pointer to next strata
  end type strata_type

  type, public :: strata_ptr_type
    type(strata_type), pointer :: ptr
  end type strata_ptr_type

  type, public :: strata_list_type
    PetscInt :: num_strata
    type(strata_type), pointer :: first
    type(strata_type), pointer :: last
    type(strata_ptr_type), pointer :: array(:)
  end type strata_list_type
\end{verbatim}

%%%%%%%%%%%%%%%%%%%%%%%%%%%%%%%%%%%%%%%%%%%%%%%%%%%%%%%%%%%%%%%%%%%%%

\protect\hypertarget{target_obj_structured_grid}{}
\subsection{Structured\_grid}

\begin{verbatim}
  type, public :: structured_grid_type

    PetscInt :: nx, ny, nz    ! Global domain dimensions of the grid.
    PetscInt :: nxy, nmax     ! nx * ny, nx * ny * nz
    PetscInt :: npx, npy, npz ! Processor partition in each direction.
    PetscInt :: nlx, nly, nlz ! Local grid dimension w/o ghost nodes.
    PetscInt :: ngx, ngy, ngz ! Local grid dimension with ghost nodes.
    PetscInt :: nxs, nys, nzs
      ! Global indices of non-ghosted corner (starting) of local domain.
    PetscInt :: ngxs, ngys, ngzs
      ! Global indices of ghosted starting corner of local domain.
    PetscInt :: nxe, nye, nze, ngxe, ngye, ngze
      ! Global indices of non-ghosted/ghosted ending corner of local domain.
    PetscInt :: nlxy, nlxz, nlyz
    PetscInt :: ngxy, ngxz, ngyz

    PetscInt :: istart, jstart, kstart, iend, jend, kend
      ! istart gives the local x-index of the non-ghosted starting (lower left)
      ! corner. iend gives the local x-index of the non-ghosted ending
      ! corner. jstart, jend correspond to y-index, kstart, kend to z-index.

    PetscInt :: nlmax  ! Total number of non-ghosted nodes in local domain.
    PetscInt :: ngmax  ! Number of ghosted & non-ghosted nodes in local domain.

    PetscReal :: origin(3)

    PetscReal, pointer :: dx0(:), dy0(:), dz0(:)

    logical :: invert_z_axis

    PetscReal, pointer :: dx(:),dy(:),dz(:),dxg(:),dyg(:),dzg(:) ! Grid spacings

    PetscFortranAddr p_samr_patch ! pointer to a SAMRAI patch object

  end type structured_grid_type
\end{verbatim}

%%%%%%%%%%%%%%%%%%%%%%%%%%%%%%%%%%%%%%%%%%%%%%%%%%%%%%%%%%%%%%%%%%%%%

\protect\hypertarget{target_obj_waypoint}{}
\subsection{Waypoint}

\begin{verbatim}
  type, public :: waypoint_type
    PetscReal :: time
    logical :: print_output
    type(output_option_type), pointer :: output_option
    logical :: update_bcs
    logical :: update_srcs
    PetscReal :: dt_max
    logical :: final  ! any waypoint after this will be deleted
    type(waypoint_type), pointer :: prev
    type(waypoint_type), pointer :: next
  end type waypoint_type

  type, public :: waypoint_list_type
    PetscInt :: num_waypoints
    type(waypoint_type), pointer :: first
    type(waypoint_type), pointer :: last
    type(waypoint_type), pointer :: array(:)
  end type waypoint_list_type
\end{verbatim}

%%%%%%%%%%%%%%%%%%%%%%%%%%%%%%%%%%%%%%%%%%%%%%%%%%%%%%%%%%%%%%%%%%%%%

\newpage
\section{FAQ}

\subsection{\sl iobuf load errors}

It may be the case that the `iobuf'module is causing problems.  That is a module that, if it's loaded, links
with an IO buffering library. It can speed up IO considerably, but there
are have been some bugs (hopefully fixed) identified with it before.  It
is loaded by default.  You might want to try a 'module unload' of that
before building PFLOTRAN, and seeing if that works. Unfortunately, it
may be necessary to mess with the configuration files for the PETSc builds to
make sure that you don't link with the iobuf library.

\end{comment}
%%%%%%%%%%%%%%%%%%%%%%%%%%%%%%%%%%%%%%%%%%%%%%%%%%%%%%%%%%%%%%%%%%%%%

\section{References}

\begin{description}

\item Balay S, Eijkhout V, Gropp WD, McInnes LC and Smith BF (1997) Modern Software Tools in Scientific Computing, Eds. Arge E, Bruaset AM and Langtangen HP (Birkha\"user Press), pp. 163--202.

\end{description}

\end{document}

