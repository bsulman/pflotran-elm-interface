%\section{Appendix: Reaction Sandbox}
\section*{Appendix C: Reaction Sandbox}
\addcontentsline{toc}{section}{\protect\numberline{}{\bf Appendix C: Reaction Sandbox}} 


\setcounter{section}{3}
\setcounter{subsection}{0}
\setcounter{equation}{0}
\setcounter{table}{0}
\setcounter{figure}{0}

\subsection{Background}

Researchers often have a suite of reactions tailored to a unique problem scenario, but these reaction networks only exist in their respective research codes. The ``reaction sandbox'' provides these researchers with a venue for implementing user-defined reactions within PFLOTRAN. Reaction networks developed within the reaction sandbox can leverage existing biogeochemical capability within PFLOTRAN (e.g. equilibrium aqueous complexation, mineral precipitation--dissolution, etc.) or function independently. Please note that although the reaction sandbox facilitates the integration of user-defined reactions, the process still requires a basic understanding of PFLOTRAN and its approach to solving reaction through the Newton-Raphson method. For instance, one must understand the purpose and function of the rt\_auxvar and global\_auxvar objects.

\subsection{Implementation}
The core framework of reaction sandbox leverages Fortran 2003 object--oriented extendable derived types and methods and consists of two modules:

\begin{itemize}
  \item[] Reaction\_Sandbox\_module (reaction\_sandbox.F90)
  \item[] Reaction\_Sandbox\_Base\_class (reaction\_sandbox\_base.F90).
\end{itemize}

To implement a new reaction within the reaction sandbox, one creates a new class by extending the Reaction\_Sandbox\_Base\_class and adds the new class to the Reaction\_Sandbox\_module.  The following steps illustrate this process through the creation of the class Reaction\_Sandbox\_Example\_class that implements a first order decay reaction.

\begin{enumerate}
  \item Copy reaction\_sandbox\_template.F90 to a new filename (e.g. reaction\_sandbox\_example.F90).
  \item Replace all references to Template/template with the new reaction name.
    \begin{itemize}
      \item[] \verb+Template+ $\rightarrow$ \verb+Example+
      \item[] \verb+template+ $\rightarrow$ \verb+example+
    \end{itemize}
  \item Add necessary variables to the module and/or the extended derived type.
\begin{verbatim}
character(len=MAXWORDLENGTH) :: species_name
PetscInt :: species_id
PetscReal :: rate_constant
\end{verbatim}
  \item Add the necessary functionality within the following subroutines:
    \begin{enumerate}[label=\alph*]
      \item ExampleCreate: Allocate the reaction object, initializing all variables to zero and nullifying arrays.  {\bf Be sure to nullify ExampleCreate\%next which comes from the base class.}  E.g.,
  \begin{verbatim}
  allocate(ExampleCreate)
  ExampleCreate%species_name = ''
  ExampleCreate%species_id = 0
  ExampleCreate%rate_constant = 0.d0
  nullify(ExampleCreate%next)
  \end{verbatim}
      \item ExampleRead: Read parameters in from the input file block \verb+EXAMPLE+.  E.g.,
  \begin{verbatim}
  ...
  case('SPECIES_NAME')
    call InputReadWord(input,option,this%species_name, &
                       PETSC_TRUE)
    call InputErrorMsg(input,option,'species_name', &
                   'CHEMISTRY,REACTION_SANDBOX,EXAMPLE')
  ...
  \end{verbatim}
      \item ExampleSetup: Construct the reaction network (e.g. array allocation, establishing linkages, etc.). E.g.,
  \begin{verbatim}
  ...
  this%species_id = &
    GetPrimarySpeciesIDFromName(this%species_name, &
                                reaction,option)
  ...
  \end{verbatim}
      \item ExampleReact: Calculate contribution of reaction to the residual (units = moles/sec) and Jacobian (units = kg water/sec). E.g.,
  \begin{verbatim}
  ...
  Residual(this%species_id) = &
    Residual(this%species_id) - &
    this%rate_constant*porosity* &
    global_auxvar%sat(iphase)*volume*1.d3* &
    rt_auxvar%total(this%species_id,iphase)
  ...
  Jacobian(this%species_id,this%species_id) = &
  Jacobian(this%species_id,this%species_id) + &
    this%rate_constant*porosity* &
    global_auxvar%sat(iphase)*volume*1.d3*
    rt_auxvar%aqueous%dtotal(this%species_id, &
                             this%species_id,iphase)
  ...
  \end{verbatim}
      \item ExampleDestroy: Deallocate any dynamic memory within the class (without deallocating the object itself).
    \end{enumerate}
  \item Ensure that the methods within the extended derived type point to the proper procedures in the module
  \begin{verbatim}
  procedure, public :: ReadInput => ExampleRead
  procedure, public :: Setup => ExampleSetup
  procedure, public :: Evaluate => ExampleReact
  procedure, public :: Destroy => ExampleDestroy
  \end{verbatim}
  \item Within reaction\_sandbox.F90:
    \begin{enumerate}[label=\alph*]
      \item Add Reaction\_Sandbox\_Example\_class+ to the list of modules to be ``used'' at the top of the file.
      \item Add a case statement in RSandboxRead2 for the keyword defining the new reaction and create the reaction within.  I.e.
  \begin{verbatim}
    case('EXAMPLE')
      new_sandbox => ExampleCreate()
  \end{verbatim}
    \end{enumerate}
\end{enumerate}

