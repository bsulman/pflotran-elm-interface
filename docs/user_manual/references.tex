\section{References}

\begin{description}

\item Balay S., V. Eijkhout V, W.D. Gropp, L.C. McInnes and B.F. Smith (1997) Modern Software Tools in Scientific Computing, Eds. Arge E, Bruaset AM and Langtangen HP (Birkha\"user Press), pp. 163--202.

\item Coats, K.H. and A.B. Ramesh (1982) Effects of Grid Type and Difference Scheme on Pattern Steamflood Simulation Results, paper SPE-11079, presented at the 57th Annual Fall Technical Conference and Exhibition of the Society of Petroleum Engineers, New Orleans, LA, September 1982.

\item Ebigbo, A., Holger Class, H., Helmig, R. (2007) CO$_2$ leakage through an abandoned well: problem-oriented benchmarks, Comput Geosciences 11:103?115 DOI 10.1007/s10596-006-9033-7.

\item Fenghour, A., W.A. Wakeham, and V. Vesovic (1998) The viscosity of carbon dioxide, J. Phys. Chem. Ref. Data, 27(1), 31--44.

\item Goode, D.J. (1996) Direct simulation of groundwater age, Water Resources Research, 32, 289--296.

\item Hammond, G.E., P.C. Lichtner, C. Lu, and R.T. Mills (2011) PFLOTRAN: Reactive Flow \& Transport Code for Use on Laptops to Leadership-Class Supercomputers, Editors: Zhang, F., G. T. Yeh, and J. C. Parker, {\em Ground Water Reactive Transport Models}, Bentham Science Publishers. ISBN 978-1-60805-029-1. 

\item Khaleel, R., E.J. Freeman (1995) Variability and scaling of hydraulic properties for 200 area soils, Hanford Site. Report WHC-EP-0883. Westinghouse Hanford Company, Richland, WA.

\item Khaleel, R., T.E. Jones, A.J. Knepp, F.M. Mann, D.A. Myers, P.M. Rogers, R.J. Serne, and M.I. Wood (2000) Modeling data package for S-SX Field Investigation Report (FIR). Report RPP-6296, Rev. 0. CH2M Hill Hanford Group, Richland, WA.

\item Lichtner, P.C., Yabusaki, S.B., Pruess K., and Steefel, C.I. (2004) Role of Competitive Cation Exchange on Chromatographic Displacement of Cesium in the Vadose Zone Beneath the Hanford S/SX Tank Farm, {\em VJZ}, {\bf 3}, 203--219.

\item Lichtner, P.C. (1996a) Continuum Formulation of Multicomponent-Multiphase Reactive Transport, In: {\em Reactive Transport in Porous Media} (eds. P. C. Lichtner, C. I. Steefel, and E. H. Oelkers), {\em Reviews in Mineralogy}, {\bf 34}, 1--81.

\item Lichtner P.C. (1996b) Modeling Reactive Flow and Transport in Natural Systems, Proceedings of the Rome Seminar on Environmental Geochemistry, Eds. G. Ottonello and L. Marini, Castelnuovo di Porto, May 22--26, Pacini Editore, Pisa, Italy, 5--72.

\item Lichtner P.C. (2000) Critique of Dual Continuum Formulations of Multicomponent Reactive Transport in Fractured Porous Media, Ed. Boris Faybishenko, {\em Dynamics of Fluids in Fractured Rock}, Geophysical Monograph {\bf 122}, 281--298.

\item Painter, S.L. (2011) Three-phase numerical model of water migration in partially frozen geological media: model formulation, validation, and applications, Computational Geosciences {\bf 15}, 69-85. 

\item Peaceman, D.W. (1977) Interpretation of Well-Block Pressures in Numerical Reservoir Simulation with Nonsquare Grid Blocks and Anisotropic Permeability, paper SPE-10528, presented at the Sixth SPE Symposium on Reservoir Simulation of the Society of Petroleum Engineers, New Orleans, LA, January 1982.  

\item Pruess, K., S. Yabusaki, C. Steefel, and P. Lichtner (2002) Fluid flow,
heat transfer, and solute transport at nuclear waste storage tanks in the Hanford vadose zone. Available at www.vadosezonejournal.org. Vadose Zone J. 1:68--88.

\item Pruess, K., and Narasimhan (1985) A practical method for modeling fluid and heat flow in fractured porous media, SPE 10509, 14--26.

\item Somerton, W.H., A.H. El-Shaarani, and S.M. Mobarak (1974) 
High temperature behavior of rocks associated with geothermal-type reservoirs. Paper SPE-4897. Proceedings of the 44th Annual 
California Regional Meeting of the Society of Petroleum Engineers. Richardson, TX: Society of Petroleum Engineers. 

\item Andreas Voegelin, Vijay M. Vulava, Florian Kuhnen, Ruben Kretzschmar (2000) Multicomponent transport of major cations predicted from binary adsorption experiments, Journal of Contaminant Hydrology, 46, 319--338.
\end{description}
