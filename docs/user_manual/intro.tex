
\section{Introduction}

PFLOTRAN solves a system of generally nonlinear partial differential equations describing multiphase, multicomponent and multiscale reactive flow and transport in porous materials. The code is designed to run on massively parallel computing architectures as well as workstations and laptops (e.g. Hammond et al., 2011). Parallelization is achieved through domain decomposition using the PETSc (Port\-a\-ble Extensible Toolkit for Scientific Computation) libraries for the parallelization framework (Balay et al., 1997). 

PFLOTRAN has been developed from the ground up for parallel scalability and has been run on up to $2^{18}$ processor cores with problem sizes up to 2 billion degrees of freedom. Written in object oriented Fortran 90, the code requires the latest compilers compatible with Fortran 2003. At the time of this writing this requires, for example, gcc 4.7.x or Intel 13.x.x or later. %and PGC compilers. 
As a requirement of running problems with a large number of degrees of freedom, PFLOTRAN allows reading input data that would be too large to fit into memory if allocated to a single processor core. The current limitation to the problem size PFLOTRAN can handle is the limitation of the HDF5 file format used for parallel IO to 32 bit integers. Noting that $2^{32} = 4,294,967,296$, this gives an estimate of several billion degrees of freedom for the maximum problem size that can be currently run with PFLOTRAN. Hopefully this limitation will be remedied in the near future.

Currently PFLOTRAN can handle a number of subsurface processes involving flow and transport in porous media including Richards equation, two-phase flow involving supercritical CO$_2$, and multicomponent reactive transport including aqueous complexing, sorption and mineral precipitation and dissolution. Reactive transport equations are solved using a fully implicit Newton-Raphson algorithm.
Operator splitting is currently not implemented. In addition to single continuum processes, a novel approach is used to solve equations resulting from a multiple interacting continuum method for modeling flow and transport in fractured media. This implementation is still under development. Finally, an elastic geomechanical model is implemented.

A novel feature of the code is its ability to run multiple input files and multiple realizations simultaneously, for example with different permeability and porosity fields, on one or more processor cores per run. This can be extremely useful when conducting sensitivity analyses and quantifying model uncertainties. When running on machines with many cores this means that hundreds of simulations can be conducted in the amount of time needed for a single realization.

Additional information can be found on the PFLOTRAN \href{https://bitbucket.org/pflotran/pflotran-dev/wiki/Home}{wiki home page} which should be consulted for the most up-to-date information.
Questions regarding installing PFLOTRAN on workstations, small clusters, and super computers, and bug reports, may be directed to: \linebreak {\footnotesize\tt pflotran-dev at googlegroups dot com}.
For questions regarding running PFLOTRAN contact: {\footnotesize\tt pflotran-users at googlegroups dot com}.
