\documentclass[12pt]{article} 
%\usepackage{times,helvet}
\usepackage{palatino}
%\usepackage[utf8]{inputenc} %useful to type directly diacritic characters
%\usepackage{ssss}
\usepackage{amsmath,amsbsy,amssymb}
\usepackage{sectsty,hangcaption}
\usepackage{deflist}
\usepackage{fancyhdr}
\usepackage{tabularx}
\usepackage{verbatim}
\usepackage{moreverb}
\usepackage{float,comment}
\usepackage{graphicx}
\usepackage{longtable}
%\usepackage{portland}
\usepackage{booktabs}

%must be last package
%\usepackage{hyperref}
\usepackage[debug=false, colorlinks=true, pdfstartview=FitV, linkcolor=blue, citecolor=blue, urlcolor=blue, pdfpagelabels=true]{hyperref}

\textwidth 6.5in
\textheight 9.5in
%\topmargin -.1in
\topmargin -.75in
\newlength{\boxwidth}
\setlength{\boxwidth}{5.8in}
\oddsidemargin -0in
\evensidemargin -0in
\headheight 0.25in

\lhead{{\sl PFLOTRAN: Governing Equations}}
\chead{\rm - \thepage\ -}
\rhead{\sl DRAFT: \today}
\cfoot{}

\newcommand\flotran{{\sl FloTran}}
\renewcommand{\baselinestretch}{1.0}
\def\EQ#1\EN{\begin{equation}#1\end{equation}}
\def\BA#1\EA{\begin{align}#1\end{align}}
\def\BS#1\ES{\begin{split}#1\end{split}}
%\newcommand{\EQ}{\begin{equation}}
%\newcommand{\EN}{\end{equation}}
\newcommand{\bcr}{\begin{center}}
\newcommand{\ecr}{\end{center}}
\newcommand{\eq}{\ =\ }
\newcommand{\degc}{$^\circ$C}
\renewcommand{\c}{{\rm CO_2}}
\newcommand{\im}{{\rm imb}}
\newcommand{\m}{{\rm mb}}
\newcommand{\ecm}{{\rm ecm}}
\newcommand{\eff}{{\rm eff}}
\newcommand{\eqr}{{\rm le}}
\newcommand{\equ}{{\rm eq}}
\newcommand{\kin}{{\rm kin}}
\newcommand{\rdx}{{\rm rdx}}
\newcommand{\ind}{{\rm id}}
\newcommand{\dep}{{\rm dp}}
\newcommand{\e}{{\rm{e}}}
\newcommand{\erf}{{\rm{erf}}}
\newcommand{\erfc}{{\rm{erfc}}}
\newcommand{\p}{{\partial}}
\newcommand{\A}{{\mathcal A}}
\newcommand{\B}{{\mathcal B}}
\newcommand{\C}{{\mathcal C}}
\newcommand{\D}{{\mathcal D}}
\newcommand{\E}{{\mathcal E}}
\newcommand{\F}{{\mathcal F}}
\newcommand{\G}{{\mathcal G}}
\newcommand{\I}{{\mathcal I}}
\newcommand{\J}{{\mathcal J}}
\newcommand{\M}{{\mathcal M}}
\newcommand{\cO}{{\mathcal O}}
\renewcommand{\P}{{{\mathcal P}}}
\newcommand{\Q}{{\mathcal Q}}
\newcommand{\R}{{{\mathcal R}}}
\renewcommand{\S}{{\mathcal S}}
\newcommand{\T}{{\mathcal T}}
\newcommand{\W}{{\mathcal W}}
\newcommand{\Y}{{\mathcal Y}}
\newcommand{\Z}{{\mathcal Z}}
\newcommand{\rev}{{\rm rev}}
\newcommand{\irr}{{\rm irr}}
\renewcommand{\a}{{\alpha}}
\newcommand{\abar}{{\bar \alpha}}
\renewcommand{\b}{{\beta}}
\renewcommand{\e}{{\epsilon}}
\newcommand{\s}{{\sigma}}
\newcommand{\w}{{\rm H_2O}}
\newcommand{\air}{{\rm N_2}}
\newcommand{\pe}{{\rm Pe}}
\newcommand{\da}{{\rm Da}}
\renewcommand{\k}{{\dot R}^0}
\renewcommand{\L}{\widehat{\mathcal L}}
%\renewcommand{\bar}{\overline}
\newcommand{\dsty}{{\displaystyle}}
\newcommand{\diff}{{\mathcal D}}
\newcommand{\surf}{\equiv \!\!\!}
\newcommand{\bnabla}{\boldsymbol{\nabla}}
\newcommand{\ba}{\boldsymbol{a}}

\newcommand{\balpha}{\boldsymbol{\alpha}}
\newcommand{\bbeta}{\boldsymbol{\beta}}
\newcommand{\bgamma}{\boldsymbol{\gamma}}
\renewcommand{\d}{{\delta}}

\newcommand{\bA}{\boldsymbol{A}}
\newcommand{\bB}{\boldsymbol{B}}
\newcommand{\bb}{\boldsymbol{b}}
\newcommand{\bC}{\boldsymbol{C}}
\newcommand{\bc}{\boldsymbol{c}}
\newcommand{\bcolon}{\boldsymbol{:}}
\newcommand{\bdot}{\boldsymbol{\cdot}}
\newcommand{\bD}{\boldsymbol{D}}
\newcommand{\bE}{\boldsymbol{E}}
\newcommand{\bF}{\boldsymbol{F}}
\newcommand{\bG}{\boldsymbol{G}}
\newcommand{\bg}{\boldsymbol{g}}
\newcommand{\bi}{\boldsymbol{i}}
\newcommand{\bI}{\boldsymbol{I}}
\newcommand{\bJ}{\boldsymbol{J}}
\newcommand{\bK}{\boldsymbol{K}}
\newcommand{\bL}{\boldsymbol{L}}
\newcommand{\bM}{\boldsymbol{M}}
\newcommand{\bn}{\boldsymbol{n}}
\newcommand{\bdelta}{\boldsymbol{\delta}}
\newcommand{\bGamma}{\boldsymbol{\Gamma}}
\newcommand{\bomega}{\boldsymbol{\omega}}
\newcommand{\bOmega}{\boldsymbol{\Omega}}
\newcommand{\bPsi}{\boldsymbol{\Psi}}
\newcommand{\bO}{\boldsymbol{O}}
\newcommand{\bnu}{\boldsymbol{\nu}}
\newcommand{\bdS}{\boldsymbol{dS}}
\newcommand{\bP}{\boldsymbol{P}}
\newcommand{\bq}{\boldsymbol{q}}
\newcommand{\br}{\boldsymbol{r}}
\newcommand{\bR}{\boldsymbol{R}}
\newcommand{\bS}{\boldsymbol{S}}
\newcommand{\bU}{\boldsymbol{U}}
\newcommand{\bu}{\boldsymbol{u}}
\newcommand{\bv}{\boldsymbol{v}}
\newcommand{\bw}{\boldsymbol{w}}
\newcommand{\bx}{\boldsymbol{x}}
\newcommand{\by}{\boldsymbol{y}}
\newcommand{\bY}{\boldsymbol{Y}}
\newcommand{\bz}{\boldsymbol{z}}
\newcommand{\bzero}{\boldsymbol{0}}

\newcommand{\arrows}{~\rightleftharpoons~}
\newcommand{\arrowstab}{\!\!\!\rightleftharpoons\!\!\!}
\newcommand{\longline}{\noindent\rule[-0.1in]{\textwidth}{0.01in}}

\def\water{H$_2$O}
\def\calcium{Ca$^{2+}$}
\def\copper{Cu$^{2+}$}
\def\magnesium{Mg$^{2+}$}
\def\sodium{Na$^+$}
\def\potassium{K$^+$}
\def\uranium{UO$_2^{2+}$}
\def\hion{H$^+$}
\def\bicarbonate{HCO$_3^-$}
\def\cotwo{CO$_2$}
\def\chloride{Cl$^-$}
\def\fluoride{F$^-$}
\def\phosphoricacid{HPO$_4^{2-}$}
\def\nitrate{NO$_3^-$}
\def\sulfate{SO$_4^{2-}$}
\def\souotwooh{$>$SOUO$_2$OH}
\def\sohuotwocothree{$>$SOHUO$_2$CO$_3$}
\def\soh{$>$SOH}

%\renewcommand{\thepage}{\roman{page}}
%\renewcommand{\thepage}{\arabic{page}}
%\renewcommand{\theequation}{\arabic{section}.\arabic{subsection}-\arabic{equation}}
%\renewcommand{\theequation}{\arabic{section}-\arabic{equation}}
%\setcounter{page}{1}

\setlength{\parindent}{0.3125in}
\setlength{\parskip}{2ex plus 0.2ex minus 0.2ex}

\setcounter{secnumdepth}{4}
\setcounter{tocdepth}{4}
\renewcommand{\contentsname}{CONTENTS}

\setlongtables

\pagestyle{fancy}

\thispagestyle{empty}

\begin{document}

\section{Modeling CO$_2$ Sequestration}

\subsection{Governing Equations}

Mass conservation equations employed by PFLOTRAN can be written in the form
\EQ\label{massconv}
\sum_\a \left\{\frac{\p}{\p t} \big(\varphi s_\a \rho_\a X_i^\a\big) + \bnabla\cdot \big[\bq_\a \rho_\a X_i^\a -\varphi s_\a \rho_\a D_\a \bnabla X_i^\a\big]\right\} \eq Q_i,
\EN
with porosity $\varphi$, molar density $\rho_\a$, diffusivity/dispersivity $D_\a$ and source/sink term $Q_i$. The sum over $\a$ is over all fluid phases present in a particular control volume. In these equations $s_\a$ denotes the volume fraction of phase $\a$ satisfying
\EQ
\sum_\a s_\a \eq 1.
\EN
The mole fraction $X_i^\a$ satisfies the constraint condition
\EQ\label{molefrac}
\sum_i X_i^\a \eq 1.
\EN
The Darcy velocity $\bq_\a$ is given as
\EQ
\bq_\a \eq -\frac{kk_\a}{\mu_\a} \bnabla \Big(P_\a - \rho_\a g z\Big),
\EN
with viscosity $\mu_\a$, acceleration of gravity $g$, and height $z$.

Summing Eqns.\eqref{massconv} over all components $i$ making use of Eqn.\eqref{molefrac} gives
\EQ\label{masstot}
\sum_\a \left\{\frac{\p}{\p t} \big(\varphi s_\a \rho_\a \big) + \bnabla\cdot \big(\bq_\a \rho_\a \big)\right\} \eq Q,
\EN
where
\EQ
Q \eq \sum_i Q_i.
\EN
The equation for $i$ = H$_2$O in Eqn.\eqref{massconv} may be replaced with Eqn.\eqref{masstot}, for example, thereby eliminating the need to consider the diffusion term for H$_2$O.

The energy conservation equation can be written in the form
\EQ
\sum_\a\left\{\frac{\p}{\p t} \big(\varphi s_\a \rho_\a U_\a\big) + \bnabla\cdot\big(\bq_\a \rho_\a H_\a\big) \right\} + \frac{\p}{\p t} \big(\rho_r C_p T \big) - \bnabla\cdot\big(\kappa\bnabla T\big) \eq Q_e,
\EN
as the sum of contributions from each fluid phase and rock,
with internal energy $U_\a$ and enthalpy $H_\a$ of fluid phase $\a$, and rock heat capacity $C_p$ and thermal conductivity $\kappa$. The quantity $Q_e$ denotes the heat source/sink term.

\subsection{Choice of Independent Variables}

Depending on the number and type of phases present within a control volume, the independent variables are chosen according to Table~\ref{tindepvar}. There exist $2^2-1=3$ possible phases in the system corresponding to liquid, gas and liquid-gas. In general for a system made up of $N_p$ individual phases, there are $2^{N_p}-1$ possible phase combinations.

\tabcolsep 3pt
\begin{table}[H]\centering
\caption{Independent variables in a two-phase system.}\label{tindepvar}
\vspace{3mm}
\begin{tabular}{lccc}
\toprule
Phases & \multicolumn{3}{c}{Variables}\\
\midrule
liquid & $P_l$ & $T$ & $X_i^l$\\
gas & $P_g$ & $T$ & $X_i^g$\\
two-phase & $P_g$ & $T$ & $s_g$\\
\bottomrule
\end{tabular}
\end{table}

\subsection{Constitutive Relations}

Various constitutive relations hold depending on the phases present.

\subsubsection{Liquid Phase: Variables $\{P_l,\,T,\,X_\c^l\}$}

For a pure liquid phase $s_l=1$. The molality of dissolved $\c$ is found from the relation
\EQ
m_\c^{l} \eq \frac{X_\c^l W_\w^{-1}}{1- X_\c^l - \displaystyle\sum_{i\ne \w,\c} \!\!\!\!\!\!X_i^l}.
\EN
with the inverse relation
\EQ
X_\c^l \eq \frac{m_\c}{ W_\w^{-1} + m_\c + \displaystyle\sum_{i\ne \w,\c}\!\!\! \!\!\! m_i}.
\EN

\subsubsection{Gas Phase: Variables $\{P_g,\,T,\,X_\c^g\}$}

For a pure gas phase $s_g=1$.

\subsubsection{Two-Phase: Variables $\{P_g,\,T,\,s_g\}$}

In a two-phase system
$X_\c^g$, is assumed to be given by
\EQ
X_\c^g \eq \frac{P_\c^g}{P_g} \eq \frac{P_g-P_\w^{\rm sat}(T)}{P_g} \eq 1-\frac{P_\w^{\rm sat}(T)}{P_g},
\EN
with total gas pressure $P_g$ equal to
\EQ\label{totalp}
P_g \eq P_\c^g + P_\w^{\rm sat}(T),
\EN
and where $P_\w^{\rm sat}(T)$ denotes the saturation pressure of pure water. One also has
\EQ
X_\c^l = 1-X_\c^g. 
\EN
The molality of dissolved $\c$ is found from the relation
\EQ
m_\c^{} \eq \frac{\phi_\c^{} X_\c^g}{\gamma_\c^{}K_\c^{}} P_g^{}.
\EN

\subsection{Conditions for Change in Phase}

Possible changes in phase that may occur in the system are listed in Table~\ref{tphasechng}. In Table~\ref{tphasechng}, $\mu_i^\a$ refers to the chemical potential of the $i$th species in phase $\a$.

\begin{table}[H]\centering
\caption{Independent variables in a two-phase system.}\label{tphasechng}
\vspace{3mm}
\begin{tabular}{rclcrcl}
\toprule
\multicolumn{3}{c}{Phase Transformation} & Condition & \multicolumn{3}{c}{Variable Switching}\\
\midrule
liquid &$\rightarrow$& two-phase & $\mu_\c^l > \mu_\c^g$ & $\{P_l, \, T, X_\c^l\}$ &$\rightarrow$& $\{P_g, \, T, s_g\}$\\
gas &$\rightarrow$& two-phase & $\mu_\c^l < \mu_\c^g$ & $\{P_g, \, T, X_\c^g\}$ &$\rightarrow$& $\{P_g, \, T, s_g\}$\\
two-phase &$\rightarrow$& liquid & $s_g < 0$ & $\{P_g, \, T, s_g\}$ &$\rightarrow$& $\{P_g, \, T, X_\c^l\}$\\
two-phase &$\rightarrow$& gas & $s_g > 0$ & $\{P_g, \, T, s_g\}$ &$\rightarrow$& $\{P_g, \, T, X_\c^g\}$\\
\bottomrule
\end{tabular}
\end{table}

\section{Finite Volume Discretization}

Using a finite volume discretization approach, the governing flow and transport equations are discretized according to
\BA
R_{in}^{k+1} \eq &\sum_\a \left\{\frac{\big(\varphi s_\a \rho_\a X_i^\a\big)_n^{k+1}-(\varphi s_\a \rho_\a X_i^\a)_n^k}{\Delta t} V_n \right. \nonumber\\
&\left. + \sum_{n'} \left[\bq_{\a,nn'}^{k+1} \rho_{\a,nn'}^{k+1} X_{inn'}^{\a,k+1} -(\varphi s_{\a} \rho_\a D_\a)_{nn'}^{k+1} \frac{X_{i n'}^{\a, k+1}-X_{i n}^{\a,k+1}}{d_{n'}+d_n}\right] A_{nn'}\right\} \nonumber\\
&- Q_i^{k+1} V_n,
\EA
providing the residual function for the $i$th component at the $n$th control volume and $k+1$st time step, where the sum over $n'$ is over all control volumes which connect to the $n$th control volume, $d_n$, $d_{n'}$ refers to the distance for the control volume center to the interface, and $A_{nn'}$ denotes the interfacial area.

\section{Flash Method}

An alternative approach to variable switching is the flash method.

\section*{Appendix: Supercritical CO$_2$--H$_2$O Equilibrium Relations}

This section follows Duan \& Sun (2003).
For an aqueous fluid in equilibrium with supercritical CO$_2$ it is necessary to use an equation of state for CO$_2$ to obtain the solubility of CO$_2$ in solution. The presentation follows Duan and Sun (2003). Equilibrium of supercritical CO$_2$ with an aqueous solution described by the reaction
\EQ
{\rm CO}_2^l \arrows {\rm CO}_2^g,
\EN
implies equality of the chemical potentials
\EQ
\mu_\c^l \eq \mu_\c^g,
\EN
where
\EQ
\mu_\c^l \eq \mu_\c^{l\ominus} + \ln a_\c^{},
\EN
and
\EQ
\mu_\c^g \eq \mu_\c^{g\ominus} + \ln f_\c^{},
\EN
with standard state chemical potentials $\mu_\c^{\ominus l}$ and $\mu_\c^{\ominus g}$. The activity of $\c$ in an aqueous solution is related to its molality $m_\c$ by
\EQ
a_\c^{} \eq \gamma_\c^{} m_\c^{},
\EN
where $\gamma_\c$ denotes the activity coefficient for aqueous $\c$.
The fugacity is given by
\EQ
f_\c^{} \eq \phi_\c^{} P_\c^{} \eq \phi_\c^{} X_\c^g P_g^{},
\EN
with fugacity coefficient $\phi_\c$ where the mole fraction of $\c$ in the supercritical (gas) phase, $X_\c^g$, is assumed to be given by
\EQ
X_\c^g \eq \frac{P_\c^g}{P_g} \eq \frac{P_g-P_\w^{\rm sat}(T)}{P_g} \eq 1-\frac{P_\w^{\rm sat}(T)}{P_g},
\EN
with total gas pressure $P_g$ equal to
\EQ\label{totalp}
P_g \eq P_\c^g + P_\w^{\rm sat}(T),
\EN
and where $P_\w^{\rm sat}(T)$ denotes the saturation pressure of pure water. 

Introducing the equilibrium constant $K_\c$ defined as
\EQ
\ln K_\c \eq -\frac{1}{RT}\big(\mu_{\rm CO_2}^{g\ominus} - \mu_{\rm CO_2}^{l\ominus}\big),
\EN
yields the mass action equation
\EQ\label{massactco2}
K_\c^{} \eq \frac{f_\c^{}}{a_\c^{}} \eq \frac{\phi_\c^{} P_\c^g}{a_\c} \eq \frac{\phi_\c^{} X_\c^g P_g^{}}{a_\c^{}}.
\EN
The equilibrium CO$_2$ molality is thus given by
\EQ
m_\c^{} \eq \frac{\phi_\c^{} X_\c^g}{\gamma_\c^{}K_\c^{}}  P_g^{}.
\EN
%Of the three quantities: $P_g$, $P_\c^g$, and $T$, two may be specified and the other computed from Eqn.\eqref{totalp}.

Solving Eqn.\eqref{massactco2} for the $\c$ partial pressure gives
\BA
P_\c^g &\eq \frac{K_\c}{\phi_\c} a_\c,\\
&\eq \widetilde K_\c a_\c,
\EA
where the effective equilibrium constant $\widetilde K_\c$ is defined as
\EQ
\widetilde K_\c \eq \frac{K_\c}{\phi_\c}.
\EN
For a two-component system molality is related to the mole fraction by the equation
\begin{subequations}
\EQ
x_\c^l \eq\frac{m_\c W_\w}{1+m_\c W_\w},
\EN
and conversely mole fraction is related to molality by the equation
\EQ
m_\c \eq \frac{x_\c^l}{W_\w(1-x_\c^l)},
\EN
\end{subequations}
where $W_\w$ denotes the formula weight of water. 

The concentration of $\c$ in the gas phase, $C_\c^g$, is obtained as
\BA
C_\c^g &\eq \frac{n_i^g}{V_g} \eq \frac{n_i^g}{N_g}\frac{N_g}{V_g} \eq \rho_\c^{} X_\c^g,\\
&\eq \rho_\c \frac{f_\c}{\phi_\c P_g},\\
&\eq \rho_\c K_\c \frac{a_\c}{\phi_\c P_g}.
\EA

\end{document}
